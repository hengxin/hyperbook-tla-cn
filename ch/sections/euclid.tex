%try
\newpage
\ctindex{1}{Euclid's Algorithm}{euclid-algorithm}%
\vspace{-\baselineskip}
 \btarget{main:euclid}%
\section{Euclid's Algorithm} \xlabel{sec:euclid}

Euclid's algorithm is a classic algorithm 
  \ctindex{1}{gcd@gcd\icmd{target}{gcdtarget}}{gcd}%
for computing the greatest common divisor (abbreviated \emph{gcd}) of
two positive integers.  We consider a simpler and much less efficient
version than the one described by Euclid in his \emph{Elements}.
However, before writing an algorithm to compute the gcd, we should
define precisely what the gcd is.

\rref{math}{predicate-logic}{If you are not familiar with the
quantifiers $\A$ and $\E$, detour here.}

  \ctindex{2}{gcd@gcd\icmd{target}{gcdtarget}}{gcd}%
  \ctindex{1}{greatest common divisor|see{\icmd{lref}{gcdtarget}{gcd}}}{g-c-d}%
  \vspace{-\baselineskip}
\subsection{The Greatest Common Divisor} \xlabel{main:euclid:gcd}

We want to define an operator $GCD$ such that $GCD(m, n)$ equals the
gcd of $m$ and $n$, for numbers $m$ and $n$.  Negative numbers and the
number 0 were unknown to Euclid, so let's assume that $m$ and $n$ are
positive integers.  (The gcd of $m$ and $n$ is undefined if either of
them equals~0.)  Since we might want to use this operator in some
specification other than that of Euclid's algorithm, the instinct of
any good engineer is to put the definition into a separate module so
it can be re-used.
  \marginpar{\popref{reuse}{Why we usually don't re-use 
     specifications in practice.}}
So, let's create a spec to contain the definition of $GCD$ and any
other related definitions and properties we might need.  

In the Toolbox, open a new specification called $GCD$.  (\tlaplus\
allows the use of the same name for both a module and a defined
operator.)  
You can make it easier to use the module in other specifications
by putting it in a separate library folder.  Library folders
are explained on the
  \helppage{gettingstarted/tla-preferences}{help page} 
for the \tlaplus\
  \helppage{gettingstarted/preferences}{preferences page}.

We'll need the usual operations on integers, so we import
them by beginning the module with the statement:
\begin{display}
\begin{twocols}
$\EXTENDS Integers$
 \midcol
\verb|EXTENDS Integers|
\end{twocols}
\end{display}
% The gcd of two positive integers is the largest integer that divides
% both of them.  To define it, we must first define what it means for
% one positive integer to divide another.  However, before doing that,
% let's make the spec a little easier to read by separating the part of
% the spec up to now from the following part, which defines the gcd.  We
% do this by adding a line of dash (\texttt{-}) characters, which is
% purely decorative:
% \begin{display}
% \verb|-------------------------------------------|
% \end{display}
% The line must contain at least four dashes.  See how that is
% %\lref{pretty-printing}{pretty-printed}.
% \sref{main}{pretty-printing}{pretty-printed}.


\tindex{1}{divisor}%
\vspace{-\baselineskip}%
\subsubsection{Divisors}

We define the operator $Divides$ so that $Divides(p,\,n)$ equals
$\TRUE$ if the integer $p$ divides the integer $n$, and equals
$\FALSE$ if it doesn't.  You learned in grade school that $p$ divides
$n$ \popref{iff}{iff} $n/p$ is an integer.  The $Integers$ module
defines $Int$ to be the set of all integers.  So, an obvious
definition of $Divides$ is
\begin{display}
\begin{twocols}
$Divides(p, n)== n/p \in Int$
\midcol
\verb|Divides(p, n) == n/p \in Int|
\end{twocols}
\end{display}
However, if we use this definition, the Toolbox reports an error
because it can't find the definition of the operator $/$\,.

The $Integers$ module is about integers, and $n/p$ is not, in general,
an integer.  The only arithmetic operations you learned in grade
school that the module defines are addition ($+$), subtraction ($-$),
multiplication ($*$), and exponentiation ($a^{b}$ is
typed~\verb|a^b|).  There is a 
  \ctindex{1}{Reals module@\mmath{Reals} module}{reals-module}%
$Reals$ module that defines ordinary division, but it is rarely used
because the TLC model checker can't evaluate the operator $/$\,.  So,
we define $Divides$ using the operators defined in the $Integers$
module.

The definition is simple.  An integer $p$ divides an integer $n$ iff
$n$ equals $q*p$ for some integer $q$.  We can therefore define $Divides$
by
\begin{twocols}
$Divides(p, n) == \E\, q \in Int : n = q*p$
\midcol
\verb|Divides(p, n) == \E q \in Int : n = q * p|
\end{twocols}
Add this definition and save the module.

Let's test our definition.  \popref{create-new-model}{Create a new TLC
model}.  In it, use TLC to 
  \marginpar{\popref{evaluate-constant-expression}{How 
   to use TLC to evaluate a constant expression.}}%
evaluate the expression \tlabox{Divides(2,\,4)}.  This produces an
error message that looks like:
\begin{display}
\tt TLC
  \ctindex{1}{non-enumerable quantifier bound}{non-enum-q-bound}%
encountered a non-enumerable quantifier bound\\
Int.\\
\underline{line 4, col 27 to line 4, col 29 of module GCD}
\end{display}
Clicking on the location in the message takes you to the
$Int$ in the definition.

TLC evaluates an expression of the form $\E\,x\in S:exp$ by computing
all the elements in the set $S$ and evaluating $exp$ for each of those
values.  It obviously can't do this if $S$ is an infinite set like
$Int$.  

We don't have to try all integers $q$ to see if there is one
satisfying $n=q*p$.  Since we're concerned only with positive
integers, it's enough to try all integers between $1$ and $n$.  So,
we could define $Divides$ by
  \[ Divides(p, n) == \E\, q \in 1\dd n: n = q*p\]
%
% Alternatively, we could use the 
%   \rref{math}{math:modulus}{modulus operator $\%$},
% (defined in the $Integers$ module so that $a\,\%\, b$ is the remainder
% when $a$ is divided by $b$).  We could then define $Divides$ by
%  \[ Divides(p, n) == n \,\%\, p = 0
%  \]
% However, I prefer the original definition; I find it more elegant.  
%
A principal goal of TLC is that it should not be necessary to modify a
spec in order to model-check it.  Instead, we let the model tell TLC
to override the definition of $Int$,
   \marginpar{\popref{definition-override}{How to override a definition
   in TLC.}}%
redefining it to equal some finite set of numbers.  Have the model
redefine $Int$ to equal $-1000\dd 1000$, and run TLC again.  This time,
TLC's evaluation of \tlabox{Divides(2,\,4)} obtains the value $\TRUE$.
Check that TLC calculates \tlabox{Divides(2,\,5)} to equal
$\FALSE$.  

% \pause
% 
% \noindent
%   \ctindex{1}{+2g@\icmd{vbar}}{+2g}% 
% Mathematicians usually write $p|n$ instead of $Divides(p,\,n)$.
% \tlaplus\ allows the use of $\,|\,$ as an infix operator, so you
% can define $p|n$ to have the same meaning as $Divides(p,\,n)$ by
% adding the following definition to the module:
% \begin{twocols}
% $p\,|\, n== \E\, q \in 0\dd n: n = q*p$
% \midcol
% \verb+p | n == \E q \in 0..n : n = q * p+
% \end{twocols}
% You can replace any use of $Divides$ with a use of $\,|\,$.
% \rref{user-definable-ops}{}{Click here} for a list of all the infix
% operator symbols that \tlaplus\ provides.
% 
% \pause 

The gcd of $m$ and $n$ is the largest divisor of both $m$
and $n$.  In other words, it is the maximum of the set of divisors of
both $m$ and $n$.  To write this definition mathematically, we first
define the set of divisors of a number and the maximum of a set of
numbers.  The set $DivisorsOf(n)$
of divisors of an integer $n$ is obviously:%
 \marginpar[3]{\rref{math}{set-constructors}{Recall
that $\{x \in S : P(x)\}$ is the
     subset of $S$ consisting of all its elements $x$
     satisfying $P(x)$.}}%
\begin{twocols} 
$DivisorsOf(n) == \{p \in Int : Divides(p, n)\}$
\midcol
\verb|DivisorsOf(n) == {p \in Int : Divides(p, n)}|
\end{twocols}
Add this definition to module $GCD$ and have TLC evaluate 
$DivisorsOf(493)$.  It should obtain
  $ \{-493,\, -29,\, -17,\, -1,\, 1,\, 17,\, 29,\, 493\}
  $.

  \ctindex{1}{choose@\icmd{textsc}{choose}}{choose}%
  \vspace{-\baselineskip}%
\subsubsection{{\sc choose} and the Maximum of a Set} \xlabel{set-max}

To define the maximum of a set of numbers, we need to introduce
the \tlaplus\ \textsc{choose} operator.  The expression
 \[ \CHOOSE x \in S : P(x) \]
equals some value $v$ in $S$ such that $P(v)$ equals \TRUE, if such
a value exists.  Its value
%
% \marginpar{\popref{choose----nondeterministic}{Is \textsc{choose} 
%    nondeterministic?}}
%
is unspecified if no such $v$ exists.  For example, if we define
 \[ Foo == \CHOOSE i \in Int : i^2 = 4\]
then $Foo$ equals either $2$ or $-2$, since these are the two elements
of $Int$ whose square equals~4.  The semantics of \tlaplus\ do not say
which of those two values $Foo$ equals.  We have absolutely no idea
what the value of this expression is:
 \[ \CHOOSE i \in Int : i^2 = -4\]
since there is no integer whose square equals $-4$.%
%
\marginpar{\rref{math}{choose}{Learn more about \textsc{choose} here.}}

Using \textsc{choose}, it's easy to define 
  \ctindex{1}{maximum!of set of numbers}{max-of-set-if-numbers}%
the maximum of a set $S$ of
numbers.  The maximum of $S$ is an element of $S$ that is
greater than or equal to every element of $S$:%
 \ctindex{1}{+5e@\mmath{\icmd{geq}} (greater than or equal)}{+5e}%
\begin{twocols}
$SetMax(S) ==$ \\
\s{1.5}$\CHOOSE i \in S : \A\, j \in S : i \geq j$
\midcol
\verb*|SetMax(S) == | \\
\verb*|    CHOOSE i \in S : \A j \in S : i >= j| 
\end{twocols}
Note that $\geq$ is typed \verb|>=|\,.  It can also be
typed \verb|\geq|\,.  Add this definition to module $GCD$ and
check it by having the Toolbox evaluate the expression
$SetMax(DivisorsOf(493))$, which should of course equal~493.  

    \ctindex{3}{gcd@gcd\icmd{target}{gcdtarget}}{gcd}%
    \vspace{-\baselineskip}%
\subsubsection{The GCD Operator} \xlabel{main:gcd-operator}

The gcd of two positive integers $m$ and $n$ is the maximum of the set
of all numbers that are divisors of both of them.  That set is just
the intersection
 \marginpar{\rref{math}{simple-setops}{If you are not familiar with the set
            operator $\cap$, detour here}.}
 of their two sets of
divisors.  We can therefore define:\target{gcd-detour-return}
\begin{twocols}
$\begin{noj}
 GCD(m, n) == \\ \s{2}SetMax(DivisorsOf(m) \cap DivisorsOf(n))
\end{noj}$
\midcol
\verb*|GCD(m, n) == | \\
\verb*|   SetMax(DivisorsOf(m) \cap DivisorsOf(n))|
\end{twocols}
Add this definition to module $GCD$ and check that it's correct
by evaluating $GCD$ for some numbers.  You will find
that TLC can quickly evaluate the gcd of pairs of numbers less than
1000.  
\begin{aquestion}{finding-gcds}
How can you easily find pairs of numbers whose gcd you know
in order to test the definition?
\end{aquestion}
This sort of testing will not satisfy a mathematician, but it's
good enough for engineers.  It checks that we haven't made a gross
error, such as misspelling something or writing $\cup$ instead of
$\cap$.  The only plausible source of error is missing a subtle corner
case.  We are claiming that this is the correct definition of
$GCD(m,\,n)$ only if $m$ and $n$ are positive integers, so obvious
corner cases are (i)~if one or both of them equals 1 and (ii)~if
they are equal.  A little thought reveals that there is nothing
exceptional about these cases.  However, it's a good idea to test them
anyway.

%try

  \tindex{1}{comments}%
  \vspace{-\baselineskip}%
\btarget{main:comments}%
\subsection{Comments} 

Mathematics is precise, compact, and elegant.  But it's hard to look
at a mathematical formula and see what it's about.  For example, suppose
instead of $Divides$, $DivisorsOf$, $SetMax$, and $GCD$,
we had named our operators $A$, $B$, $C$, and $D$.  Their definitions 
would then look like this.
\begin{display}
\begin{notla}
A(p, n) == \E q \in Int : n = q * p
B(n)    == {p \in Int : A(p, n)}
C(S)    == CHOOSE i \in S : \A j \in S : i >= j
D(m, n) == C(B(m) \cap B(n))
\end{notla}
\begin{tlatex}
 \@x{ A ( p ,\, n )\@s{4.25} \.{\defeq} \E\, q \.{\in} Int \.{:} n \.{=} q * p}%
\@x{ B ( n )\@s{15.48} \.{\defeq} \{ p \.{\in} Int \.{:} A ( p ,\, n ) \}}%
 \@x{ C ( S )\@s{14.50} \.{\defeq} {\CHOOSE} i \.{\in} S \.{:} \A\, j \.{\in}
 S \.{:} i \.{\geq} j}%
\@x{ D ( m ,\, n ) \.{\defeq} C ( B ( m ) \.{\cap} B ( n ) )}%
\end{tlatex}
\end{display}
Imagine how hard it would now be to figure out what these operators mean.

Choosing explanatory names certainly helps, but it's seldom
enough to make our specifications easy to understand.  We need to add
explanatory comments---for example, as in this definition
of $Divides$.
\begin{display}
\begin{notla}
Divides(p, n) == \E q \in Int : n = q * p
\end{notla}
\begin{tlatex}
 \@x{ Divides ( p ,\, n ) \.{\defeq} \E\, q \.{\in} Int \.{:} n \.{=} q * p}%
\end{tlatex}
\par
%\vspace{-1.2\baselineskip}
% \par
\mbox{}\s{2}\Comment{For integers $p$ and $n$, 
                     equals $\TRUE$ iff $p$ divides $n$.  }
\end{display}
There are two ways to write comments in \tlaplus.  Text between
  \ctindex{1}{+4kj@\mmath{(*\ldots*)} (comment)}{+4kj}%
\verb|(*| and \verb|*)| is a comment, and all text that follows a
  \ctindex{2}{+2oj@\mmath{\icmd{backslash}*} (end-of-line comment)}{+2oj}%
\verb|\*| on the same line is a comment.  Thus, the comment above
following the definition of $Divides$ can be written in either of
the following two ways:
\begin{display}
\verb|(* For integers p and n, equals   |\\
\verb|   TRUE iff p divides n. *)| \V{.6}
\verb|\* For integers p and n, equals TRUE iff p divides n.|
\end{display}
Comments can be nested within one another, as in
\begin{display}
\verb|(* This is all (* commented *) text *)|
\end{display}
Nesting comments is useful for commenting out parts of a specification
during testing, but don't do it in actual comments.  The
\sref{main}{pretty-printing}{pretty-printer} ignores comments inside
comments.  The one exception is that comments inside a PlusCal
algorithm are handled properly, even though the algorithm appears
inside a comment.

I like to make comments more visible in the \textsc{ascii} version
by 
  \ctindex{1}{comments!boxed}{comments-boxed}%
boxing them like this:
\begin{display}
\verb|(********************************)|\\
\verb|(* For integers p and n, equals *)|\\
\verb|(* TRUE iff p divides n.        *)|\\
\verb|(********************************)|
\end{display}
The Toolbox provides commands for writing boxed comments.  They are
described in the \emph{Editing Comments} section of the
\helppage{spec/editing-modules}{\emph{Editing Modules} help page}.


The pretty-printer handles boxed comments properly---even if
you write something like 
this.\marginpar{\popref{box-comment-example}{Give it a try.}}
\begin{widedisplay}
\begin{verbatim}
Divides(p, n) ==              (**********************************)
      \E q \in Int :          (* For integers p and n, equals   *)
              n = q * p       (* TRUE iff p divides n -- which  *)
                              (* I think is really neat; don't  *)
                              (* you?                           *)
                              (**********************************)
\end{verbatim}
\end{widedisplay}
The pretty-printer generally does a reasonably good job of formatting
the comments.  However, if you want nicely
    \ctindex{1}{comments!pretty printed}{comments-pretty-printed}%
    \ctindex{1}{pretty printing!comments}{pretty-printing-comments}%
printed comments for others to
read, you will have to help it.  To find out how, see the Toolbox's
\helppage{spec/help-print}{\emph{Helping the Pretty-Printer} help page}.

Because I explain the specifications in the text as I present them, I
will usually omit comments in this hyperbook.  You should not omit
comments from your specs.  Unless you're going to stand next to all
the readers of your spec as they read it, and you can project yourself
into the future to explain the spec to yourself when you read it a
year later, include extensive comments.  Every definition and the
purpose of every declared variable should be explained in a comment.

Comments are especially important in \tlaplus\ because it is untyped.
In a typed language, you would have to declare that the arguments of
$Divides$ are integers and its value is a Boolean.  The absence of
type declarations makes the definition shorter and mathematically
simpler.  However, it imposes on us the responsibility%
 \marginpar{\popref{what-divides-means}{What does $Divides(p,n)$ 
 mean if $p$ and $n$ are not integers---or not even numbers?}}
of telling the reader that we expect the arguments to be integers.
(It's pretty obvious in this case that the value of
$Divides(p,n)$ is a Boolean.)

Text that comes in the file before or after the module is ignored; it
can be used to record any information about the spec that you don't
want to put in comments within it.  The pretty-printer does output
this text, but it might not do a very good job of formatting it.

% \medskip
% 
% You can also make your modules easier to read by adding ``section
% separators'' consisting of four or more dashes (\texttt{-}
% characters), as in:
% \medskip
% \begin{display}
% \begin{minipage}{.8\textwidth}
% \begin{verbatim}
% SetMax(S) ==  CHOOSE i \in S : \A j \in S : i >= j
% -----------------------------------------------------------
% Divides(p, n) == \E q \in Int : n = p * q
% \end{verbatim}
% \end{minipage}
% \end{display}
% %
% \medskip
% %
% Here's how it is printed:
% \medskip
% \begin{display}
% \begin{minipage}{.8\textwidth}
% \begin{notla}
% SetMax(S) ==  CHOOSE i \in S : \A j \in S : i >= j
% -------------------------------------------------
% Divides(p, n) == \E q \in Int : n = p * q
% \end{notla}
% \begin{tlatex}
%  \@x{ SetMax ( S ) \.{\defeq}\@s{4.1} {\CHOOSE} i \.{\in} S \.{:} \A\, j
%  \.{\in} S \.{:} i \.{\geq} j}%
% \@x{}\midbar\@xx{}%
%  \@x{ Divides ( p ,\, n ) \.{\defeq} \E\, q \.{\in} Int \.{:} n \.{=} p \.{*}
%  q}%
% \end{tlatex}
% \end{minipage}
% \end{display}
% 
%try

\subsection{The Algorithm} \label{sec:euclid}\xlabel{main:euclid-alg}


Let the positive integers whose gcd we are computing be $M$ and $N$.
Euclid's algorithm uses two variables, which we call $x$ and $y$.  It
can be described informally as follows.
\begin{itemize}
\item Start with $x$ equal to $M$ and $y$ equal to $N$.

\item Keep subtracting the smaller of $x$ and $y$ from the larger 
one, until $x$ and $y$ are equal.

\item When $x$ and $y$ are equal, they equal the gcd of $M$ and $N$.
\end{itemize}
We represent the algorithm in the standard model, describing it 
in PlusCal.

Open the Toolbox and open a new spec with root module $Euclid$.  We'll
want to use the definition of $GCD$, so we want to import it with an
\textsc{extends} statement.  Since the $GCD$ module extends the
$Integers$ module, the \textsc{extends} statement will also import the
$Integers$ module.  However, I think the spec is easier to understand
if it explicitly includes $Integers$ in the \textsc{extends}
statement, even if it is redundant.  So, we begin the module with
\begin{twocols}
$\EXTENDS Integers,\ GCD$
 \midcol
\verb|EXTENDS Integers, GCD|
\end{twocols}
We need to declare $M$ and $N$, which we do by writing.%
  \ctindex{1}{constant@\icmd{textsc}{constant}}{constant}%
  \ctindex{1}{constants@\icmd{textsc}{constants}}{constants}%
  \tindex{1}{constant declaration}%
\marginpar[-1]{The keywords \textsc{constant} and \textsc{constants}
           are equivalent.}%
\begin{twocols} 
$\CONSTANTS M,\, N$%
 \midcol 
\tt CONSTANTS M, N 
\end{twocols}
This declares $M$ and $N$ to be
unspecified constants---unspecified because we are saying nothing
about their values, and constants because their values do not change
during the course of a behavior.

We don't want the values of $M$ and $N$ to be totally unspecified; we
want them to be positive integers.  To assert this assumption, we must
express the set of positive integers in \tlaplus.  The $Integers$
module defines
  \ctindex{1}{Nat@\mmath{Nat}}{Nat}%
$Nat$ to be the set of all natural numbers
(non-negative integers).  The set of positive integers is the set of
all natural numbers except 0, which can be written 
with the \popref{set-difference}{set difference operator}
  \tindex{3}{set difference}%
  \ctindex{3}{+2o@\mmath{\icmd{backslash}} (set difference)}{2oa}%
$:\:$ as 
$Nat :\: \{0\}$.
Our assumption about $M$ and $N$ can therefore be written as follows:%
  \ctindex{1}{assume (TLA+ statement)@\icmd{textsc}{assume} (\icmd{tlaplus} statement)}{assume}%
\begin{twocols}
$\ASSUME \; 
     \begin{conj}
        M \in Nat :\: \{0\} \\ N \in Nat :\: \{0\}
     \end{conj}$
\midcol
\begin{verbatim*}
ASSUME /\ M \in Nat \ {0}
       /\ N \in Nat \ {0}
\end{verbatim*}
\end{twocols}
\begin{aquestion}{euclid-assump}
Use set notation to write this assumption more compactly.
\end{aquestion}
\begin{aquestion}{writing-pos-ints}
How many other ways can you write the set of positive integers
in \tlaplus?
\end{aquestion}
%
As always, the algorithm appears inside a multi-line comment,
beginning with the keyword \verb|--algorithm| and followed by the name
and an opening \texttt{\bf\{}\,.  Let's name the algorithm $Euclid$.
\begin{display}
\begin{verbatim}
(****************************************************
--algorithm Euclid {

}
*****************************************************)
\end{verbatim}
\end{display}
The algorithm uses the two
variables $x$ and $y$, initially equal to $M$ and $N$,
respectively.
\begin{twocols}
\textbf{variables} $x = M$, $y = N$ ;
\midcol
\verb*| variables x = M, y = N ;|
\end{twocols}
This is followed by the body of the algorithm, enclosed
in curly braces.

Euclid's algorithm works by continually subtracting the smaller of $x$
and $y$ from the larger, stopping when $x$ equals $y$.  If you have
used an ordinary programming language, you will probably understand
this code, which follows the variable declaration.%
    \ctindex{1}{while (PlusCal statement)@\icmd{textbf}{while} (PlusCal statement)}{while}%
\begin{twocols}% [.75]
\begin{tabbing}
\{ \= \textbf{while} $(x # y)$ % \`\mbox{}\\
   %\> \s{1} \= 
    \= \{ \= \textbf{if} $(x < y)$ \= \{$y := y - x$\}\s{3}\\
   \>      \>    \> \textbf{else} \>  \{$x := x-y$\} \\
   \>      \> \}\\
\}
\end{tabbing}
\midcol
\begin{verbatim*}
 { while (x # y) { if (x < y) { y := y - x }
                   else       { x := x - y }
                 };
 }
\end{verbatim*}
\end{twocols}
If you don't understand the code, be patient.  We'll soon see exactly
what it means.

Having finished the algorithm, 
% 
% we now add the following two one-line
% comments after the multi-line comment that contains the algorithm.
% \begin{display}
% \begin{verbatim}
% \* BEGIN TRANSLATION
% \* END TRANSLATION
% \end{verbatim}
% \end{display}
% Next, 
% 
you must run the translator to compile it to a \tlaplus\
specification.  Do this with the \textsf{File} menu's
\textsf{Translate PlusCal Algorithm} command, or by typing
\textsf{control+t}.  The translator inserts the \tlaplus\ translation
after the end of the comment containing the algorithm, between
\verb|BEGIN| \verb|TRANSLATION| and \verb|END| \verb|TRANSLATION|
comment lines.  If the file already contains such comment lines, the
translator replaces everything between those lines with the
algorithm's translation.

\subsection{The TLAPlus Translation}

The \tlaplus\ translation describes the precise meaning of the PlusCal
algorithm.  It begins by declaring the algorithm's variables:
\begin{display}
$\VARIABLES x,\ y,\ pc$
\end{display}
The translation has added a new 
   \ctindex{1}{pc variable@\mmath{pc} variable}{pc-variable}%
variable $pc$, which is short for
  \tindex{1}{program control variable}%
\emph{program control}.  The intuitive meaning of the \textbf{while}
loop is that it continues to execute as long as $x#y$ is true.  When
that formula becomes false, the code following the \textbf{while} loop
is executed.  In the \lref{main:standard-model}{Standard Model}
underlying \tlaplus, there is no concept of code.  An execution is
represented simply as a sequence of states.  What code is being
executed must be described within the state.  In the PlusCal
translation, it is described by the value of the variable $pc$.

After declaring the variables, the translation defines the identifier
  \ctindex{1}{vars@\mmath{vars}}{vars}%
$vars$ to equal the triple of all the variables.
\begin{display}
$vars == << x,\ y,\ pc >>$
\end{display}
  \ctindex{1}{+4o@\mmath{\icmd{langle}e_{1}, \ldots , e_{n}\icmd{rangle}} (tuple)}{+4o}%
  \tindex{1}{tuple}%
In \tlaplus, tuples
are 
  \marginpar{\rref{math}{tuples}{Tuples are explained here.}}%
enclosed between angle brackets $<<$ and $>>$, which are typed 
\verb|<<| and \verb|>>|, so the definition of $vars$ is written
\begin{display}
\verb|vars == << x, y, pc >>|
\end{display}
Next comes the definition of the initial predicate.%
%
   \marginpar{I have reformatted the translation slightly to make it
   a bit easier to read.}
\begin{display}
\begin{notla}
Init == /\ x = M
        /\ y = N
        /\ pc = "Lbl_1"
\end{notla}
\begin{tlatex}
\@x{ Init \.{\defeq} \.{\land} x \.{=} M}%
\@x{\@s{35.70} \.{\land} y\@s{0.10} \.{=} N}%
\@x{\@s{35.70} \.{\land} pc \.{=}\@w{Lbl\_1}}%
\end{tlatex}
\end{display}
The variables $x$ and $y$ have the expected initial values; $pc$
initially equals the string $"Lbl\_1"$.  We shall see later what this
value means and how it was chosen.

The translation next defines $Lbl\_1$ to be the action that
describes the \popref{step}{steps} that can be taken when execution is
at the control point $"Lbl\_1"$.  Such a step represents the execution
of a single iteration of the \textbf{while} loop.%
  \marginpar{\popref{euclid-pcal2}{Here is a pop-up window with
                                 this definition.}}
% \begin{display}
% \begin{notla}
% Lbl_1 == /\  pc = "Lbl_1"
%          /\  IF x # y
%                 THEN /\ IF x < y
%                           THEN /\ y' = y - x
%                                /\ x' = x
%                           ELSE /\ x' = x - y
%                                /\ y' = y
%                      /\ pc' = "Lbl_1"
%                 ELSE /\ pc' = "Done"
%                      /\ UNCHANGED << x, y >>
% \end{notla}
% \begin{tlatex}
% \@x{ Lbl\_1 \.{\defeq} \.{\land}\@s{4.1} pc \.{=}\@w{Lbl\_1}}%
% \@x{\@s{42.84} \.{\land}\@s{4.09} {\IF} x \.{\neq} y}%
% \@x{\@s{70.21} \.{\THEN} \.{\land} {\IF} x \.{<} y}%
% \@x{\@s{120.83} \.{\THEN} \.{\land} y \.{'}\@s{0.10} \.{=} y \.{-} x}%
% \@x{\@s{152.14} \.{\land} x \.{'} \.{=} x}%
% \@x{\@s{120.83} \.{\ELSE} \.{\land} x \.{'} \.{=} x \.{-} y}%
% \@x{\@s{152.14} \.{\land} y \.{'}\@s{0.10} \.{=} y}%
% \@x{\@s{101.52} \.{\land} pc \.{'} \.{=}\@w{Lbl\_1}}%
% \@x{\@s{70.21} \.{\ELSE} \.{\land} pc \.{'} \.{=}\@w{Done}}%
% \@x{\@s{101.52} \.{\land} {\UNCHANGED} {\langle} x ,\, y {\rangle}}%
% \end{tlatex}
% \end{display}
The first conjunct of action $Lbl\_1$ has no primed variables, so it
is an 
  \tindex{1}{enabling condition}%
enabling condition.  It asserts that an $Lbl\_1$ step can occur
only when $pc$ equals $"Lbl\_1"$, meaning only when control is at the
beginning of the \textbf{while} statement.  

The second conjunct, which is an
 \popref{if-vs-if}{\textsc{if}/\textsc{then}/\textsc{else} expression},
specifies the new values of the three variables $x$, $y$, and $pc$.
Let's first look at the new value of $pc$, which is specified by the
value of $pc'$.  If $x#y$ is true, then the second conjunct of the
outermost \textsc{then} clause asserts $pc'="Lbl\_1"$.  When $x$ and
$y$ are not equal, executing one iteration of the \textbf{while}
statement leaves $pc$ equal to $"Lbl\_1"$, meaning that it leaves
control at the beginning of the \textbf{while}.  If $x#y$ is false, so
$x$ and $y$ are equal, then the first conjunct of the outermost
\textsc{else} clause asserts $pc'="Done"$, meaning that control is
after the \textbf{while} loop.

Let's now look at the new values of $x$ and $y$, which are specified
by the values of $x'$ and $y'$.  If $x#y$ is true, then these values
are specified by the first conjunct of the outermost \textsc{then}
clause, which is an $\IF$\ldots$\THEN$\ldots$\ELSE\!\!\!$ expression.  
This inner \textsc{if} expression asserts that, if $x<y$ is true, then
$x'$ equals $x$ and $y'$ equals $y-x$; otherwise $x'$ equals $x-y$ and
$y'$ equals $y$.  If $x#y$ is false (so $x$ equals $y$), then the
outermost \textsc{else} clause (of the \textsc{if}~$x#y$) asserts 
 $\UNCHANGED <<x, y>>$.
The built-in \tlaplus\ operator 
  \ctindex{1}{UNCHANGED@\icmd{textsc}{unchanged}}{unchanged}%
\textsc{unchanged} is defined by
 \[ \UNCHANGED e == e' = e \NOTLA \target{main:unchanged-vars}\]
for any expression $e$.  
Priming an expression $e$
means priming all the variables in $e$ (after
fully expanding the definitions of all symbols that occur in $e$).
We therefore have%
 \[ \NOTLA \newcommand{\pb}[1]{\parbox[t]{.6\textwidth}{#1}} 
\TLA
\begin{noj4}
    \UNCHANGED <<x, y>> \;
         & \Leftrightarrow \; &  <<x, y>>' =  <<x, y>> &
    \mbox{By definition of \textsc{unchanged}.\vs{.4}} \\
         & \Leftrightarrow  & <<x', y'>> =  <<x, y>> &
     \pb{By definition of priming an expression.\vs{.4}} \\
        & \Leftrightarrow  & (x' = x) /\ (y' = y) &
     \pb{Because two ordered pairs are equal iff
       their corresponding elements are equal.\vs{.2}}
    \end{noj4}
 \]
Putting this all together, we see that action $Lbl\_1$ describes a
step that
\begin{itemize}
\item can occur only when $pc$ equals $"Lbl\_1"$.

\item if $x#y$, subtracts the smaller of $x$ and $y$ from the larger,
leaving the smaller of them and $pc$ unchanged.

\item if $x=y$, sets $pc$ to $"Done"$, leaving the values of $x$ and
$y$ unchanged.
\end{itemize}
%
The algorithm begins with $pc$ equal to $"Lbl\_1"$.  As long as $x#y$,
it can execute $Lbl\_1$ steps that leave $pc$ equal to $"Lbl\_1"$ and
decrease $x$ or $y$.  If $x=y$, it can execute an $Lbl\_1$ step that
leaves $x$ and $y$ unchanged and sets $pc$ to $"Done"$.  When $pc$
equals $"Done"$, the algorithm has terminated and it can do nothing
else.  We therefore expect $Lbl\_1$ to be the algorithm's next-state
action.  However, the translation defines $Next$ to be the disjunction
of $Lbl\_1$ and another formula.  Let's forget about that other
formula for now; we'll return to it soon.

The translation then defines two temporal formulas.  A temporal
formula is a predicate on behaviors (a formula that is true or false
of a behavior).  Formula $Spec$ is defined to equal
  $Init /\ [][Next]_{vars}$, where
$vars$ is defined to be the triple $<<x, y, pc>>$ of the algorithm's
variables.  
(The formula is written in \textsc{ascii} as \verb|Init /\ [][Next]_vars|.)
We will see later that this temporal formula is true of a
behavior iff the behavior is a possible execution of the algorithm.
In other words, formula $Spec$ is the \tlaplus\ 
  \tindex{2}{behavior specification}%
  \ctindex{2}{specification!behavior}{spec-behavior}%
behavior specification of the algorithm.

The second temporal formula defined by the translation is
$Termination$.  As we will also see later, it is true of a behavior
iff the behavior eventually reaches a state in which $pc$ equals
$"Done"$.  Hence, formula $Termination$ asserts (of a behavior) that
the algorithm terminates.

\pause

\noindent
You may have remarked that the variable $pc$ did not appear in the
translations of our previous PlusCal algorithms: the one-bit clock
algorithm $Clock$ and algorithm $DieHard$.  The translator is clever
enough to realize that control is always at the same point in an 
execution of those algorithms, so $pc$ is not needed.

\subsection{Checking Safety}

Correctness of algorithm $Euclid$ means that it satisfies two
properties:
\begin{itemize}
\item If the algorithm terminates, it does so with $x$ and $y$ both equal
to $GCD(M,N)$.

\item The algorithm eventually terminates.
\end{itemize}
The first property is what is called a
   \ctindex{2}{property!safety}{safety}%
   \tindex{2}{safety property}%
safety property; 
  \marginpar{\popref{safe-live-intro}{What are safety 
      and liveness properties?}}%
the second is a
  \ctindex{2}{property!liveness}{liveness}%
  \tindex{2}{liveness property}%
liveness property.  We consider the safety property.

The algorithm has terminated iff $pc$ equals $"Done"$.  Therefore, the
safety property is equivalent to the assertion that the following
formula is an invariant of the algorithm (true in all reachable states):
  \[  (pc = "Done") => (x = y) /\ (x = GCD(M, N))
  \]
So, let's have TLC check that it is an invariant of the algorithm.

\popref{create-new-model}{Create a new TLC model} for the $Euclid$
specification.  The Toolbox reports two errors in the model, because 
the model must specify the values of the declared constants $M$ and $N$.
Double-clicking on a constant in the \textsf{What is the model?} section
of the \textsf{Model Overview} page of the model pops up a window in
which you can enter the value.  (Keep the default \textsf{Ordinary assignment}
selection.)  Set $M$ to 30 and $N$ to 18.

The Toolbox has set the model's behavior specification to the temporal
formula $Spec$.  Before checking the invariant, let's just run TLC to
make sure there is no error in the algorithm's specification.  TLC
finds no errors, and reports that there are 6 reachable states and the
diameter of the \lref{main:state-graph}{state graph} is 5.  This is what
we expect for an algorithm with a single possible behavior that
terminates after taking 5 steps.

Let's now check the invariant.  We can enter the invariant directly into
the model.  However, we might as well put the invariant in a definition
in the specification itself.  The property of an algorithm that it
terminates only with the correct result is called 
  \tindex{1}{partial correctness}%
\emph{partial correctness}, so let's add to module $Euclid$ the definition:
\begin{twocols}[.485]
\begin{notla}
PartialCorrectness ==
  (pc = "Done") => (x = y) /\ (x = GCD(M, N))
\end{notla}
\begin{tlatex}
\@x{ PartialCorrectness \.{\defeq}}%
 \@x{\@s{8.2} ( pc \.{=}\@w{Done} ) \.{\implies} ( x \.{=} y ) \.{\land} ( x
 \.{=} GCD ( M ,\, N ) )}%
\end{tlatex}
\midcol
\begin{verbatim*}
PartialCorrectness ==
  (pc = "Done") => (x = y) /\ (x = GCD(M, N))
\end{verbatim*}
\end{twocols}
Add the invariant $PartialCorrectness$ to the \textsf{Invariants} part
of the \textsf{What to check?} section of the \textsf{Model Overview}
page and run TLC\@.  This produces an error, with the not very helpful
error message
\begin{display}
\texttt{Evaluating invariant PartialCorrectness failed.}
\end{display}
The error trace shows that this error occurred when TLC was evaluating
the invariant on the last state of a complete execution.  This
is the first state TLC computed in which $pc="Done"$ equals true,
so it is the first state in which it had to compute $GCD(M,N)$ when
evaluating $PartialCorrectness$.  TLC can't evaluate $GCD(M,N)$ unless
we override the definition of $Int$ to make it a finite set.  As we did
for the $GCD$ spec, use the \textsf{Definition Override} section of
the \textsf{Advanced Options} page to have the model redefine $Int$
to equal $-1000\dd 1000$.  TLC should now find no error, verifying that
the algorithm terminated with $x$ and $y$ equal to $GCD(M,N)$.

Try changing the values of $M$ and $N$ and running TLC again.  Each
run should take a couple of seconds for values of $M$ and $N$ less
than 1000.  Since we know that Euclid's algorithm is correct, checking
a few values of $M$ and $N$ will give us confidence that our PlusCal
version is correct.  

If we didn't know that Euclid's algorithm was correct, we would need
to check it for many more values.  Instead of checking that our
algorithm computes the gcd of $M$ and $N$, let's check that it
computes the gcd of all pairs of numbers in $1\dd N$.  We do this by
  \marginpar{This is one situation where there
             is no good way to test the algorithm without
             modifying it.}%
declaring the initial values of $x$ and $y$ to be arbitrary elements
of $1\dd N$.  We also add two variables $x0$ and $y0$ that initially
equal $x$ and $y$, respectively, and whose values are left unchanged.
We then check that, when the algorithm terminates, $x$ and $y$ equal
$GCD(x0,y0)$.

Change the \textbf{variables} declaration of the algorithm to:
\begin{display}
\begin{nopcal}
variables x \in 1..N, y \in 1..N, x0 = x, y0 = y ;
\end{nopcal}
\begin{tlatex}
 \@x{ {\p@variables} x \.{\in} 1 \.{\dotdot} N ,\, y \.{\in} 1 \.{\dotdot} N
 ,\, x0 \.{=} x ,\, y0 \.{=} y {\p@semicolon}}%
\end{tlatex}
\end{display}
Rerun the translator and examine the formulas $Init$ and $Next$ that
it produces.  Formula $Init$ should be what you expect it to be, and
formula $Next$ is the same as before except for a conjunct asserting
that $x0$ and $y0$ are unchanged.

Create a new model by 
  \popref{clone-model}{cloning the model} you already created.
In the model's \textsf{Invariants} section, uncheck the invariant
$PartialCorrectness$ and add the invariant:
\begin{display}
$(pc = "Done") => (x = y) /\ (x = GCD(x0, y0))$
\end{display}
When you're not sure how long checking a model will take, start with a
very small model.  Set the value of $N$ to be~5, so there are 25
possible behaviors of the algorithm (because there are 25 different
initial states).  Even with such a small model, running TLC with a
single \popref{worker-threads}{worker thread} takes 30 seconds on my
computer.  Why is it so slow?

TLC is spending almost all its time computing $GCD(x0, y0)$ when
evaluating the invariant.  Doing that requires it to compute
$Divisors(x0)$ and $Divisors(y0)$.  TLC computes $Divisors(n)$ from
the definition of $Divisors$ by enumerating all the elements $p$ in
$Int$ and checking if $Divides(p,n)$ is true.  In the common case when
$p$ does not divide $n$, this computing $Divides(p,n)$ requires TLC to
check that $n$ does not equal $p*q$ for every element $q$ of $Int$.
Since our model redefines $Int$ to be a set with about 2000 elements,
computing $GCD(x0, y0)$ requires TLC to compute an expression of the
form $n=p*q$ about 8 million times.  It computes $GCD(x0, y0)$ 25
times for this model---once for the final state of each of the
behaviors.  Experimentation reveals that there is a constant 7 second
start-up overhead, and simple arithmetic then shows that it takes TLC
a little over .1 microsecond to compute $n=p*q$.  This is about 100
times longer than it takes a Java program to evaluate the same
expression on my computer.

All the positive divisors of a positive integer $n$ are elements of
$1\dd n$.  TLC will therefore correctly compute $GCD(x0,y0)$ for $x0$
and $y0$ in $1\dd N$ if we redefine $Int$ to equal $1\dd N$.  Change
the model to override the definition of $Int$ with this value.  It now
takes TLC only 7 seconds to run the model on my computer for $N=5$.
For $N=100$, it takes 33 seconds.  

This example illustrates that for checking a spec, it helps to have a
basic understanding of how TLC works.  It also shows that the
simplicity and elegance of mathematics compared to programming
languages comes at a high price in efficiency of execution.
Fortunately, checking all behaviors of a small model is generally more
effective at finding errors in an algorithm than checking randomly
chosen behaviors of a programming-language implementation.

\bigskip

Instead of checking the algorithm by adding an invariant to the model,
we can add an \textbf{assert} statement to the algorithm.  Place the
following statement right after the \textbf{while} statement:
  \ctindex{1}{assert (PlusCal statement)@\icmd{textbf}{assert} (PlusCal statement)}{assert}%
\begin{twocols}
\textbf{assert} $(x = y) \; /\ \; (x = GCD(x0, y0))$
\midcol
\verb|assert (x = y) /\ (x = GCD(x0, y0))|
\end{twocols}
Execution of the statement \textbf{assert}~$P$ does nothing if $P$ is
true, and it reports an error if $P$ is false.  
Save the module and
run the translator again.  If you followed the directions above
exactly, this will yield a translator error reporting a missing
  \ctindex{1}{semicolon (\mmath{;}) (PlusCal separator)}{semicolon-pluscal}%
   \ctindex{1}{+5pk@\mmath{;} (PlusCal separator)}{+5pj}%
semicolon (;) before the \textbf{assert}.
Separate PlusCal statements must be separated by semicolons.  (A
semicolon can be placed at the end of a sequence of statements, but it
is not required.)  Insert the missing semicolon, which most people
place just to the right of the \textbf{\}} that ends the \textbf{while}
statement.  Save the module and run the translator again.  This should
result in the parser error:%
  \marginpar{\popref{translator-bug}{If you got a different error,
                                     click here.}}
\begin{display} \tt
Unknown operator: `Assert'.
\end{display}
The translation of the \textbf{assert} statement uses a special
operator 
  \ctindex{1}{Assert (defined in \mmath{TLC} module)}{Assert-TLC-module}%
$Assert$ defined in the 
  \ctindex{1}{TLC module@\mmath{TLC} module}{tlc-module}%
standard $TLC$ module.  It defines $Assert(P,m)$ to equal $\TRUE$ if
$P$ equals $\TRUE$.  If TLC evaluates $P$ to be different from
$\TRUE$, it reports an error that includes $m$.  (In that case the
value of $P$ shouldn't matter.)  Add $TLC$ to the \textsc{extends}
statement and save the module.  The parser error disappear, and you
can now run TLC\@.

Try changing the \textbf{assert} statement to cause an error---for
example change $x=y$ to $x#y$---and run TLC\@.  Clicking on the
appropriate links in the error message will take you to the
\textbf{assert} statement and to its translation.




\subsection{Checking Liveness} \xlabel{main:liveness}

Open the model for the $Euclid$ algorithm that you have been checking
with TLC. Open the \textsf{Properties} part of the \textsf{What to
check?} section of the \textsf{Model Overview} page.  It should list
the property $Termination$, but with it unchecked.  Remember that
$Termination$ is the temporal formula that is true of a behavior iff
the behavior terminates (reaches a state with $pc$ equal to $"Done"$).
(If it's not in the list, add it.)  Check that property to tell TLC to
check it.  Have the model set $N$ to 10 and run TLC on it.

TLC reports that
\begin{display}
\tt Temporal properties were violated.
\end{display}
and it produces an error trace consisting of a single state, which is a
possible initial state (one satisfying the $Init$ predicate), followed
by the mysterious indication $\langle$\textsf{Stuttering}$\rangle$.
This trace describes a behavior consisting of a single state,
representing an execution that stops in an initial state.  (It will
become clear later why the trace says \emph{Stuttering}.)

A behavior of the algorithm is a sequence $s_{1}-> s_{2}-> \cdots$
that satisfies two conditions:
\begin{enumerate}
\item $Init$ is true if the variables have their values in state $s_{1}$.
(Remember that a state is an assignment of values to variables.)

\item For any pair $s_{i}->s_{i+1}$ of successive states, $Next$ is
true if the unprimed variables have their values in $s_{i}$ and
primed variables have their values in $s_{i+1}$.  
\end{enumerate}
It seems natural also to require that the behavior doesn't end before
it has to---in other words, to add the condition:
\begin{enumerate}
\item[3.] The behavior does not end in a state $s_{n}$ if there exists
a state $s_{n+1}$ such that the sequence $s_{1}->\ldots->s_{n+1}$ also
satisfies condition~2.
\end{enumerate}
However, the PlusCal algorithms we have written thus far do not have
this requirement.  They allow all behaviors that satisfy conditions 1
and 2, including behaviors that stop in the initial state.  More
precisely, the temporal formulas $Spec$ that are those algorithms'
translations allow all such behaviors.

To add requirement 3 for the behaviors of an algorithm, 
instead of beginning the algorithm with \texttt{-{}-algorithm},
we begin it with:
\begin{display}
\tt -\mbox{}-fair algorithm
   \ctindex{1}{fair algorithm@\icmd{fair} \icmd{algorithm}}{fair-alg}%
\end{display}
Make this change, run the translator, and run TLC again on the model.
This time, TLC finds no error, verifying that for the model, all
behaviors terminate.

Examining the translation, we find that the new definition of 
the behavior specification $Spec$
is the conjunction of its original definition and the formula
$\WF_{vars}(Next)$ (written in \textsc{ascii} as \verb|WF_vars(Next)|).
It is this formula that expresses condition~3.  The requirement is
called 
    \ctindex{1}{weak fairness!of next-state action}{weak-fairness-next}%
    \ctindex{1}{fairness!weak, of next-state action}{fairness-weak-next}%
\emph{weak fairness} of the action $Next$.  We will study
fairness formulas later.  For now, you need only know that this
particular formula, with $Next$ the specification's next-state action,
asserts condition~3.%
     \marginpar[3]{\popref{why-require-fair}{Why don't we require 
     condition~3 to hold for all algorithms?}}


\subsection{The Translation Revisited}

Let's return to the definition of $Next$ in the translation, which is
\begin{display}
$Next == Lbl\_1 \; \/ \; (pc = "Done" \, /\ \, \UNCHANGED vars)$
\end{display}
where $vars$ is defined to equal 
 \tlabox{<< x,\, y,\, x0,\, y0,\, pc >>}.  
Action $Lbl\_1$ describes the steps allowed by the body of the
algorithm.  The second disjunct allows steps that start in a state in
which $pc$ equals $"Done"$ and leaves the algorithm's five variables
unchanged.  A step that leaves all of a specification's variables
unchanged is called a
     \tindex{1}{stuttering step}%
     \ctindex{1}{step!stuttering}{step-stuttering}%
\emph{stuttering} step.  

The comment added by the translator tells us that this disjunct is
added to prevent deadlock on termination.  To verify that it's needed,
comment out the disjunct, save the module, and run TLC on the same
model.  (An easy way to comment out those two lines is to select
them and type \textsf{control+/}.)  Indeed, TLC reports that deadlock
was reached and shows an error trace ending in a terminated state.

TLC considers a reachable state from which there is no next state
satisfying the next-state action to be a deadlock error.  The only
difference between
  \ctindex{1}{termination!versus deadlock}{term-vs-deadlock}%
  \ctindex{1}{deadlock!versus termination}{deadlock-vs-term}%
deadlock and termination is that termination is deadlock that we want
to happen---or equivalently, that deadlock is termination we don't
want to happen.  TLC doesn't know whether or not we wanted this
deadlock to happen.  We can tell TLC to ignore deadlock by unchecking
the \emph{Deadlock} box in the \emph{What to check} section of the
model overview page.  However, it's possible to write PlusCal
algorithms that can deadlock at a state with $pc#"Done"$.  This
usually indicates an error---that is, deadlock that we didn't want to
happen---so we want TLC to report it.  Therefore, the translation adds
this disjunct to the next-state action so TLC doesn't treat
termination as deadlock.

\subsection{The Grain of Atomicity}


The \tlaplus\ translation defines the next-state action $Next$ for
which an execution of one iteration of the \textbf{while} loop is a
single step.  Why?  Why didn't the translator produce a definition of
$Next$ in which evaluating the \textbf{while} test and executing the
body of the \textbf{while} statement are represented as two separate
steps?  Perhaps it should have made execution of the \textbf{if}
statement two steps, one evaluating the condition and the second
executing either the \textbf{if} or the \textbf{else} clause.

In PlusCal, what constitutes a step is specified by the use of 
    \tindex{1}{labels, PlusCal}%
    \ctindex{1}{PlusCal!labels in}{pluscal-labels-in}%
labels
in the code.  A step is execution from one label to the next.  For
uniprocessor algorithms like the ones we have written so far, we can
omit the labels and let the translator decide where they belong.  For
algorithm $Euclid$, the translator decided that there should be a
single label $Lbl\_1$ on the \textbf{while} statement.  To see that
this is the case, let's explicitly add the label $abc$ to the
\textbf{while} loop, so it becomes:
\begin{display}
\begin{nopcal}
abc: while (x # y) { ...
\end{nopcal}
\begin{tlatex}
 \@x{ abc\@s{.5}\textrm{:}\@s{3} {\p@while} {\p@lparen} x \.{\neq} y
 {\p@rparen} {\p@lbrace} \.{\dots}}%
\end{tlatex}
\end{display}
Run the translator.  The translation is exactly the same as before
except that formula $Lbl\_1$ has become formula $abc$, whose
definition is the same as the original definition of $Lbl\_1$ except
that the string $"Lbl\_1"$ has been replaced by $"abc"$.

There are rules for where labels must go and where they may not go.
Most of the rules serve to make the translation simple, which is
important because we want to be able to reason about it.
You'll learn the rules as we go along, and the translator's error
messages will tell you if you've omitted a necessary label or put
one where it shouldn't go.  The first 
  \ctindex{1}{rules!labeling}{rules-labeling}%
  \ctindex{1}{labels, PlusCal!rules for}{labels-rules}%
two rules are:
\begin{itemize}
\item The first statement in the body of the algorithm must have a label.

\item A \textbf{while} statement must have a label.
\end{itemize}
Both imply that the translator had to add a (virtual) label where it
did.  If we let it decide where the labels should be, it uses as few
as possible.  This produces a specification in which an execution has
the fewest possible steps, which makes model checking most efficient.
It also produces the simplest translation.  For uniprocess algorithms,
we usually care only about the answer they produce and not what
constitutes a step.

Let's see what happens when we add another label.  Put the label $d$ on the
\textbf{if} statement, so the body of the algorithm becomes:
\begin{display}
\begin{nopcal}
abc: while (x # y) { d: if (x < y) { y := y - x }
                        else       { x := x - y }
                   } ;
     assert (x = y) /\ (x = GCD(x0, y0))
\end{nopcal}
\begin{tlatex}
 \@x{ abc\@s{.5}\textrm{:}\@s{3} {\p@while} {\p@lparen} x \.{\neq} y
 {\p@rparen} {\p@lbrace} d\@s{.5}\textrm{:}\@s{3} {\p@if} {\p@lparen} x \.{<}
 y {\p@rparen} {\p@lbrace} y \.{:=} y\@s{0.10} \.{-} x {\p@rbrace}}%
 \@x{\@s{118.19} {\p@else}\@s{24.59}\@s{9.09} 
{\p@lbrace} x \.{:=} x \.{-} y
 {\p@rbrace}}%
\@x{\@s{96.19} {\p@rbrace} {\p@semicolon}}%
 \@x{\@s{20.64} {\p@assert} ( x \.{=} y ) \.{\land} ( x \.{=} GCD ( x0 ,\, y0
 ) )}%
\end{tlatex}
\end{display}
There are two kinds of steps in an execution of this algorithm:
\begin{description}
\item[An \emph{abc} step:] The step starts with control at $abc$ and, based
on the value of the test $x#y$, either moves control to $d$ or else
executes the \textbf{assert} statement and moves control to $Done$
(the implicit control point at the end of the algorithm).

\item[A \emph{d} step:] A step that starts with control at $d$, executes
the \textbf{if} step, and then moves control to $abc$.
\end{description}
Run the translator.  The translation defines two subactions, $abc$ and
$d$, that describe these two kinds of steps.  It defines $Next$ to be
the disjunction of these two subactions and of the subaction allowing
stuttering steps when the algorithm has terminated.

Try adding other labels in addition to or instead of $d$.  Make
sure you understand the translations.  In this algorithm, you can add
a label at the beginning of any complete statement.  The only
requirement is that the \textbf{while} statement be labeled.  As you
have already figured out, the translation defines a subaction for each
label.  Run TLC on the different versions (for a small value of $N$)
and compare their numbers of reachable states.

\subsection{Why Euclid's Algorithm Is Correct} \xlabel{why-euclid-correct}

Checking an algorithm with TLC can give us some confidence that an
algorithm is correct.  How much confidence depends on the algorithm.
It cannot show us \emph{why} the algorithm is correct.  For that, we
need a proof.

In this track, we write only informal correctness proofs.  Writing any
kind of proof helps you understand an algorithm and therefore helps
you avoid errors.  However, it's often easy to write an incorrect
informal proof that claims to prove a property that an algorithm
doesn't satisfy---especially for a safety property.  The informal
safety proofs we will write can be made as rigorous as necessary to
give us sufficient confidence in their correctness.  (What constitutes
sufficient confidence depends on what the algorithm is going to be
used for.)  If necessary, they can be turned into formal \tlaplus\
proofs and checked with the
  \tindex{1}{TLAPS}%
TLAPS proof system.  Few readers will ever need to write a formal
proof.  However, learning to write formal proofs will improve your
ability to write rigorous informal ones.  I therefore urge you
to learn how to write and check formal proofs by reading at least the
beginning of the 
 \rref{proof}{top}{\tlaplus\ Proof Track}.

Since we are reasoning about the algorithm, not testing it, let's use
the simpler, original version of the algorithm.  Recall that this
version computed the gcd of $M$ and $N$ with $x$ and $y$ the only
(declared) variables, and it had no labels and no \textbf{assert}
statement.  Change the algorithm in the $Euclid$ module back to that
version and run the translator.

\subsubsection{Proving Invariance}  \xlabel{proving-invariance}

The safety property we want to prove about algorithm $Euclid$ is the
invariance of the state predicate $PartialCorrectness$, which is defined
to equal
 \[ (pc = "Done") \;=>\; (x = y) \, /\ \, (x = GCD(M, N))
 \]
A 
  \ctindex{1}{state predicate}{state-predicate}%
  \ctindex{1}{predicate!state}{predicate-state}%
state predicate is a formula that is true or false of a state.  In
other words, it is a Boolean-valued expression that may contain
variables but no primes (or temporal operators).  Invariance of a
state predicate means that it is true in every state of every behavior
of the algorithm.  To prove that a state predicate $Inv$ is true in
every state of a particular behavior
  $s_{1}->s_{2}-> \ldots$\,, 
we prove:
\begin{enumerate}
\item $Inv$ is true in state $s_{1}$.

\item For every step $s_{n}->s_{n+1}$ in the behavior, if $Inv$ is
true in state $s_{n}$ then it is true in state $s_{n+1}$.
\end{enumerate}
It follows by induction from 1 and 2 that $Inv$ is true for every
state $s_{n}$ of the behavior.  This reasoning shows that we
can prove that $Inv$ is true in every state of every behavior by
proving:
\begin{enumerate}
\item $Inv$ is true for any initial state, and

\item If $Inv$ is true in a state $s$ and $s->t$ is a possible step
of the algorithm, then $Inv$ is true in state $t$.
\end{enumerate}
An initial state is one that satisfies the initial predicate $Init$.
Therefore the first condition is equivalent to the truth of:
\begin{enumerate}
\item[I1.] $Init => Inv$
\end{enumerate}
A step $s -> t$ is a possible step of the algorithm only if the
next-state action $Next$ is true when each unprimed variable has its
value in state $s$ and each primed variable has its value in state
$t$.  For any state predicate $P$, we define $P'$ to be the formula
obtained from $P$ by priming all the variables in it.  For example,
$PartialCorrectness'$ equals
 \[ (pc' = "Done") \;=>\; (x' = y') \,/\ \,(x' = GCD(M, N))
 \]
Condition~2 is then satisfied if the following formula is true:
\begin{enumerate}
\item[I2.] 
$Inv /\ Next => Inv'$%
 \target{main:inv-i2}
\end{enumerate}
Make sure you understand why the truth of I2 implies the truth of
condition~2 above.

An invariant 
  \target{main:inductive-invariant}%
$Inv$ satisfying I1 and I2 is called an
   \ctindex{1}{invariant!inductive}{inv-inductive}%
   \tindex{1}{inductive invariant}%
\emph{inductive invariant} of the algorithm.  (A predicate satisfying
I2 is sometimes called an inductive invariant of the next-state action
$Next$.)  Although $PartialCorrectness$ is an invariant of algorithm
$Euclid$, it is not an inductive invariant.  It satisfies I1 but not I2.
For example, consider the following values for the unprimed and primed
variables:
 \[ x = 42 \s{1.5} y = 42 \s{1.5} pc = "Lbl\_1" 
       \s{2} x' = 42 \s{1.5} y' = 42 \s{1.5} pc' = "Done"
 \]
You can check that $Next$ is true for these values of the primed and
unprimed variables by substituting them in the definition of $Lbl\_1$
and checking that the resulting formula equals $\TRUE$.  This is perhaps
easier to see by observing that the step
%
%
 \[ \pestate{42}{42}{"Lbl\_1"}\  -> \ \pestate{42}{42}{"Done"} 
 \]
\begin{sloppypar}
\noindent which starts with control at the beginning of the \textbf{while}
statement and ends with control at the end of the algorithm, is
allowed by the code in the algorithm's body.  With those values of the
primed and unprimed variables, $PartialCorrectness$ equals $\TRUE$
(because $pc=Done$ equals \FALSE), and $PartialCorrectness'$ equals
the formula
$42 = GCD(M,N)$ (because $pc'="Done"$ and $x'=y'$ both equal
$\TRUE$).  Hence with these substitutions, I2 becomes $\TRUE /\ \TRUE
=> (42 = GCD(M,N))$, which equals $42 = GCD(M,N)$.  Thus, I2 is false
for $Inv$ equal to $PartialCorrectness$ unless the gcd of $M$ and $N$
happens to equal 42.  In that case, we can replace 42 by another
number to get an example in which I2 is false.  Therefore,
$PartialCorrectness$ is not an inductive invariant.
\end{sloppypar}

This was a long calculation to demonstration something that should
have been obvious.  Formula $PartialCorrectness$ is true in any state
with $pc$ not equal to $"Done"$.  Its truth tells us nothing about the
relation between the values of $x$, $y$, and $GCD(M,N)$ during the
algorithm's execution, so its truth during the execution can't imply
that it will be true upon termination.  However, doing this long
calculation should help you understand that, by describing the
algorithm with two formulas, $Init$ and $Next$, \tlaplus\ reduces
reasoning about an algorithm to simple mathematics.

We are still left with the problem of proving the invariance of
$PartialCorrectness$.  We do that by finding an inductive invariant
$Inv$ that, in addition to I1 and I2, satisfies:
\begin{enumerate}
\item[I3.] $Inv => PartialCorrectness$
\end{enumerate}
Conditions I1 and I2 imply that $Inv$ is true in all reachable states,
which by I3 implies that $PartialCorrectness$ is true in all reachable
states, so it is an invariant.  

The fundamental reason why Euclid's algorithm computes the gcd is that
it maintains the invariance of the state predicate:
 \[ GCD(x, y) = GCD(M, N)
 \]
This is an inductive invariant of the algorithm.  However, it doesn't 
satisfy I3 because it doesn't imply that $x$ equals $y$ on
termination.  An inductive invariant $Inv$ that satisfies $I3$
is:
 \[ Inv == \begin{conj}
           GCD(x, y) = GCD(M, N) \V{.2}
           (pc = "Done") => (x = y)
           \end{conj}
 \]
The proof that $Inv$ satisfies I1--I3 requires three facts about the gcd.
These facts, which we call $GCD1$--$GCD3$, are expressed by the
following theorems:
\begin{display}
\begin{notla}
THEOREM GCD1 == \A m \in Nat \ {0} : GCD(m, m) = m

THEOREM GCD2 == \A m, n \in Nat \ {0} : GCD(m, n) = GCD(n, m)

THEOREM GCD3 == \A m, n \in Nat \ {0} : (n > m) => (GCD(m, n) = GCD(m, n-m))
\end{notla}
\begin{tlatex}
 \@x{ {\THEOREM} GCD1 \.{\defeq} \A\, m \.{\in} Nat \.{\,\backslash\,} \{ 0 \}
 \.{:} GCD ( m ,\, m ) \.{=} m}%
\@pvspace{6.0pt}%
 \@x{ {\THEOREM} GCD2 \.{\defeq} \A\, m ,\, n \.{\in} Nat \.{\,\backslash\,}
 \{ 0 \} \.{:} GCD ( m ,\, n ) \.{=} GCD ( n ,\, m )}%
\@pvspace{6.0pt}%
 \@x{ {\THEOREM} GCD3 \.{\defeq} \A\, m ,\, n \.{\in} Nat \.{\,\backslash\,}
 \{ 0 \} \.{:} ( n \.{>} m ) \.{\implies} ( GCD ( m ,\, n ) \.{=} GCD ( m ,\,
 n \.{-} m ) )\s{-10}}%
\end{tlatex}
\end{display}
Let's just assume them for now; we'll return to them later.

\popref{euclid-inv-proof}{Click here} for a proof of the invariance of
$PartialCorrectness$.  The first thing you will notice is that this
proof doesn't look like an ordinary mathematician's proof.  Instead,
  \tindex{1}{structured proof}%
  \ctindex{1}{proof!structured}{proofs-structured}%
it is hierarchically structured.  Proofs of algorithms can be quite
complicated, and the way to handle complexity is by hierarchical
structure.  Here are some other things to observe about the proof style.
\begin{itemize}
\item A proof is either a leaf proof, consisting of a short paragraph;
or it is a sequence of steps, each with a proof, ending with a
\textsc{qed} step.  

\item A \textsc{qed} step asserts the goal of the current level of
proof.  Its proof shows that this goal is proved by the preceding
steps.

\item A \textsc{case} statement asserts that the current proof's goal
is true if the \textsc{case} assumption is.
\end{itemize}
Learning to write proofs that are correct and easy to read is an art.
Here are a couple of tips.
\begin{itemize}
\item If a leaf proof is too long to be easily understood, it should
be decomposed into a non-leaf proof, adding another level to the
hierarchy.  A leaf proof that is not easy to understand could easily
be incorrect.

\item Any previous proof steps required by a leaf proof should
be explicitly mentioned, as should other significant facts being
used (such as $GCD1$--$GCD3$).
\end{itemize}
This proof may seem rigorous.  Actually it is incorrect---for a reason
that should eventually become obvious to you.  Throughout the proof,
there is an implicit assumption that $x$, $y$, $M$, and $N$ are
positive integers.  The \textsc{assume} statement in the module
asserts that $M$ and $N$ are positive integers, justifying that
assumption.  Step~1 is therefore correct, though its proof should
mention that it uses the assumption.  However, there is nothing in the
hypotheses of any other step to imply that $x$ and $y$ are positive
integers---an assumption that is needed to apply $GCD1$--$GCD3$.  We
can't even prove that $x$ and $y$ are numbers.

In all the proofs except for that of step~1, we get to assume that
$Inv$ is true.  Thus, we can justify the assumption that $x$ and $y$
are positive integers by having $Inv$ assert it.  Let's do that
by defining:
  \[ TypeOK == \begin{conj}
               x \in Nat :\: \{0\} \V{.2}
               y \in Nat :\: \{0\} \V{.2}
               \end{conj}
 \]
and changing the definition of $Inv$ to 
 \[ Inv == \begin{conj}
           TypeOK\V{.2}
           GCD(x, y) = GCD(M, N) \V{.2}
           (pc = "Done") => (x = y)
           \end{conj}
 \]
With this change, our proof becomes correct in the sense that the
assertion made by every step is true.  However, a more rigorous proof
would mention that the proof uses $TypeOK$.  Also, in steps 2.1--2.3,
the proofs of $Inv'$ need to prove $TypeOK'$.

In general, an inductive invariant must contain a type-correctness
condition.  Since that's not a very interesting part of the invariant,
we encapsulate this condition in a separate formula that I like to
call $TypeOK$.  The formula usually has a conjunct for each variable,
asserting that the variable is an element of some set.  For uniprocess
PlusCal algorithms such as this one, there may be no need of a
type-correctness condition for the variable~$pc$.  We may not bother
mentioning the use of $TypeOK$ in an informal proof.  However, we should
include it in the inductive invariant, because proving statements that
are not true is a bad habit to get into.

\begin{aquestion}{euclid-inv-fg}
Modify the algorithm by labeling the \textbf{while} loop $abc$ and
labeling the \textbf{if} statement $d$.  Show that the formula
$Inv$ defined above is not an inductive invariant of the resulting
algorithm.  Find an inductive invariant of this algorithm that implies
$PartialCorrectness$.
\end{aquestion}
%
\popref{floyd-hoare}{Click here if you already learned how to prove partial
correctness of programs.}


\subsubsection{Verifying \emph{GCD}1--\emph{GCD}3} \xlabel{verifying-gcd}

A complete proof of Euclid's algorithm should include a proof of 
$GCD1$--$GCD3$.  However, before we do any proving, we should use TLC
to check the correctness of these theorems.  It's a lot easier to
prove something if it's true.  And even an obviously true theorem
could be incorrect because of a typo.

Open the $GCD$ spec in the Toolbox and create a new model.  We can
check all three theorems at once by having the model tell TLC to
  \popref{evaluate-constant-expression}{evaluate the constant expression}
\begin{display}
\begin{twocols}
$<<GCD1,\ GCD2,\ GCD3>>$
\midcol
\verb|<<GCD1, GCD2, GCD3>>|
\end{twocols}
\end{display}
We saw in \lref{\xlink{main:euclid:gcd}}{Section~\xref{main:euclid:gcd}}
that we must override the definition of
$Int$ with a finite set of integers to allow TLC to evaluate the
$GCD$ operator.  The three theorems are all of the form 
 \[\A \ldots\in Nat :\: \{0\}: \ldots \]
TLC can evaluate such formulas only if we override the definition 
of $Nat$.  Have the model \popref{definition-override}{override the
 definitions} of $Nat$ and $Int$ with small sets of integers---for example,
$0\dd5$---and run TLC on it.  If you've made no error, it should
report the value $<<TRUE, \ TRUE, \ TRUE>>$.  You can then check
$GCD1$--$GCD3$ on a larger model.  It should take TLC one or two
minutes to do this for a model that defines $Nat$ and $Int$ to equal
$0\dd100$.
%% 78 seconds for Nat = 0..100 and Int = 0..100

Having checked $GCD1$--$GCD3$ with TLC, we can now think about proving
them.  Theorems $GCD1$ and $GCD2$ follow easily from the definition of
$GCD$ and we won't bother proving them.  The proof of $GCD3$ uses this
simple fact
\begin{display}
\textbf{Lemma Div} For any integers $m$, $n$, and $d$, if $d$ divides
both $m$ and $n$ then it also divides both $m+n$ and $n-m$.
\end{display}
You should have no trouble proving it.

Here is \popref{gcd-proof-1}{a proof of $GCD3$}.  The structure of the
proof becomes clearer, and the proof easier to read, if we introduce
notation to replace some of the prose, obtaining
\popref{gcd-proof-2}{this proof}.  Compare the two proofs.  To help
you understand the second proof, \popref{gcd-proof-3}{here it is with
comments.} Although the notation may seem strange, you should be able
to see that it makes the second proof easier to read.  Ease of reading
is very important for complex proofs.

\bigskip

%\noindent 
The formal \tlaplus\ versions of the invariance proof of Euclid's
algorithm and the proof of $GCD1$---$GCD3$ are in
Section~\xref{proof:euclid} of the \rref{proof}{top}{Proof Track}.

\subsubsection{Proving Termination}

To prove that algorithm $Euclid$ always terminates (assuming
fairness), we observe that each step of the algorithm that doesn't
reach a terminating step decreases either $x$ or $y$ and leaves the
other unchanged.  Thus, such a step decreases $x+y$.  Since $x$ and
$y$ are always positive integers, $x+y$ can be decreased only a finite
number of times.  Hence, the algorithm can take only a finite number
of steps without terminating.

In general, to prove that an algorithm terminates, we find
an integer-valued state function $W$ for which:
\begin{itemize}
\item $W \geq 0$ in any reachable, non-terminating state.

\item If $s$ is any reachable state and $s->t$ is any step satisfying
the next-state action, then either the value of $W$ in state $s$ is
greater than its value in state $t$, or the algorithm is terminated in
state $t$.
\end{itemize}
To reason about reachable states, we use an invariant---which by
definition is a predicate that is true in every reachable state of an
algorithm.  Let $Next$ be the algorithm's next-state action.  The same
kind of reasoning that led to \lref{main:inv-i2}{condition I2 above}
shows that we can prove termination by finding a state function $W$
and an invariant $I$ of the algorithm satisfying:
\begin{enumerate}
\item[L1.] $I => (W \in Nat) \/ (pc = "Done")$%
 \target{main:cond-l1}

\item[L2.] $I /\ Next => (W > W') \/ (pc' = "Done")$
\end{enumerate}
The state function $W$ is called a
 \tindex{1}{variant function}%
 \ctindex{1}{function!variant}{function-variant}%
\emph{variant function}.  For algorithm $Euclid$, we let $W$ be $x+y$
and we let $I$ be the (inductive) invariant $Inv$.

The use of the ordering $>$ on natural numbers in this method can be
generalized to any 
  \popref{well-founded}{well-founded ordering} on a set.  However, the
generalization is seldom needed to prove termination of
uniprocess algorithms.

\ctindex{1}{Euclid's Algorithm!for sets}{euclid-algorithm-sets}%
\vspace{-\baselineskip}
\subsection{Euclid's Algorithm for Sets}

We now consider a generalization of Euclid's algorithm that I find
elegant.  It computes the gcd of a set of numbers, rather than of just
two numbers.  We start by defining $SetGCD(T)$ to be the gcd of a set
$T$ of positive integers.  It equals the maximum of the set of numbers
that divide all the numbers in $T$:
\begin{display}
\begin{notla}
SetGCD(T) == SetMax({d \in Int : \A t \in T : Divides(d, t)})
\end{notla}\midcol
\begin{tlatex}
 \@x{ SetGCD ( T ) \.{\defeq} SetMax ( \{ d \.{\in} Int \.{:} \A\, t \.{\in} T
 \.{:} Divides ( d ,\, t ) \} )}\ascii{setgcd-ascii}%
\end{tlatex}
\end{display}
Add the definition to module $GCD$ and check it for one or two small
sets of positive integers.  (Use the same model you did before, which
defined $Int$ to be a finite set of integers.)

The algorithm, which computes the gcd of a non-empty set $Input$ of positive
integers, uses a single variable $S$.  Its informal description is:
\begin{itemize}
\item Start with $S$ equal to $Input$.

\item While $S$ has more than one element, choose elements $x$ and $y$
in $S$ with $y > x$, remove $y$ from $S$ and insert $y-x$ in $S$.

\item If $S$ contains a single element, that element is $SetGCD(Input)$.
\end{itemize}
(If you don't understand how removing  one element from $S$ and inserting
another could reduce the number of elements in $S$, you need to
\rref{math}{sets-intro}{read about sets}.)

The body of the PlusCal algorithm should be a \textbf{while} loop
whose test asserts that $S$ has more than one element.  The standard
$FiniteSets$ module defines
  \ctindex{2}{Cardinality@\mmath{Cardinality}}{cardinality}%
  $Cardinality(S)$ to equal the
number of elements in $S$, if $S$ is a finite set.  The value of
$Cardinality(S)$ is unspecified if $S$ is not a finite set.  
We will assume that $Input$ is a finite set, so $S$ will always
be a finite set.  The \textbf{while} loop's test can therefore
be written as $Cardinality(S) > 1$.  Here is the complete
body of the algorithm:%
\begin{display}
\begin{nopcal}
while (Cardinality(S) > 1) 
  {  with (  x \in S,  y \in {s \in S : s > x}  ) 
       {  S := (S \ {y}) \cup {y-x}  }
  }
\end{nopcal}
\begin{tlatex}
\@x{ {\p@while} {\p@lparen} Cardinality ( S ) \.{>} 1 {\p@rparen}}%
  \target{seteuclid-body}%
 \@x{\@s{8.2} {\p@lbrace}\@s{4.1} {\p@with} {\p@lparen}\@s{4.1} x \.{\in} S
 ,\,\@s{4.1} y \.{\in} \{ s \.{\in} S \.{:} s \.{>} x \}\@s{4.1} {\p@rparen}}%
 \@x{\@s{30.08} {\p@lbrace}\@s{4.1} S \.{:=} ( S \.{\,\backslash\,} \{ y \} )
 \.{\cup} \{ y \.{-} x \}\@s{4.1} {\p@rbrace}}%
\@x{\@s{8.2} {\p@rbrace}}%
\end{tlatex}
\end{display}
To understand the meaning of the \textbf{with} statement, let's look at
the translation of the algorithm's body:

\begin{display}
\begin{notla}
Lbl_1 == /\ pc = "Lbl_1"
         /\ IF Cardinality(S) > 1
               THEN /\ \E x \in S:
                         \E y \in {s \in S : s > x}:
                           S' = ((S \ {y}) \cup {y-x})
                    /\ pc' = "Lbl_1"
               ELSE /\ pc' = "Done"
                    /\ S' = S
\end{notla}
\begin{tlatex} %\gray 
\@x{Lbl\_1 \.{\defeq} \.{\land} pc \.{=}\@w{Lbl\_1}}%
\@x{\@s{42.84} \.{\land} {\IF} Cardinality ( S ) \.{>} 1}%
\@x{\@s{66.11} \.{\THEN} \.{\land} \red \E\, x \.{\in} S \.{:}}%
 \@x{\@s{116.73} \red \E\, y \.{\in} \{ s \.{\in} S \.{:} s \.{>} x \}
 \.{:}}%  \@s{4.86} removed
 \@x{\@s{124.93} \darkgreen S' \.{=} ( ( S \.{\,\backslash\,} \{ y \} ) \cup \{ y - x \} )}%
\@x{\@s{97.42} \.{\land} pc' \.{=}\@w{Lbl\_1}}%
\@x{\@s{66.11} \.{\ELSE} \.{\land} pc' \.{=}\@w{Done}}%
\@x{\@s{97.42} \.{\land} S' \.{=} S}%
\end{tlatex}
\end{display}
The colored formula is the translation of the \textbf{with} statement;
the green formula is the translation of its body, the assignment to $S$.
Let's examine the formula piece by piece.
\begin{description}
\item[\darkgreen$S \,\backslash\, \{y\}$] \mbox{}\\
The set obtained by removing all the
elements in the set $\{y\}$ from $S$---in other words, obtained by
removing $y$ from $S$.

\item[\darkgreen$(S \,\backslash\, \{y\})\cup \{y - x\}$] \mbox{}\\
The set obtained from $S$ by removing $y$ and inserting $y-x$.
\end{description}
The green formula therefore asserts that $S'$ (the new value of $S$)
equals the set obtained by removing $y$ from (the old value of) $S$
and inserting $y-x$.
\begin{description}
\item[\red$\{s \in S : s > x\}$] \mbox{}\\ 
The set of elements in $S$
that are greater than $x$.
\end{description}
The entire formula therefore asserts that there exist $x$ and $y$ in
$S$, with $y>x$, such that the green formula is true.  
 \target{with-stmt}
The meaning
of the 
       \ctindex{2}{with (PlusCal keyword)@\icmd{with} 
                   (PlusCal keyword)}{with-pcal}%
\textbf{with} statement is therefore:
\begin{display}
Execute the body with $x$ an arbitrary element of $S$ and $y$ an
arbitrary element of $S$ greater than $x$.
\end{display}
In general, the statement\target{main:with}
\begin{display}
\textbf{with (} $v_{1} \in S_{1}$, \ldots\,, $v_{1} \in S_{1}$ \textbf{) \{}
 $\Sigma$ \textbf{\}}%
 \marginpar[.5]{\popref{with-in-vs-equal}{\textbf{with (\,$v\in S$\,)}
  versus \\ \textbf{with (}\,$v=S$\,)}}
\end{display}
is executed by letting each $v_{i}$ be arbitrary element in $S_{i}$
and executing $\Sigma$ with those values of the $v_{i}$.  TLC will check
the executions obtained by all possible choices of the $v_{i}$.

Create a new module named $SetEuclid$ that \textsc{extends} module
$GCD$ and $Integers$.  Enter the PlusCal specification, translate it,
and test that TLC executes the specification on a model with $Init$ a
small set of positive integers.  Make it a fair algorithm (beginning
with \,\texttt{-{}-fair algorithm}\,) and have TLC check that it terminates.

Partial correctness of Euclid's algorithm, which asserts that 
on termination $S$ contains the single element $SetGCD(Input)$, 
is expressed by the invariance of:
\begin{display}
\begin{notla}
PartialCorrectness == (pc = "Done") => (S = {SetGCD(Input)})
\end{notla}
\begin{tlatex}
 \@x{ PartialCorrectness \.{\defeq} ( pc \.{=}\@w{Done} ) \.{\implies} ( S
 \.{=} \{ SetGCD ( Input ) \} )}%
\end{tlatex}
\end{display}
Add this definition to module $SetEuclid$ and have TLC check that the
invariance of $PartialCorrectness$.  (Since evaluating this formula
requires TLC to compute $SetGCD(Input)$, your model will have to
override the definition of $Int$.)

The proof of partial correctness is analogous to that of 
algorithm $Euclid$ and is based on the invariance of
 \[ SetGCD(S) = SetGCD(Input) \]
A rigorous proof uses the inductive invariant
\begin{display}
\begin{notla}
SInv == /\ TypeOK
        /\ SetGCD(S) = SetGCD(Input)
        /\ PartialCorrectness
\end{notla}
\begin{tlatex}
\@x{ SInv \.{\defeq} \.{\land} TypeOK}%
\@x{\@s{39.66} \.{\land} SetGCD ( S ) \.{=} SetGCD ( Input )}%
\@x{\@s{39.66} \.{\land} PartialCorrectness}%
\end{tlatex}
\end{display}
where the type invariant is defined by
\begin{display}
\begin{notla}
TypeOK == /\ S \subseteq Nat \ {0}
          /\ S # {} 
          /\ IsFiniteSet(S)
\end{notla}
\begin{tlatex}
 \@x{ TypeOK \.{\defeq} \.{\land} S \.{\subseteq} Nat \.{\,\backslash\,} \{ 0
 \}}%
\@x{\@s{56.14} \.{\land} S \.{\neq} \{ \}}%
\@x{\@s{56.14} \.{\land} IsFiniteSet ( S )}%
\end{tlatex}
\end{display}
The assumption that $S$ is finite is required because we don't know
what the expression $Cardinality(S)$ in the \textbf{while} test means
if $S$ is not a finite set.  To prove that $TypeOK$ is true in the
initial state, we need the assumption:
\begin{display}
\begin{notla}
ASSUME == /\ Input \subseteq Nat \ {0}
          /\ Input # {}
          /\ IsFiniteSet(Input)
\end{notla}
\begin{tlatex}
 \@x{ {\ASSUME} \.{\defeq} \.{\land} Input \.{\subseteq} Nat
 \.{\,\backslash\,} \{ 0 \}}%
\@x{\@s{57.13} \.{\land} Input \.{\neq} \{ \}}%
\@x{\@s{57.13} \.{\land} IsFiniteSet ( Input )}%
\end{tlatex}
\end{display}
\begin{aquestion}{euclid-answer-no-cardinality}
Rewrite the algorithm without using the $Cardinality$ operator, so
partial correctness is true even for infinite sets $Input$.  (The
rewritten algorithm obviously does not terminate if $Input$ is an
infinite set.)
\end{aquestion}
%
The proof of termination is based on the observation that each
non-terminating step of the algorithm decreases the sum of all the
elements of $S$.  To state this rigorously, we must define the sum of
the elements in a finite set of numbers.  The only way I know to define
this mathematically is with a 
    \ctindex{1}{operator!recursive definition of}{op-recursive-def-of}%
    \ctindex{1}{recursive!operator definition}{recursive-op-def}%
recursive definition (also called by
mathematicians an 
  \tindex{1}{inductive definition}%
inductive definition): 
\begin{itemize}
\item The sum of the empty set is 0.

\item The sum of a non-empty set $T$ of integers is the sum of some
element $t$ in $T$ plus the sum of the elements in $T :\: \{t\}$.
\end{itemize}
Since it doesn't matter what element $t$ of $T$ is chosen in the
recursive step, we can use the \textsc{choose} operator to select it.
The obvious definition is then:
\begin{display}
\begin{notla}
SetSum(T) == IF T = {} THEN 0
                       ELSE LET t == CHOOSE x \in T : TRUE
                            IN  t + SetSum(T \ {t}) 
\end{notla}
\begin{tlatex}
\@x{ SetSum ( T ) \.{\defeq} {\IF} T \.{=} \{ \} \.{\THEN} 0}%
  \ascii{set-sum-ascii}%
 \@x{\@s{112.60} \.{\ELSE} \.{\LET} t \.{\defeq} {\CHOOSE} x \.{\in} T \.{:}
 {\TRUE}}%
\@x{\@s{143.91} \.{\IN} t \.{+} SetSum ( T \.{\,\backslash\,} \{ t \} )}%
\end{tlatex}
\end{display}
Add the definition to module $SetEuclid$ and save the module.  This
produces a parsing error complaining that the operator $SetSum$ is
undefined when used on the right-hand side of the symbol.  In \tlaplus,
a symbol must be defined or declared before it can be used.  To allow
such a recursive definition of $SetSum$, the definition must be preceded
by this \textsc{recursive} declaration:
\begin{display}
\begin{twocols}
\textsc{recursive} $SetSum(\_)$
\midcol
\verb|RECURSIVE SetSum(_)|
\end{twocols}
\end{display}
The declaration should be put right before the definition.
%
The parser should now accept the specification.  You can use TLC
to check that this is a correct definition of the sum of a finite
set of integers.

To prove termination, we prove
 \lref{main:cond-l1}{conditions L1 and L2}
with the invariant $I$ equal to $SInv$ and the variant function
$W$ equal to $SetSum(S)$.  The informal proof is straightforward.
It uses the following fact about $SetSum$:
\begin{display}
\begin{notla}
\A T \in SUBSET Nat : 
    IsFiniteSet(T) => /\ \A t \in T : SetSum(T \ {t}) = SetSum(T) - t 
                      /\ \A t \in Nat : SetSum(T \cup {t}) =< SetSum(T) + t 
\end{notla}
\begin{tlatex}
\@x{ \A\, T \.{\in} {\SUBSET} Nat \.{:}}%
 \@x{\@s{11.32} IsFiniteSet ( T ) \.{\implies} \.{\land} \A\, t \.{\in} T
 \.{:} SetSum ( T \.{\,\backslash\,} \{ t \} ) \.{=} SetSum ( T ) \.{-} t}%
 \@x{\@s{91.78} \.{\land} \A\, t \.{\in} Nat \.{:} SetSum ( T \.{\cup} \{ t \}
 ) \.{\leq} SetSum ( T ) \.{+} t}%
\end{tlatex}
\end{display}
where 
  \ctindex{2}{SUBSET@\icmd{textsc}{subset}}{big-subset}%
$\SUBSET Nat$ is the set of all subsets of the set $Nat$ of natural
numbers.  A rigorous proof reveals that some simple facts about finite
sets and $Cardinality$ are also required.

\begin{aquestion}{seteuclid-thm-answer} \targetlabel{question:gcd-rep}
Show that the correctness of algorithm $SetEuclid$ implies the
following important result from number theory.  The gcd of a set
$\{n_{1}, \ldots, n_{k}\}$ of positive integers equals
 $\, i_{1}*n_{1} + \ldots + i_{k}*n_{k} \,$
for some integers $i_{j}$.
\end{aquestion}

\bigskip
\noindent
\rref{proof}{top}{If you want to learn how to write formal \tlaplus\
proofs, you can now start reading the Proof track.}