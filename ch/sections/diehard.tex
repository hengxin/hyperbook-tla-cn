%try

% 1st
%%%%%%%%%%%%%%%
\begin{en}
\newpage
   \tindex{1}{Die Hard}%
\vspace{-\baselineskip}%

\section{The Die Hard Problem} \xlabel{sec:diehard}

In the movie \emph{Die Hard 3}, the heroes must solve the problem of
obtaining exactly 4 gallons of water using a 5 gallon jug, a 3 gallon
jug, and a water faucet.  We now apply \tlaplus\ and the TLC
model checker to solve this problem.
\end{en}

\begin{ch}
\newpage
   \tindex{1}{Die Hard}%
\vspace{-\baselineskip}%

\section{Die Hard 问题} \xlabel{sec:diehard}

在电影《虎胆龙威(三)》中,
两位英雄必须解决这样一个问题:
用一个 5 加仑的水壶,一个 3 加仑的水壶,\fixme{和一个水龙头},
如何得到 4 加仑的水?
现在,我们用 \tlaplus\ 和 \tlc{} 模型验证器解决该问题。
\end{ch}
%%%%%%%%%%%%%%%

% 2nd
%%%%%%%%%%%%%%%
\begin{en}
\subsection{Representing the Problem in TLAPlus}%

The first step in solving the problem is to model the physical system
of heroes, jugs, and faucet mathematically as a discrete system.
%
The only relevant state of the hero/jug/faucet system is the amount of
water in the two jugs.  So, we model the system with two variables,
$big$ and $small$, whose values represent the number of gallons of
water in the two jugs.  After choosing the variables, a good way to
figure out how to write a specification is to write down the first few
states of a possible behavior of the system.  Initially, the jugs are
empty, so $big$ and $small$ both equal~0.  Here's one possible
beginning of a behavior.  (Remember that a state is an assignment of
values to the variables, in this case $big$ and $small$.)
\begin{display}
\begin{tabbing}
The small jug is filled from the big one: \= \kill
%
\>
$\dhstate{0}{0}$\V{1.01}
%
\>The big jug is filled from the faucet.\'\s{2.7}$\downarrow$\V{.31}
%
\>
$\dhstate{5}{0}$\V{1.01} 
%
%
\>The small jug is filled from the big one.\'\s{2.7}$\downarrow$\V{.31}
%
\>
$\dhstate{2}{3}$\V{1.01}
%
\>The small jug is emptied (onto the ground).\'\s{2.7}$\downarrow$\V{.31}
%
\> $\dhstate{2}{0}$
\end{tabbing}
\end{display}
\end{en}

\begin{ch}
  \subsection{用 TLAPlus 描述 Die Hard 问题}%

  解决该问题的首要任务
  是以数学的方式将由英雄、水壶和水龙头构成的物理系统建模为一个离散系统。
  在该物理系统中,唯一相关的状态是两个水壶中的水量。
  因此,我们用两个变量建模该系统:
  $big$ 和 $small$ 分别表示大小两个水壶中的水量。
  选定变量之后,弄清如何编写规约的一个好方法是
  先写系统的某个行为的头几个状态。
  一开始,水壶是空的,因此 $big$ 和 $small$ 都等于 0。
  系统的某个行为的开头可能如下所示
  (一个状态是一种变量赋值。
  在本例中,变量是 $big$ 和 $small$。):
  \begin{display}
  \begin{tabbing}
  将大水壶里的水倒入小水壶,直到小水壶装满水为止。\= \kill
  %
  \>
  $\dhstate{0}{0}$\V{1.01}
  %
  \>大水壶从水龙头灌满水。\'\s{2.7}$\downarrow$\V{.31}
  %
  \>
  $\dhstate{5}{0}$\V{1.01} 
  %
  %
  \>将大水壶里的水倒入小水壶,直到小水壶装满水为止。\'\s{2.7}$\downarrow$\V{.31}
  %
  \>
  $\dhstate{2}{3}$\V{1.01}
  %
  \>将小水壶中的水倾空(倒在地上)。\'\s{2.7}$\downarrow$\V{.31}
  %
  \> $\dhstate{2}{0}$
  \end{tabbing}
  \end{display}
\end{ch}
%%%%%%%%%%%%%%%
%  in the same paragraph
% 3rd
%%%%%%%%%%%%%%%
\begin{en}
A little thought reveals that there are three kinds of steps in a
behavior:
\begin{itemize}
\item Filling a jug.
\item Emptying a jug.
\item Pouring from one jug to the other.  There are two cases:
\begin{itemize}
\item This empties the first jug.
\item This fills the second jug, possibly leaving water
      in the first jug.
\end{itemize}
\end{itemize}
We can now write the specification.  Let's \popref{open-new-spec}{open
a new specification} named $DieHard$ in the Toolbox.  Since the spec
will require arithmetic operations, it begins with:
\begin{display}
\begin{twocols}%[.45]
$\EXTENDS\ Integers$
\midcol
\verb|EXTENDS Integers|
\end{twocols}
\end{display}
We declare the variables and write the initial predicate, which we
give the conventional name $Init$.
\smallskip
\begin{display}
\begin{twocols}%[.45]
$\VARIABLES big,\ small$ \V{.4}
$Init == 
\begin{conj}
big = 0 \\ small = 0
\end{conj}
$
\midcol
\verb*|VARIABLES big, small|\V{.4}
\verb*|Init == /\ big = 0| \\
\verb*|        /\ small = 0|
\end{twocols}
\end{display}
\smallskip
\end{en}
% in the same paragraph 
\begin{ch}
  稍作思考,我们会发现一个行为包含三类\tlastep{}:
  \begin{itemize}
  \item (从水龙头)装满某个水壶。
  \item 倾空某个水壶。
  \item 从一个水壶往另一个水壶倒水。这分两种情况:
  \begin{itemize}
  \item 第一个水壶被倾空。
  \item 第二个水壶被装满。此时,第一个水壶中可能还有水。
  \end{itemize}
  \end{itemize}
  现在,我们可以编写规约了。
  在\tlatoolbox{}中\popref{open-new-spec}{打开一个新的规约},
  命名为 $DieHard$。
  由于该规约需要用到算术操作,所以它以
  \begin{display}
  \begin{twocols}%[.45]
  $\EXTENDS\ Integers$
  \midcol
  \verb|EXTENDS Integers|
  \end{twocols}
  \end{display}
  开头。
  我们声明变量,编写\tlainitpredicate{},
  \fixme{并按照惯例}将\tlainitpredicate{}命名为 $Init$。
  \smallskip
  \begin{display}
  \begin{twocols}%[.45]
  $\VARIABLES big,\ small$ \V{.4}
  $Init == 
  \begin{conj}
  big = 0 \\ small = 0
  \end{conj}
  $
  \midcol
  \verb*|VARIABLES big, small|\V{.4}
  \verb*|Init == /\ big = 0| \\
  \verb*|        /\ small = 0|
  \end{twocols}
  \end{display}
  \smallskip
\end{ch}
%%%%%%%%%%%%%%%
% % in the same paragraph
% 4th
%%%%%%%%%%%%%%%
\begin{en}
Each of the three possible kinds of steps has two possibilities---one
for each jug (each first jug for the third type).  This suggests
writing the next state action as the disjunction of six formulas, each
allowing one of these six possible kinds of step.  We can therefore
define the next-state action, which by convention is called $Next$, as
follows:
\medskip
\begin{twocols}%[.45]
\begin{notla}
Next == \/ FillSmall 
        \/ FillBig    
        \/ EmptySmall 
        \/ EmptyBig    
        \/ SmallToBig    
        \/ BigToSmall 
\end{notla}
\begin{tlatex}
\@x{ Next \.{\defeq} \.{\lor} FillSmall}%
\@x{\@s{39.83} \.{\lor} FillBig}%
\@x{\@s{39.83} \.{\lor} EmptySmall}%
\@x{\@s{39.83} \.{\lor} EmptyBig}%
\@x{\@s{39.83} \.{\lor} SmallToBig}%
\@x{\@s{39.83} \.{\lor} BigToSmall}%
\end{tlatex}
\midcol
\begin{verbatim*}
Next == \/ FillSmall 
        \/ FillBig    
        \/ EmptySmall 
        \/ EmptyBig    
        \/ SmallToBig    
        \/ BigToSmall 
\end{verbatim*}
\end{twocols}
\medskip
\end{en}
% in the same paragraph 
\begin{ch}
  每一类可能的\tlastep{}都有两种情况,
  分别对应于不同的水壶
  (对于第三类\tlastep{},第一个水壶可以是两个水壶中的任意一个)。
  这提示我们将\tlanextstateaction{}写成 6 个公式的析取式,
  每个公式对应一类可能的\tlastep{}。
  因此,我们如下定义\tlanextstateaction{},并按照惯例将其命名为 $Next$:
  \medskip
  \begin{twocols}%[.45]
  \begin{notla}
  Next == \/ FillSmall 
	  \/ FillBig    
	  \/ EmptySmall 
	  \/ EmptyBig    
	  \/ SmallToBig    
	  \/ BigToSmall 
  \end{notla}
  \begin{tlatex}
  \@x{ Next \.{\defeq} \.{\lor} FillSmall}%
  \@x{\@s{39.83} \.{\lor} FillBig}%
  \@x{\@s{39.83} \.{\lor} EmptySmall}%
  \@x{\@s{39.83} \.{\lor} EmptyBig}%
  \@x{\@s{39.83} \.{\lor} SmallToBig}%
  \@x{\@s{39.83} \.{\lor} BigToSmall}%
  \end{tlatex}
  \midcol
  \begin{verbatim*}
  Next == \/ FillSmall 
	  \/ FillBig    
	  \/ EmptySmall 
	  \/ EmptyBig    
	  \/ SmallToBig    
	  \/ BigToSmall 
  \end{verbatim*}
  \end{twocols}
  \medskip
\end{ch}
%%%%%%%%%%%%%%%
% % in the same paragraph
% 5th
%%%%%%%%%%%%%%%
\begin{en}
The definitions of the six formulas $FillSmall$, \ldots\,, $BigToSmall$,
which often called 
 \tindex{1}{subaction}%
\emph{subactions} of the next-state action, must
precede the definition of $Next$ in the module.  (In \tlaplus, a
symbol must be defined or declared before it can be used.)  Let's now
define them.  
\end{en}
% in the same paragraph 
\begin{ch}
  这 6 个公式 —— 通常称为\tlanextstateaction{}的%
   \tindex{1}{subaction}%
  \emph{\tlasubaction{}} —— 
  的定义必须出现在 $Next$ 的定义之前。
  (在 \tlaplus 中,一个符号需要先定义或者先声明之后才可使用。)
  现在,我们定义这些\tlasubaction{}。
\end{ch}
%%%%%%%%%%%%%%%

% 6th
%%%%%%%%%%%%%%%
\begin{en}
Most programmers would expect the definition of $FillSmall$ to be
 \[ FillSmall == small' = 3 \]
This formula is certainly satisfied by a step like
 \[ \dhstate{2}{1} -> \dhstate{2}{3}
 \]
However, the formula is also satisfied by this step
  \[ \dhstate{2}{1} -> \dhstate{\sqrt{42}}{3}
 \]
because substituting
 \[ big <- 2, \ \ small <- 1, \ \ big' <- \sqrt{42}, \ \ small' <- 3
 \]
in the formula produces the true formula $3=3$.  
\end{en}

\begin{ch}
  大多数程序员期望看到如下定义的 $FillSmall$ 公式:
  \[ FillSmall == small' = 3 \]
  该公式当然可以被如下所示的\tlastep{}所满足:
  \[ \dhstate{2}{1} -> \dhstate{2}{3}
  \]
  但是,它也可以被\tlastep{}
  \[ \dhstate{2}{1} -> \dhstate{\sqrt{42}}{3}
  \]
  所满足,这是因为在公式中施行替换
  \[ big <- 2, \ \ small <- 1, \ \ big' <- \sqrt{42}, \ \ small' <- 3
  \]
  会产生永真式 $3=3$。
\end{ch}
%%%%%%%%%%%%%%%
%  % in the same paragraph
% 7th
%%%%%%%%%%%%%%%
\begin{en}
Since a step that
fills the small jug should leave the contents of the big jug
unchanged, the subaction $FillSmall$ must assert that $big'$ equals
$big$.  With this observation, the definitions of the first four
subactions are obvious:
 \medskip
\begin{twocols}%[.45]
\begin{notla}
FillSmall  == /\ small' = 3 
              /\ big' = big

FillBig    == /\ big' = 5 
              /\ small' = small

EmptySmall == /\ small' = 0 
              /\ big' = big

EmptyBig   == /\ big' = 0 
              /\ small' = small
\end{notla}
\begin{tlatex}
\@x{ FillSmall\@s{13.03} \.{\defeq} \.{\land} small' \.{=} 3}%
\@x{\@s{72.70} \.{\land} big' \.{=} big}%
\@pvspace{8.0pt}%
\@x{ FillBig\@s{23.00} \.{\defeq} \.{\land} big' \.{=} 5}%
\@x{\@s{72.70} \.{\land} small' \.{=} small}%
\@pvspace{8.0pt}%
\@x{ EmptySmall \.{\defeq} \.{\land} small' \.{=} 0}%
\@x{\@s{72.70} \.{\land} big' \.{=} big}%
\@pvspace{8.0pt}%
\@x{ EmptyBig\@s{9.97} \.{\defeq} \.{\land} big' \.{=} 0}%
\@x{\@s{72.70} \.{\land} small' \.{=} small}%
\end{tlatex}
\midcol
\begin{verbatim*}
FillSmall  == /\ small' = 3 
              /\ big' = big

FillBig    == /\ big' = 5 
              /\ small' = small

EmptySmall == /\ small' = 0 
              /\ big' = big

EmptyBig   == /\ big' = 0 
              /\ small' = small
\end{verbatim*} 
\end{twocols} 
\medskip
\end{en}
% in the same paragraph 
\begin{ch}
  由于``将小水壶装满''这一\tlastep{}应该保持大水壶里的水量不变,
  所以\tlasubaction{} $FillSmall$ 必须保证 $big'$ 等于 $big$。
  \fixme{基于这种观察},
  前四个\tlasubaction{}的定义是显而易见的:
  \medskip
  \begin{twocols}%[.45]
  \begin{notla}
  FillSmall  == /\ small' = 3 
		/\ big' = big

  FillBig    == /\ big' = 5 
		/\ small' = small

  EmptySmall == /\ small' = 0 
		/\ big' = big

  EmptyBig   == /\ big' = 0 
		/\ small' = small
  \end{notla}
  \begin{tlatex}
  \@x{ FillSmall\@s{13.03} \.{\defeq} \.{\land} small' \.{=} 3}%
  \@x{\@s{72.70} \.{\land} big' \.{=} big}%
  \@pvspace{8.0pt}%
  \@x{ FillBig\@s{23.00} \.{\defeq} \.{\land} big' \.{=} 5}%
  \@x{\@s{72.70} \.{\land} small' \.{=} small}%
  \@pvspace{8.0pt}%
  \@x{ EmptySmall \.{\defeq} \.{\land} small' \.{=} 0}%
  \@x{\@s{72.70} \.{\land} big' \.{=} big}%
  \@pvspace{8.0pt}%
  \@x{ EmptyBig\@s{9.97} \.{\defeq} \.{\land} big' \.{=} 0}%
  \@x{\@s{72.70} \.{\land} small' \.{=} small}%
  \end{tlatex}
  \midcol
  \begin{verbatim*}
  FillSmall  == /\ small' = 3 
		/\ big' = big

  FillBig    == /\ big' = 5 
		/\ small' = small

  EmptySmall == /\ small' = 0 
		/\ big' = big

  EmptyBig   == /\ big' = 0 
		/\ small' = small
  \end{verbatim*} 
  \end{twocols} 
  \medskip
\end{ch}
%%%%%%%%%%%%%%%
% % in the same paragraph
% 8th
%%%%%%%%%%%%%%%
\begin{en}
The definitions of the last two, $SmallToBig$ and $BigToSmall$, are
trickier because each has two cases.  Let's consider $SmallToBig$.
We can express the two possibilities as the disjunction of two
formulas:
\begin{display}
\begin{notla}
SmallToBig == \/ /\ big + small > 5
                 /\ big' = 5
                 /\ small' = small - (5 - big)
              \/ /\ big + small =< 5
                 /\ big' = big + small
                 /\ small' = 0
\end{notla}
\begin{tlatex}
\@x{ SmallToBig \.{\defeq} \.{\lor} \.{\land} big \.{+} small \.{>} 5}%
\@x{\@s{81.62} \.{\land} big' \.{=} 5}%
\@x{\@s{81.62} \.{\land} small' \.{=} small \.{-} ( 5 \.{-} big )}%
%
\marginpar{If the water doesn't all fit in the big jug, then $5-big$
           gallons are poured out of the little jug.} 
%
\@x{\@s{70.51} \.{\lor} \.{\land} big \.{+} small \.{\leq} 5}%
\@x{\@s{81.62} \.{\land} big' \.{=} big \.{+} small}%
\@x{\@s{81.62} \.{\land} small' \.{=} 0}%
\end{tlatex}
\end{display}
\end{en}
% in the same paragraph 
\begin{ch}
  后两个\tlasubaction{} $SmalltoBig$ 和 $BigToSmall$ 的定义稍显复杂,
  因为每个都有两种情况。
  考虑 $SmallToBig$。
  我们可以将它的两种情况表示成两个公式的析取:
  \begin{display}
  \begin{notla}
  SmallToBig == \/ /\ big + small > 5
		   /\ big' = 5
		   /\ small' = small - (5 - big)
		\/ /\ big + small =< 5
		   /\ big' = big + small
		   /\ small' = 0
  \end{notla}
  \begin{tlatex}
  \@x{ SmallToBig \.{\defeq} \.{\lor} \.{\land} big \.{+} small \.{>} 5}%
  \@x{\@s{81.62} \.{\land} big' \.{=} 5}%
  \@x{\@s{81.62} \.{\land} small' \.{=} small \.{-} ( 5 \.{-} big )}%
  %
  \marginpar{如果大水壶装不下小水壶中所有的水,
    那么从小水壶中倒入大水壶中的水量是 $5-big$ 加仑。} 
  %
  \@x{\@s{70.51} \.{\lor} \.{\land} big \.{+} small \.{\leq} 5}%
  \@x{\@s{81.62} \.{\land} big' \.{=} big \.{+} small}%
  \@x{\@s{81.62} \.{\land} small' \.{=} 0}%
  \end{tlatex}
  \end{display}
\end{ch}
%%%%%%%%%%%%%%%
% % in the same paragraph
% 9th
%%%%%%%%%%%%%%%
\begin{en}
This definition is fine, but it can be expressed more compactly.
Observe that a $SmallToBig$ step sets the value of $big$ to the
smaller of $big+small$ and 5.  Let's define $Min$ so that $Min(m, n)$
is the smaller of $m$ and $n$, if $m$ and $n$ are numbers.
\begin{twocols}
\begin{notla}
Min(m,n) == IF m < n THEN m ELSE n
\end{notla}
\begin{tlatex}
\@x{ Min ( m ,\, n ) \.{\defeq} {\IF} m \.{<} n \.{\THEN} m \.{\ELSE} n}%
\end{tlatex}
\midcol
\verb|Min(m,n) == IF m < n THEN m ELSE n|
\end{twocols}
Since the amount of water removed from the small jug equals the amount
added to the big jug, we can define $SmallToBig$ by:
\begin{display}
\begin{notla}
SmallToBig == /\ big' = Min(big + small, 5)
              /\ small' = small - (big' - big)
\end{notla}
\begin{tlatex}
 \@x{ SmallToBig \.{\defeq} \.{\land} big' \.{=} Min ( big \.{+} small
 ,\, 5 )}%
 \@x{\@s{70.51} \.{\land} small' \.{=} small\@s{0.54} \.{-} ( big'
 \.{-} big )}%
\end{tlatex}
\end{display}
\end{en}
% in the same paragraph 
\begin{ch}
  这个定义还不错,但是可以表达得更紧凑些。
  注意到 $SmallToBig$ \tlastep{} 将 $big$ 设置为 $big+small$ 与 5 中的较小值。
  我们定义 $Min$, 使得 $Min(m,n)$ 等于 $m$ 与 $n$ 的较小值
  (如果 $m$ 与 $n$ 是数字)。
  \begin{twocols}
  \begin{notla}
  Min(m,n) == IF m < n THEN m ELSE n
  \end{notla}
  \begin{tlatex}
  \@x{ Min ( m ,\, n ) \.{\defeq} {\IF} m \.{<} n \.{\THEN} m \.{\ELSE} n}%
  \end{tlatex}
  \midcol
  \verb|Min(m,n) == IF m < n THEN m ELSE n|
  \end{twocols}
  因为从小水壶中倒出的水量等于大水壶增加的水量,
  所以我们可以如下定义 $SmallToBig$:
  \begin{display}
  \begin{notla}
  SmallToBig == /\ big' = Min(big + small, 5)
		/\ small' = small - (big' - big)
  \end{notla}
  \begin{tlatex}
   \@x{ SmallToBig \.{\defeq} \.{\land} big' \.{=} Min ( big \.{+} small
   ,\, 5 )}%
   \@x{\@s{70.51} \.{\land} small' \.{=} small\@s{0.54} \.{-} ( big'
   \.{-} big )}%
  \end{tlatex}
  \end{display}
\end{ch}
%%%%%%%%%%%%%%%
% % in the same paragraph
% 10th
%%%%%%%%%%%%%%%
\begin{en}
This definition has one drawback.  When reading an action formula, we
often want to see how a particular variable's value changes.  This is
easiest to do if the value of the primed variable is expressed as a
function of the values of the unprimed variables.  However, this
definition expresses the value of $small'$ in terms of $big'$ as well
as of $big$ and $small$.  We could fix that by writing the definition as:
\begin{display}
\begin{notla}
SmallToBig == /\ big' = Min(big + small, 5)
              /\ small' = small - (Min(big + small, 5) - big)
\end{notla}
\begin{tlatex}
 \@x{ SmallToBig \.{\defeq} \.{\land} big' \.{=} Min ( big \.{+} small
 ,\, 5 )}%
 \@x{\@s{70.51} \.{\land} small' \.{=} small\@s{0.54} \.{-} ( Min ( big
 \.{+} small ,\, 5 ) \.{-} big )}%
\end{tlatex}
\end{display}
\end{en}
% in the same paragraph 
\begin{ch}
  这个定义有一个缺陷。
  当阅读一个\tlaaction{}公式时,
  我们通常想知道一个特定变量的值是如何变化的。
  如果\tlaprimedvariable{}的值被表示成\tlaunprimedvariable{}的值的函数,
  那么这\fixme{很容易}做到。
  但是,该定义用 $big'$(以及 $big$ 与 $small$)的值来表示 $small'$ 的值。
  我们可以通过重写定义修复这个缺陷:
  \begin{display}
  \begin{notla}
  SmallToBig == /\ big' = Min(big + small, 5)
		/\ small' = small - (Min(big + small, 5) - big)
  \end{notla}
  \begin{tlatex}
   \@x{ SmallToBig \.{\defeq} \.{\land} big' \.{=} Min ( big \.{+} small
   ,\, 5 )}%
   \@x{\@s{70.51} \.{\land} small' \.{=} small\@s{0.54} \.{-} ( Min ( big
   \.{+} small ,\, 5 ) \.{-} big )}%
  \end{tlatex}
  \end{display}
\end{ch}
%%%%%%%%%%%%%%%
% % in the same paragraph
% 11th
%%%%%%%%%%%%%%%
\begin{en}
However, it's better not to repeat the expression $Min(big + small, 5)$.
I find it more elegant to write the action in terms the amount
of water poured from one jug to the other.  I prefer writing the
action as follows, using the \tlaplus\ 
  \rref{math}{let-in}{\textsc{let}\,/\,\textsc{in} construct},
which allows us to make local definitions within an expression.
 \medskip
\begin{twocols}%[.45]
\setboolean{shading}{true}
\begin{notla}
SmallToBig == 
  LET poured == Min(big + small, 5) - big
  IN  /\ big'   = big + poured
      /\ small' = small - poured
\end{notla}
\begin{tlatex}
\@x{ SmallToBig \.{\defeq}}%
 \@x{\@s{8.2} \.{\LET} poured \.{\defeq} Min ( big \.{+} small ,\, 5 ) \.{-}
 big}%
\@x{\@s{8.2} \.{\IN} \.{\land} big'\@s{10.88} \.{=} big \.{+} poured}%
\@x{\@s{28.59} \.{\land} small' \.{=} small \.{-} poured}%
\end{tlatex}
\midcol
\begin{verbatim*}
SmallToBig == 
  LET poured == Min(big + small, 5) - big
  IN  /\ big'   = big + poured
      /\ small' = small - poured
\end{verbatim*}
\end{twocols}
 \medskip
(Note that $poured$ equals $Min(small, 5-big)$.)
\end{en}
% in the same paragraph
\begin{ch}
  不过,最好不要重复使用表达式 $Min(big + small, 5)$。
  我发现使用\fixme{从一个水壶到另一个水壶的倒水量}来表达该\tlaaction{}更为简练。
  我更喜欢使用 \tlaplus\ 提供的
    \rref{math}{let-in}{\textsc{let}\,/\,\textsc{in} \tlaconstruct{}}
  将它写成如下形式(\textsc{let}\,/\,\textsc{in} 允许我们\fixme{在表达式内部给出局部定义}):
  \medskip
  \begin{twocols}%[.45]
  \setboolean{shading}{true}
  \begin{notla}
  SmallToBig == 
    LET poured == Min(big + small, 5) - big
    IN  /\ big'   = big + poured
	/\ small' = small - poured
  \end{notla}
  \begin{tlatex}
  \@x{ SmallToBig \.{\defeq}}%
   \@x{\@s{8.2} \.{\LET} poured \.{\defeq} Min ( big \.{+} small ,\, 5 ) \.{-}
   big}%
  \@x{\@s{8.2} \.{\IN} \.{\land} big'\@s{10.88} \.{=} big \.{+} poured}%
  \@x{\@s{28.59} \.{\land} small' \.{=} small \.{-} poured}%
  \end{tlatex}
  \midcol
  \begin{verbatim*}
  SmallToBig == 
    LET poured == Min(big + small, 5) - big
    IN  /\ big'   = big + poured
	/\ small' = small - poured
  \end{verbatim*}
  \end{twocols}
   \medskip
  (注意 $poured$ 等于 $Min(small, 5-big)$。)
\end{ch}
%%%%%%%%%%%%%%%
% in the same paragraph
% 12th
%%%%%%%%%%%%%%%
\begin{en}
The definition of the $BigToSmall$ subaction is similar.
 \medskip
\begin{twocols}
\begin{notla}
BigToSmall == 
  LET poured == Min(big + small, 3) - small
  IN  /\ big'   = big - poured
      /\ small' = small + poured
\end{notla}
\begin{tlatex}
\@x{ BigToSmall \.{\defeq}}%
 \@x{\@s{8.2} \.{\LET} poured \.{\defeq} Min ( big \.{+} small ,\, 3 ) \.{-}
 small}%
\@x{\@s{8.2} \.{\IN} \.{\land} big' \@s{10.88} \.{=} big \.{-} poured}%
\@x{\@s{28.59} \.{\land} small' \.{=} small \.{+} poured}%
\end{tlatex}
\midcol
\begin{verbatim*}
BigToSmall == 
  LET poured == Min(big + small, 3) - small
  IN  /\ big'   = big - poured
      /\ small' = small + poured
\end{verbatim*}
\end{twocols}
 \medskip
\end{en}
% in the same paragraph
\begin{ch}
  \tlasubaction{} $BigToSmall$ 的定义与之类似。
 \medskip
  \begin{twocols}
  \begin{notla}
  BigToSmall == 
    LET poured == Min(big + small, 3) - small
    IN  /\ big'   = big - poured
	/\ small' = small + poured
  \end{notla}
  \begin{tlatex}
  \@x{ BigToSmall \.{\defeq}}%
   \@x{\@s{8.2} \.{\LET} poured \.{\defeq} Min ( big \.{+} small ,\, 3 ) \.{-}
   small}%
  \@x{\@s{8.2} \.{\IN} \.{\land} big' \@s{10.88} \.{=} big \.{-} poured}%
  \@x{\@s{28.59} \.{\land} small' \.{=} small \.{+} poured}%
  \end{tlatex}
  \midcol
  \begin{verbatim*}
  BigToSmall == 
    LET poured == Min(big + small, 3) - small
    IN  /\ big'   = big - poured
	/\ small' = small + poured
  \end{verbatim*}
  \end{twocols}
 \medskip
\end{ch}
%%%%%%%%%%%%%%%
% in the same paragraph
% 13th
%%%%%%%%%%%%%%%
\begin{en}
We should also define a type invariant.  Clearly, the values of
both $big$ and $small$ should be natural numbers, with $big\leq 5$
and $small\leq3$.  To express this, we use the operator 
  \ctindex{1}{+3i@\mmath{\icmd{dd}} (integer interval)}{+3i}%
$\dd$ defined in the $Integers$ module so that $i\dd j$ is the set of
integers from $i$ through $j$.  More precisely, $i\dd j$ is defined to
be the set of all integers $k$ such that $i\leq k$ and $k\leq j$ hold,
so $i\dd j$ is the empty set if $j<i$.  The definition of $\dd$ is:%
 \marginpar{\rref{math}{set-constructors}{$\{x \in S : P(x)\}$ is the
     subset of $S$ consisting of all its elements $x$
     satisfying $P(x)$.}}%
 \[ i \dd j == \{k \in Int : (i \leq k) /\ (k \leq j)\}
 \]
where 
  \ctindex{1}{Int@\mmath{Int}}{Int}%
$Int$ is defined in the $Integers$ module to be the set of all
integers.   The type invariant is then:
 \medskip
\begin{twocols}
\begin{notla}
TypeOK == /\ big   \in 0..5
          /\ small \in 0..3 
\end{notla}
\begin{tlatex}
\@x{ TypeOK \.{\defeq} \.{\land} big\@s{10.88} \.{\in} 0 \.{\dotdot} 5}%
\@x{\@s{56.14} \.{\land} small \.{\in} 0 \.{\dotdot} 3}%
\end{tlatex}
\midcol
\begin{verbatim*}
TypeOK == /\ big   \in 0..5
          /\ small \in 0..3 
\end{verbatim*}
\end{twocols}
 \medskip
This definition is best put right after the declaration of the
variables $big$ and $small$.
\end{en}
% in the same paragraph
\begin{ch}
  我们还应该定义一个\tlatypeinvariant{}。
  显然,变量 $big$ 和 $small$ 的值都是自然数,
  并且 $big \leq 5$,$small \leq 3$。
  为了\fixme{表达这一点},
  我们使用在 $Integers$ 模块中定义的%
    \ctindex{1}{+3i@\mmath{\icmd{dd}} (integer interval)}{+3i}%
  $\dd$ 操作符。
  $i\dd j$ 表示从 $i$ 到 $j$ 的整数集合。
  更精确地讲,$i\dd j$ 被定义为满足 $i\leq k$ 与 $k\leq j$ 条件
  的整数 $k$ 构成的集合。
  如果 $j < i$,则 $i\dd j$ 是空集。
  操作符 $\dd$ 的定义是:
  \marginpar{\rref{math}{set-constructors}{$\{x \in S : P(x)\}$ 
    是 $S$ 中所有满足 $P(x)$ 的元素 $x$ 构成的 $S$ 的子集。}}%
  \[ i \dd j == \{k \in Int : (i \leq k) /\ (k \leq j)\}
  \]
  这里, 
    \ctindex{1}{Int@\mmath{Int}}{Int}%
  $Int$ 是所有整数构成的集合,定义在 $Integers$ 模块中。
  \tlatypeinvariant{}的定义如下所示:
   \medskip
  \begin{twocols}
  \begin{notla}
  TypeOK == /\ big   \in 0..5
            /\ small \in 0..3 
  \end{notla}
  \begin{tlatex}
  \@x{ TypeOK \.{\defeq} \.{\land} big\@s{10.88} \.{\in} 0 \.{\dotdot} 5}%
  \@x{\@s{56.14} \.{\land} small \.{\in} 0 \.{\dotdot} 3}%
  \end{tlatex}
  \midcol
  \begin{verbatim*}
  TypeOK == /\ big   \in 0..5
            /\ small \in 0..3 
  \end{verbatim*}
  \end{twocols}
   \medskip
  该定义最好紧跟在变量 $big$ 与 $small$ 的声明语句之后。
\end{ch}
%%%%%%%%%%%%%%%
\subsection{Applying TLC}


Let's now test our spec.  \popref{create-new-model}{Create a new TLC
model}.  Since we used the conventional names for the initial
predicate and next-state action, the Toolbox fills in the
\textsf{What is the behavior spec?} section of the model.  Add 
$TypeOK$ as an invariant in the \textsf{What to check?} section
and run TLC on the model.  TLC should find no errors.  It will 
report that the system has 16 distinct reachable states.

The \emph{Die Hard} problem makes learning to write \tlaplus\
specifications a little more fun.  But could a \tlaplus\ specification
have helped our heroes---especially when they had to solve the problem
before a bomb exploded?  The answer is yes---at least if they were
carrying a computer and were able to write the spec very quickly.
They then could have let TLC solve the problem for them.

Remember that their problem was to put 4 gallons of water in a jug,
which of course had to be the big jug.  All they had to do was have TLC
check an invariant asserting that there are not 4 gallons of water in
the big jug.  Add the invariant 
  \marginpar{\rref{summary}{ascii}{How to type $\neq$\,.}}%
$big # 4$ 
to your model and run TLC on
it.  TLC will report that the invariant is violated, and the error
trace it produces to demonstrate the violation is a solution to the
problem.  Moreover, if you select 
1~\popref{worker-threads}{worker thread} in the \textsf{How
to run?} section of the \textsf{Model Overview} page, TLC will produce
a minimal-length error trace.  The solution it produces is then one
with that takes fewest steps possible---namely, six.

\subsection{Expressing the Problem in PlusCal}

Although they did solve the problem, the \emph{Die Hard} heroes did not
seem to be mathematically sophisticated.  They would probably have
preferred to write their specification in PlusCal.  Let's now see how
they could have done that.

Create a new specification called $PDieHard$.  The algorithm will use
arithmetic operations and the $Min$ operator, so copy the \textsc{extends}
statement and the definition of $Min$ from the $DieHard$ spec and
put them at the beginning of module $PDieHard$. 


The algorithm is inserted in a comment.  It begins with its name,
which we take to be $DieHard$, and with a 
  \marginpar{The PlusCal keywords {\rm\textbf{variable}}
             and {\rm\textbf{variables}} are synonyms.}
\textbf{variables} statement
that declares the variables and their initial values.  The algorithm
looks like this:
\begin{display}
\begin{nopcal}
--algorithm DieHard  {
     variables big = 0, small = 0;  
     {  \* body of the algorithm
     }
}
\end{nopcal}
\begin{tlatex}
\@x{ {\p@mmalgorithm} DieHard\@s{4.1} {\p@lbrace}}%
\@x{\@s{20.5} {\p@variables} big \.{=} 0 ,\, small \.{=} 0 {\p@semicolon}}%
\@x{\@s{20.5} {\p@lbrace}\@s{4.1}}%
\@y{%
  body of the algorithm
}%
\@xx{}%
\@x{\@s{20.5} {\p@rbrace}}%
\@x{ {\p@rbrace}}%
\end{tlatex}
\end{display}
We now write the body of the algorithm.
The \tlaplus\ specification defines the next-state action $Next$ to be
the disjunction of six subactions.  We first see how to express each
of those subactions as a PlusCal statement.  

It's easy to express the
first four subactions, $FillSmall$, \ldots\,, $EmptyBig$.  For example,
$FillSmall$ is expressed by the assignment statement
 \[ small := 3 \]
There's no need to assert that the value of $big$ is unchanged.
PlusCal is like a very simple programming language in that a statement
that does not explicitly change a variable leaves the value of the
variable unchanged.  (This makes it unlike many real programming
languages.)

The $SmallToBig$ and $BigToSmall$ subactions each have two cases.
It's easy to express them with \textbf{if} statements.  For example,
the $SmallToBig$ subaction could be described by
\begin{display}
\begin{nopcal}
if (big + small > 5) { small := small - (5 - big); 
                       big := 5                     }
else { big := big + small ;
       small := 0           }
\end{nopcal}
\begin{tlatex}
 \@x{ {\p@if} {\p@lparen} big \.{+} small \.{>} 5 {\p@rparen} {\p@lbrace}
 small \.{:=} small \.{-} ( 5 \.{-} big ) {\p@semicolon}}%
\@x{\@s{108.30} big \.{:=} 5\@s{82.0} {\p@rbrace}}%
\@x{ {\p@else} {\p@lbrace} big \.{:=} big \.{+} small {\p@semicolon}}%
\@x{\@s{32.02} small \.{:=} 0\@s{41.0} {\p@rbrace}}% was 35.02
\end{tlatex}
\end{display}
As we would expect of a programming language, the order of assignment
statements matters.  If we changed the order of the two assignments in
the \textbf{else} clause, the assignment to $big$ would be performed
with $small$ equal to 0, so $big$ would be unchanged.

Although this \textbf{if} statement correctly describes the
$SmallToBig$ subaction, it isn't very elegant.  It would be nicer to
copy the way the subaction is defined in \tlaplus\ and write:
\begin{display}
\begin{nopcal}
big   := big + poured ;                 
small := small - poured
\end{nopcal}
\begin{tlatex}
\@x{ big\@s{10.88} \.{:=} big \.{+} poured {\p@semicolon}}%
\@x{ small \.{:=} small \.{-} poured}%
\end{tlatex}
\end{display}
where $poured$ is defined locally to equal $Min (big + small, 5) - big$.
This is written in PlusCal as follows using a 
     \ctindex{1}{with (PlusCal keyword)@\icmd{with} 
                   (PlusCal keyword)}{with-pcal}%
     \target{main:with-eq}%
\textbf{with} statement.
\begin{display}
\begin{nopcal}
with (poured = Min (big + small, 5) - big)  
  { big   := big + poured ;                 
    small := small - poured  }
\end{nopcal}
\begin{tlatex}
 \@x{ {\p@with} {\p@lparen} poured \.{=} Min ( big \.{+} small ,\, 5 ) \.{-}
 big {\p@rparen}}%
 \@x{\@s{8.2} {\p@lbrace} big\@s{10.88} \.{:=} big \.{+} poured
 {\p@semicolon}}%
\@x{\@s{17.78} small \.{:=} small \.{-} poured\@s{4.1} {\p@rbrace}}%
\end{tlatex}
\end{display}
The $BigToSmall$ subaction is described by a similar \textbf{with}
statement.

In the \tlaplus\ spec, the next-state action is the disjunction of the
six subactions, meaning that a step is either a $FillBig$ step or a
$FillSmall$ step or \ldots\ or a $BigToSmall$ step.  Such a disjunction is
expressed in PlusCal by an \textbf{either}\,/\,\textbf{or} statement.
So, we can write this disjunction as follows:
\begin{display}
\begin{nopcal}
either big := 5    
or     small := 3  
...
or     with (poured = Min (big + small, 3) - small)  
         { big   := big - poured ;                 
           small := small + poured }
\end{nopcal}
\begin{tlatex}
\@x{ {\p@either} big \.{:=} 5}%
\@x{ {\p@or}\@s{18.84} small \.{:=} 3}%
\@x{ \.{\dots}}%
 \@x{ {\p@or}\@s{16.4} {\p@with} {\p@lparen} poured \.{=} Min ( big \.{+}
 small ,\, 3 ) \.{-} small {\p@rparen}}%
 \@x{\@s{38.91} {\p@lbrace} big\@s{10.88} \.{:=} big \.{-} poured
 {\p@semicolon}}%
\@x{\@s{48.50} small \.{:=} small \.{+} poured {\p@rbrace}}%
\end{tlatex}
\end{display}
If the body of the algorithm consisted only of this
\textbf{either}\,/\,\textbf{or} statement, an execution of the
algorithm would execute the statement once and then halt.  The
\tlaplus\ spec describes a system that keeps taking steps forever.  To
get our algorithm do the same, we put the
\textbf{either}\,/\,\textbf{or} in a \textbf{while}\,(\TRUE) loop.

The complete algorithm 
 \popref{pdiehard}{is here}, and 
the \textsc{ascii} version 
 \popref{pdiehard-ascii}{is here}.
Since the
PlusCal version lacks the helpful subaction names, I have added comments to
explain each clause of the \textbf{either}\,/\,\textbf{or} statement.
(The comments are shaded in the pretty-printed version.)
