\documentclass[fleqn,leqno]{article}
\usepackage{hypertlabook}
\pdftitle{A Proof of GCD3}
\begin{popup}
\subsection*{A Proof of GCD3}

\begin{proof}
\pflongnumbers
\pflongindent
\beforePfSpace{0pt}
\afterPfSpace{0pt}
\interStepSpace{0pt}


\step{1}{It suffices to assume that $m$ and $n$ are positive
integers and $d$ is any integers,
and to prove that $d$ divides both $m$ and $n$ iff % \popref{iff}{iff}
$d$ divides both $m$ and $n-m$.}
\vspace{.201em}
\begin{proof}
\pf\ Since the gcd of two numbers is the largest integer that
divides both of them, it suffices to show that $m$ and $n$
have the same common divisors as $m$ and $n-m$.
\end{proof}

\vspace{.6em}

\step{2}{If $d$ divides both $m$ and $n$, then $d$
divides both $m$ and $n-m$.} \vspace{.201em}

\begin{proof}
\pf\ That $d$ divides $m$ follows from the assumptions; that it divides
$n-m$ follows from the assumptions and Lemma~Div.
\end{proof}

\vspace{.6em}

\step{3}{If $d$ divides both $m$ and $n-m$, then 
$d$ divides both $m$ and $n$.} \vspace{.201em}
\begin{proof}
\pf\ That $d$ divides $m$ follows from the assumptions; that it divides
$n$ follows from the assumptions, Lemma~Div, and the simple algebraic
relation:
 $n = m + (n-m)$.
\end{proof}

\vspace{.6em}

\qedstep
\vspace{.201em}
\begin{proof}
\pf\ GCD3 follows from steps \stepref{1}, \stepref{2}, and \stepref{3}.
\end{proof}
\end{proof}

\end{popup}
\makepopup