\documentclass[fleqn,leqno]{article}
\usepackage{hypertlabook}
%\pdffalse
\makeindex

%%% Hack for showing index elements.  
%%
% \let\mmath=\ensuremath
% \def\icmd#1{\csname#1\endcsname}
% \renewcommand{\tindex}[2]{\marginpar{\red #2 (#1)}}
% \renewcommand{\ctindex}[3]{\marginpar{\red #2 (#1)}}

% Definitions of proof state components
\newcommand{\G}{\textbf{G}}
\newcommand{\Sym}{\ensuremath{\mathcal{S}}}
\newcommand{\FK}{\ensuremath{\mathcal{F}_{K}}}
\newcommand{\FU}{\ensuremath{\mathcal{F}_{U}}}
\newcommand{\DK}{\ensuremath{\mathcal{D}_{K}}}
\newcommand{\DU}{\ensuremath{\mathcal{D}_{U}}}
\newcommand{\BB}{\ensuremath{\mathcal{B}}}



\pdftitle{Proofs}  % This didn't work with \tlaplus
\title{\bf The \tlaplus\ Tutorial}
\author{Leslie Lamport}
\date{\today}
\file{proof}



\setcounter{section}{\prooftrack}
\addtocounter{section}{-1}

%\small
\begin{document}
\renewcommand{\contentsname}{The \protect\tlaplus\ Proof Track\protect\target{top}}
\showversions
\tableofcontents
\hideversions

\newpage

  \tindex{2}{TLAPS}%
   \vspace{-\baselineskip}%
\btarget{tlaps}
\section{About Proofs and Proving}

\subsection{About Proofs}

When writing proofs of properties of systems or algorithms, we should
not look at mathematicians' 
 \ctindex{1}{proof!mathematicians'}{proof-math}%
%
proofs for models.  There are three
reasons why the way (almost all) mathematicians write proofs do
not work for our proofs:
\begin{itemize}
\item The theorems mathematicians prove are very different from what
we are proving.  Mathematical theorems are usually deep, being based
on knowledge that has been developed over centuries.  Our theorems are
shallow but wide.  They generally use only simple mathematics, but
require checking many details.

\item Mathematicians don't care if their theorems are not quite
correct.  Omitting a simple hypothesis such as that a certain set is
nonempty would not even be considered an error.  For the proof of an
algorithm, such an omission means a bug in a ``corner case''.

\item The proofs mathematicians write are unreliable.  Anecdotal
evidence suggests that a significant fraction of published, refereed
mathematical papers contain incorrect theorems---not ones ignoring
corner cases, but results that mathematicians would consider wrong.
Based on a tiny amount of data, I would guess that fraction to be more
than a tenth.  (This includes results that have a correct proof but
are wrong because the proof relies on theorems that are wrong.)

Mathematicians depend on the social process to weed out incorrect
results.  Most theorems are ignored and soon forgotten.  The few
important results are scrutinized by many mathematicians, and errors
in them are eventually discovered.  Two false 19\tth\ century proofs
of the four-color theorem were believed for 11 years.  However, the
greater number of mathematicians and improved communications make it
very likely that an incorrect proof of such a major result would today
be quickly discovered.
\end{itemize}
When a mathematician's style of proof is used to prove properties of
systems, the result is often disastrous.  There is no social process
to find errors an engineer makes in a proof of a system she is
designing.  Even for published algorithms, the social process doesn't
work very well.  One dramatic example is provided by:%
 \tindex{1}{Zave, Pamela}%
\begin{display}
 Pamela Zave.  \emph{Lightweight verification of network protocols: 
 The case of Chord}. AT\&T Technical Report, January 2010.
\end{display}
The 
  \tindex{1}{Chord}%
Chord algorithm was first published in:
\begin{display}
I. Stoica, R. Morris, D. Karger, M. F. Kaashoek, and
H. Balakrishnan. \emph{Chord: A scalable peer-to-peer lookup
service for Internet applications}. Proceedings of
SIGCOMM. ACM, August 2001.
\end{display}
This paper states:
\begin{display}
Three features that distinguish Chord from many other
peer-to-peer lookup protocols are its simplicity, provable
correctness, and provable performance.
\end{display}
However, Zave reports:
\begin{display}
[T]he Chord routing protocol is neither proven nor
correct. The only published proof of correctness excludes
failures from consideration. Even within its
scope the proof does not compel belief, due to ill-defined
terms and missing or unjustified steps. The full protocol
is clearly incorrect, even after bugs with straightforward
fixes have been eliminated. Not one of the six
properties claimed invariant for the full protocol \ldots\
is invariantly true.
\end{display}
Chord is quite well known.  A web search on its title yields many
thousands of references.  The paper won a SIGCOMM Test of Time Award in
2011.  Yet, the errors in the algorithm went unnoticed for almost 10
years.  Zave found them by writing a formal specification of the
algorithm (in the 
  \tindex{1}{Alloy}%
Alloy language) and applying a form of model
checking.  What distinguishes Chord from most published
algorithms is that someone applied formal methods to check if it
really was correct.

One way to make hand proofs more reliable is to structure them
hierarchically.  The \emph{Principles} track of this hyperbook uses
such structured proofs.  Structuring is equally important in handling
the complexity of a formal proof, and formal proofs written in
\tlaplus\ are hierarchically structured.



\subsection{About TLAPS} \xlabel{about-tlaps}

This track is about writing formal \tlaplus\ proofs that can be
checked by TLAPS, the \tlaplus\ Proof System.  TLAPS is a separate
system that is called by the \tlaplus\ Toolbox.  It is not 
distributed with the Toolbox; you can download it 
\hyperref{http://tla.msr-inria.inria.fr/tlaps/}{}{}{from here}.

\tlaplus\ 
  \ctindex{1}{proof!TLA+@\icmd{tlaplus}}{proof-tlaplus}%
proofs are written in a declarative style that has the goal of making
proofs independent of the proof engines used to check them.  A proof
specifies what facts are needed to justify a proof step; it does not
say what to do with those facts.  The writer of a \tlaplus\ proof
should not have to know how the actual proof checking works.  TLAPS
has been designed to come as close as we
could to achieving this goal.

In TLAPS, a proof 
  \tindex{1}{proof manager}%
  \ctindex{1}{proof!manager}{proof-manager}%
manager translates the proof into a set of separate
proof 
  \tindex{1}{proof obligation}%
  \ctindex{1}{obligation!proof}{obligation-proof}%
obligations, checks the ones that are trivially true, and sends
the others to one or more 
  \tindex{1}{backend prover}%
  \ctindex{1}{prover!backend}{prover-backend}%
\popref{backends}{backend provers}.  Its default 
  \tindex{1}{default backend provers}%
behavior is to try these three backend provers: \emph{SMT},
\popref{zenon-isabelle}{\emph{Zenon}, and \emph{Isabelle}}---in that order.

%


% Currently, TLAPS uses two
% primary backend provers: 
%    \tindex{2}{Zenon}%
%   \tindex{2}{Isabelle}%
% \popref{zenon - isabelle}{Zenon and Isabelle}.
% The default is to send an obligation first to Zenon.  If Zenon fails
% to prove it, the obligation is sent to Isabelle.  Zenon can fail by
% either deciding that it can't prove the obligation, or by taking too
% long and being stopped by the proof manager.  (The length of time
% Zenon is given can be adjusted.)
% 
% TLAPS can use other backend provers.  A proof can specify the use of
% certain pseudo-facts that cause the proof manager to send an
% obligation to a particular backend prover.  
% % 
% % As much as possible, such pseudo-facts describe the general nature of
% % the reasoning that is required rather than what prover must be used.
% % 
% One such fact that is very useful is
%   \ctindex{1}{SMT@\mmath{SMT}}{SMT}%
% $SMT$\@.  It invokes a type of backend prover called an \emph{SMT
% Solver}.  TLAPS can use several different SMT solver backends; $SMT$
% invokes the default one.  For most proof obligations, it makes no
% difference which one you use.  An SMT solver is what you will usually
% use for obligations that require only simple arithmetical reasoning.
% Sometimes, it can prove in one step an obligation that other back ends
% require to be broken into many smaller steps.

The TLAPS distribution includes a library of TLA modules that contain
useful theorems that assert properties of mathematical objects like
sets, functions, and sequences, as well as theorems that assert proof
rules such as ones for mathematical induction.  How to access library
modules in your specifications is explained on the
  \helppage{gettingstarted/tla-preferences}{help page} 
for the \tlaplus\
  \helppage{gettingstarted/preferences}{preferences page}.


The proof manager remembers if it has already proved an obligation.
Its default behavior is not to try to prove an obligation that has
already been proved.  TLAPS stores a record of what obligations have
been proved in a 
  \tindex{1}{fingerprint file}%
  \ctindex{1}{file!fingerprint}{file-fingerprint}%
\emph{fingerprint} file.  For a module named $M$, the fingerprint file
is the file named $fingerprints$ inside the folder (directory) $M.tlaps$.
TLAPS may try to prove an obligation that it has already proved for
any of the following reasons: you chose to override the default (using
the 
  \ctindex{1}{Launch Prover (Toolbox command)@\icmd{textsf}{Launch Prover} (Toolbox command)}{launch-prover}%
\textsf{Launch Prover} command), the fingerprint file is not
present (for example, if you are using a different computer), or some
change to TLAPS has made the information in the fingerprint file
obsolete.

Because the correctness of a proof obligation can depend on many
things, some not at all obvious, fingerprinting is subtle.  There have
been bugs that cause the Toolbox to report that an obligation has been
proved despite a change to the specification making the obligation no
longer provable.  If any part of the specification has changed since
you proved something, you should reprove it without using fingerprints
to be sure that it is still proved.  In particular, it's a good idea
to do this after you have finished your complete proof.

Even if nothing has changed, a proof that succeeded may fail if tried
again, or vice-versa.  There are two possible reasons for this.
\begin{itemize}
\item Whether or not a backend prover succeeds before it times out can
depend on what computer you are using and on what other programs are
running.

\item If you are using a different version of TLAPS, then a TLAPS
library file or a backend prover may have changed.  We keep trying to
make the libraries and the backend provers better.  However,
automatic proving involves tradeoffs, and making a library or a prover
work better on some proofs may make it worse on others.  We hope that any
change we make helps more often than it hurts, but we cannot guarantee
that it won't make one of your proofs fail.
\end{itemize}





\newpage

  \ctindex{1}{Euclid's Algorithm!TLAPLUS proof}{euclids-algorithm-proof-of}%
\btarget{theorems}%
\section{Correctness of Euclid's Algorithm} \xlabel{proof:euclid}

We introduce the prover using Euclid's algorithm of
\rref{main}{\xlink{sec:euclid}}{Section~\xref{sec:euclid}} as an
example.  If you have not already done so, you should read the
informal proof of Euclid's algorithm's correctness in
 \rref{main}{\xlink{why-euclid-correct}}{Section~\xref{why-euclid-correct}}.
You should also read about stuttering in 
  \rref{main}{\xlink{fairness-revisited}}{Section~\xref{fairness-revisited}}.


Here is 
\popref{pcaleuclid}{the specification we wrote} in Section~\xref{sec:euclid},
including the definitions of the invariants used in the proof,
and here is 
  \popref{pcaleuclid-ascii}{the \textsc{ascii} version}.
It calls \popref{gcd}{module $GCD$}; here is 
\popref{gcd-ascii}{the \textsc{ascii} version} of that module.
Open the spec in the Toolbox.  You will have to run the PlusCal translator
to make the parsing errors go away.

\subsection{Proving Safety}
The correctness property we will prove is that $PartialCorrectness$ is
an invariant of the algorithm.  This is expressed by the formula
  $Spec => []Correctness$.
As we saw in Section~\xref{sec:euclid}, to do this we have to
prove conditions $I1$, $I2$, and $I3$---except in light of
stuttering we must replace $I2$ by
  \[Inv /\ [Next]_{vars} \,=>\, Inv'\]
Conditions $I1$--$I3$ are the first three steps of the following proof,
which you should add to the end of module $Euclid$.
\begin{display}
\begin{notla}
THEOREM Spec => []PartialCorrectness
<1>1. Init => Inv
<1>2. Inv /\ [Next]_vars => Inv'
<1>3. Inv => PartialCorrectness
<1>4. QED
\end{notla}
\begin{tlatex}
\@x{ {\THEOREM} Spec \.{\implies} {\Box} PartialCorrectness\vs{.2}}%
\ascii{ep-ascii1}%
\@x{\@pfstepnum{1}{1.}\  Init \.{\implies} Inv\vs{.2}}%
 \@x{\@pfstepnum{1}{2.}\  Inv \.{\land} [ Next ]_{ vars} \.{\implies} Inv
 \.{'}\vs{.2}}%
\@x{\@pfstepnum{1}{3.}\  Inv \.{\implies} PartialCorrectness\vs{.2}}%
\@x{\@pfstepnum{1}{4.}\  {\QED}}%
\end{tlatex}
\end{display}
The proof of a theorem is a sequence of steps, the last step being a
  \ctindex{1}{QED@{\icmd{sc}qed}}{QED}%
\textsc{qed} step.  Each step may be followed by its proof.
Of course, the theorem hasn't been proved until all of these four steps
have been proved.  We can prove the steps in any order, or work on several
of them at once.  However, it's usually a good idea to prove the \textsc{qed}
step first.

\textsc{qed} means \emph{that which was to be proved}, which in this
proof is $Spec => []PartialCorrectness$.  The proof of the
\textsc{qed} step must show that the truth of this formula follows
from the truth of what is already known, which are the laws of
mathematics and steps $<<1>>1$--$<<1>>3$.  If we think that the proof
of a statement should be obvious, we can use the proof
   \ctindex{1}{BY@\icmd{textsc}{by}}{by}%
\textsc{obvious}.  So, write the word \texttt{OBVIOUS} after 
\texttt{QED}.  I like to write a step's proof indented, beginning
on the line after the step; so I would type this as
\begin{display}
\begin{verbatim}
<1>4. QED
  OBVIOUS
\end{verbatim}
\end{display}
Indentation and line breaks are ignored in a proof---except as they
affect the meaning of a list of conjuncts or disjuncts within an
individual expression.  However, you should begin each step on a new
line, otherwise some Toolbox commands may not behave properly.

Any proof can begin with the optional word \texttt{PROOF}, so we can
also write
\begin{display}
\begin{verbatim}
<1>4. QED
  PROOF OBVIOUS
\end{verbatim}
\end{display}
and 
\begin{display}
\begin{verbatim}
PROOF
<1>1. Init => Inv
  ...
\end{verbatim}
\end{display}
Have TLAPS check the proof of the \textsc{qed} step by
putting the \popref{cursor}{cursor} in the step or its proof and 
executing the 
\ctindex{1}{Prove Step or Module (Toolbox command)@\icmd{textsf}{Prove Step or Module} (Toolbox command)}{prove-step}%
\textsf{Prove Step or Module} command either by
right-clicking and choosing it on the menu, or by typing
\textsf{control+g control+g}.  

The proof fails, shading the step (approximately) red and raising
a window showing the obligation that the backend provers failed
to prove---which in this case is
\begin{display}
\begin{verbatim}
ASSUME NEW CONSTANT M,
       NEW CONSTANT N,
       NEW VARIABLE x,
       NEW VARIABLE y,
       NEW VARIABLE pc
PROVE  Spec  =>  []PartialCorrectness
\end{verbatim}
\end{display}
This obligation is exactly what the backend provers are trying to
prove.  They are trying to show that the goal, which is the formula in
the \verb|PROVE| section, follows from the list of assumptions in the
\verb|ASSUME| section together with the ordinary laws of mathematics.
That's all the backend provers know.  They know nothing about the
spec or about anything else in the proof.

Imagine printing out this obligation, giving it to some mathematician
you have never seen before, and expecting her to verify that the goal
follows from the assumptions.  Obviously, she couldn't.  How could she
possibly determine the truth of a formula containing the symbols
$Spec$ and $PartialCorrectness$ knowing only that $M$ and $N$ are
constants and $x$, $y$, and $pc$ are variables?

The truth of the goal follows from the truth of $<<1>>1$--$<<1>>3$.
So, we have to tell the provers to use those facts.  To do that,
we replace the proof \textsc{obvious} with%
  \ctindex{1}{BY@\icmd{textsc}{by}}{by}%
\begin{display}
\begin{verbatim}
BY <1>1, <1>2, <1>3
\end{verbatim}
\end{display}
The proof again fails because the backend provers can't prove
the obligation:
\begin{display}
\begin{verbatim}
ASSUME NEW CONSTANT M,
        ...
       NEW VARIABLE pc,
       Init  =>  Inv,
       Inv  /\  [Next]_vars  =>  Inv',
       Inv  =>  PartialCorrectness
PROVE  Spec  =>  []PartialCorrectness
\end{verbatim}
\end{display}
Can you see what the mathematician would say if you gave here this
obligation?  She'd say, ``How can I prove a formula containing the
symbol $Spec$ if none of the assumptions even mention that symbol?''
To prove the goal, we have to replace the symbol $Spec$ with its
definition.  We tell TLAPS to do that by adding a 
      \ctindex{1}{DEF@\icmd{textsc}{def}}{def}%
\textsc{def} clause
to the proof:%
\begin{display}
\begin{verbatim}
BY <1>1, <1>2, <1>3 DEF Spec
\end{verbatim}
\end{display}
The proof still fails, this time with the obligation:
\begin{display}
\begin{verbatim}
ASSUME NEW CONSTANT M,
        ...
       NEW VARIABLE pc,
       Init  =>  Inv,
       Inv  /\  [Next]_vars  =>  Inv',
       Inv  =>  PartialCorrectness
PROVE  Init  /\  [][Next]_vars  =>  []PartialCorrectness
\end{verbatim}
\end{display}
From everything you've read so far, it should be clear that the goal
follows from the three formulas in the \texttt{ASSUME} list.  Why did
the proof fail?  Suppose we asked the mathematician if the goal
follows from the assumptions?  If she were a randomly chosen
mathematician, she would probably know nothing about temporal logic or
\tlaplus, and she would say, ``I don't know if that's true because I
don't know what $[]$ or $[Next]_{vars}$ mean.'' Sometimes the backend
provers also try to tell you why they can't prove it in the window
showing the obligation, on the line that begins \emph{Obligation 1}.
But this time, they say nothing.  Most of the time, the only reason
reported by a backend that makes any sense to you or me is
\emph{timeout}, indicating that it timed out.  In that case, the
prover might be able to prove the obligation if given more time, but
it seldom can.

The default provers handle action reasoning---ordinary formulas that
may contain primed and unprimed variables.  They know what
$[Next]_{vars}$ means because that's an action formula, but not what
$[]$ means.  To prove our \textsc{qed} step, we use the 
$PTL$ 
  \ctindex{1}{PTL@\mmath{PTL} (backend prover)}{PTL}%
backend.
($PTL$ stands for Propositional Temporal Logic.)  To do this, we add
the ``fact'' $PTL$.  The symbol $PTL$ is defined in the $TLAPS$
library module.  The actual definition in that module is 
\tlabox{PTL \,\deq\, \TRUE}, but a bit of magic that has nothing to do with \tlaplus\ makes
putting $PTL$ in a \textsc{by} clause cause TLAPS to send the
obligation to $PTL$ rather than the default backends.
The proof
\begin{display}
\begin{verbatim}
BY <1>1, <1>2, <1>3, PTL DEF Spec
\end{verbatim}
\end{display}
works, and TLAPS colors the \textsc{qed} step green.

\medskip

Let's now prove the first step.  We will have to tell the TLAPS to expand
the definitions of $Init$ and $Inv$, and expanding the definition
of $Inv$ requires expanding the definition of $TypeOK$ as well.  So,
we can try the proof
\begin{display}
\texttt{BY DEF Init, Inv, TypeOK}%
  \marginpar{We can write \texttt{DEFS} instead of \texttt{DEF}.}

\end{display}
(The order of items in a \textsc{def} clause makes no difference.)
TLAPS fails, reporting this obligation.
\begin{display}
\begin{verbatim}
ASSUME NEW CONSTANT M,
        ...
       NEW VARIABLE pc
PROVE  (/\ x  =  M
        /\ y  =  N
        /\ pc  =  "Lbl_1")
        => 
       (/\ /\ x  \in  Nat  \  {0}
           /\ y  \in  Nat  \  {0}
        /\ GCD(x, y)  =  GCD(M, N)
        /\ pc  =  "Done"  =>  x  =  y)
\end{verbatim}
\end{display}
%try
If TLAPS fails to prove an obligation this simple, it's almost always
because you've forgotten to tell it to use some fact or definition
that it needs.  We haven't told it to expand the definition of $GCD$,
but this isn't necessary because $GCD(x,y)=GCD(M,N)$ follows from
$x=M$ and $y=N$.  Can you see what the problem is?  

If you can't see the problem quickly, the quickest way to find it is
usually to break the proof into steps.  The logical structure of what
we're trying to prove tells us how to do that.  We're trying to prove
that $Init$ implies a conjunction, and this is done by proving
separately that $Init$ implies each of the conjuncts.  The Toolbox's
      \ctindex{1}{Decompose Proof (Toolbox command)@\icmd{textsf}{Decompose Proof} (Toolbox command)}{decompose-command}%
\textsf{Decompose Proof (control+g control+r)} command allows you to
do that with a few mouse clicks.  Put the cursor anywhere in that step
or its proof and execute the command.  The Toolbox raises a window
like this:
\begin{display}
\pict(250,90){.75}{euclid1.png}
\end{display}
%\pictdim{.75}{euclid1.png}
The only thing you can do next is click on the button labeled \verb|=>|,
so do it.  The window changes to:
\begin{display}
\pict(250,113){.75}{euclid2.png}
\end{display}
%\pictdim{.75}{euclid2.png}
This tells you that the decomposition so far is to change what is
to be proved from $Init=>Inv$ to%
  \ctindex{1}{assume (proof construct)@\icmd{textsc}{assume} (proof construct)}{assume}%
  \ctindex{1}{prove@\icmd{textsc}{prove}}{prove}%
  \ctindex{1}{assume/prove}{assume-prove}%
%
\begin{display}
%  \ctindex{1}{assume (proof construct)@\icmd{textsc}{assume} (proof construct)}{assume}%
%  \ctindex{1}{prove@\icmd{textsc}{prove}}{prove}%
\begin{notla}
ASSUME Init
PROVE  Inv
\end{notla}
\begin{tlatex}
\@x{ {\ASSUME} Init}%
\@x{ {\PROVE}\@s{4.1} Inv}%
\end{tlatex}
\end{display}
By itself, this accomplishes nothing.  The implication and the
\textsc{assume/prove} are logically equivalent.  However, the button
labeled \verb|/\| next to $Inv$ tells you that $Inv$ is a conjunction
and you can use that to decompose the proof.  The \textsf{P} next
to the button means that clicking on it generates the decomposition.
Uncheck the \textsf{Use SUFFICES} and \textsf{Use CASE} options first
and then click on that button.  This replaces the proof with:
\begin{display}
\small
\begin{notla}
<2>1. ASSUME Init
      PROVE  TypeOK
  BY <2>1 DEF Init, Inv, TypeOK
<2>2. ASSUME Init
      PROVE  GCD(x, y) = GCD(M, N)
  BY <2>2 DEF Init, Inv, TypeOK
<2>3. ASSUME Init
      PROVE  (pc = "Done") => (x = y)
  BY <2>3 DEF Init, Inv, TypeOK
<2>4. QED
  BY <2>1, <2>2, <2>3 DEF Inv
\end{notla}
\begin{tlatex}
\@x{\@pfstepnum{2}{1.}\  {\ASSUME} Init}%
\@x{\@s{23.88} {\PROVE}\@s{4.1} TypeOK}%
\@x{\@s{8.2} {\BY}\@pfstepnum{2}{1}\  {\DEF} Init ,\, Inv ,\, TypeOK}%
\@x{\@pfstepnum{2}{2.}\  {\ASSUME} Init}%
\@x{\@s{23.88} {\PROVE}\@s{4.1} GCD ( x ,\, y ) \.{=} GCD ( M ,\, N )}%
\@x{\@s{8.2} {\BY}\@pfstepnum{2}{2}\  {\DEF} Init ,\, Inv ,\, TypeOK}%
\@x{\@pfstepnum{2}{3.}\  {\ASSUME} Init}%
 \@x{\@s{23.88} {\PROVE}\@s{4.1} ( pc \.{=}\@w{Done} ) \.{\implies} ( x \.{=}
 y )}%
\@x{\@s{8.2} {\BY}\@pfstepnum{2}{3}\  {\DEF} Init ,\, Inv ,\, TypeOK}%
\@x{\@pfstepnum{2}{4.}\  {\QED}}%
 \@x{\@s{8.2} {\BY}\@pfstepnum{2}{1} ,\,\@pfstepnum{2}{2}
 ,\,\@pfstepnum{2}{3}\  {\DEF} Inv}%
\end{tlatex}
\end{display}
Note that the original proof has been put into the proofs of steps
$<<2>>1$--$<<2>>3$.  Rather surprisingly, each of those steps seems to
be using itself in its own proof.  The $<<2>>1$ in the proof of step
$<<2>>1$ actually refers to the assumption $Init$ of the \textsc{assume}
clause.  Like any other fact, we must explicitly tell TLAPS when to
use the assumptions in an \textsc{assume} clause.  Within the proof of
an \textsc{assume}/\textsc{prove} step, the name of the step refers to
the \textsc{assume} clause's assumptions.

Save the module and run TLAPS on this proof by putting the cursor in
step $<<1>>1$ and executing \textsf{Prove Step or Module (control+g
control+g)}.  Everything succeeds except the proof of $TypeOK$, which
fails to prove the obligation:
\begin{display}
\begin{verbatim}
ASSUME NEW CONSTANT M,
       NEW CONSTANT N,
       ...
       /\ x  =  M
       /\ y  =  N
       /\ pc  =  "Lbl_1" (now)
PROVE  /\ x  \in  Nat  \  {0}
       /\ y  \in  Nat  \  {0}
\end{verbatim}
\end{display} 
The problem should now be clear.  This obligation is true only because
the module asserts the assumptions that $M$ and $N$ are in 
 $Nat :\: \{0\}$, 
but this assumption doesn't appear in the obligation.  We have to tell
TLAPS to use it.

In a \textsc{by} clause, instead of referring to a fact by its name,
you can give the fact itself.  For example, you can replace $<<2>>2$
in the proof of $<<2>>2$ with $Init$.  However, the fact has to 
be stated in the syntactically identical form as its assertion in
the module.  TLAPS proves step $<<2>>1$ with the proof
\begin{display}
\begin{notla}
BY <2>1,  /\ M \in Nat \ {0}
          /\ N \in Nat \ {0}   
 DEF Init, Inv, TypeOK
\end{notla}
\begin{tlatex}
 \@x{ {\BY}\@pfstepnum{2}{1} ,\,\@s{4.1} \.{\land} M \.{\in} Nat
 \.{\,\backslash\,} \{ 0 \}}%
\@x{\@s{43.69} \.{\land} N\@s{1.53} \.{\in} Nat \.{\,\backslash\,} \{ 0 \}}%
\@x{\@s{4.1} {\DEF} Init ,\, Inv ,\, TypeOK}%
\end{tlatex}
\end{display}
but not with the proof
\begin{display}
\begin{notla}
BY <2>1, M \in Nat \ {0}  /\  N \in Nat \ {0} DEF Init, Inv, TypeOK
\end{notla}
\begin{tlatex}
 \@x{ {\BY}\@pfstepnum{2}{1} ,\, M \.{\in} Nat \.{\,\backslash\,} \{ 0 \}
 \,\.{\land}\, N \.{\in} Nat \.{\,\backslash\,} \{ 0 \} {\DEF} Init ,\, Inv ,\,
 TypeOK}%
\end{tlatex}
\end{display}
%try
Rather than doing that, let's give the assumption a name.  But first,
there's no need to decompose the proof; the provers should have no
problem proving $<<1>>1$ with this additional fact.  So, let's undo
the decomposition of step $<<1>>1$'s proof with the Toolbox's
\textsf{Undo (control+z)} command.  Now let's give the assumption the
name $MNPosInt$ by changing it to:
\begin{display}
\begin{notla}
ASSUME MNPosInt == /\ M \in Nat \ {0}
                   /\ N \in Nat \ {0}
\end{notla}
\begin{tlatex}
 \@x{ {\ASSUME} MNPosInt \.{\defeq} \.{\land} M \.{\in} Nat \.{\,\backslash\,}
 \{ 0 \}}%
\@x{\@s{103.26} \.{\land} N\@s{1.53} \.{\in} Nat \.{\,\backslash\,} \{ 0 \}}%
\end{tlatex}
\end{display}
TLAPS will now verify this proof of step $<<1>>1$: \[ \BY MNPosInt
\DEF Init, Inv, TypeOK \] 
\medskip 
%
Let's now prove step $<<1>>2$, the step that proves that $Inv$ is an
inductive invariant.  This spec is simple enough that this step can be
proved with a \textsc{by} proof.  But in most real-life examples,
you're going to have to decompose this step, so let's start right away
by executing the \textsf{Decompose Proof} command on the step.  This
provides the only option of a \,\verb|=>|\, decomposition, so click on
it, producing this window.
\begin{display}
\pict(249,111){.75}{euclid3.png}
\end{display}
%\pictdim{.75}{euclid3.png}
There are two basic ways to decompose this proof.
\begin{itemize}
\item We can prove each conjunction of $Inv'$ separately.

\item The next-state action $Next$ is a disjunction, and we can prove
 \tlabox{A_{1} \/ \ldots \/ A_{n} => B}
by proving $A_{i}=>B$ for each~$i$.  
\end{itemize}
I usually find the second approach easier, though I'm not sure if it really
is or if it's just what I'm used to doing.  Anyway, let's do it that way.
Click on the top \,\verb|/\|\, button to turn the assumption
  \tlabox{Inv /\ [Next]_{vars}} into the part of it that can
be further decomposed:
\begin{display}
\pict(249,125){.75}{euclid4.png}
\end{display}
%\pictdim{.75}{euclid4.png}
Click on the \,\verb|\/|\, button to get
\begin{display}
\pict(249,158){.75}{euclid5.png}
\end{display}
%\pictdim{.75}{euclid5.png}
This tells us that the assumption $[Next]_{vars}$ equals
  \tlabox{Next \/ (\UNCHANGED vars)}
and we can split the proof of $Inv'$ into the two cases of assuming
$Next$ and of assuming \tlabox{\UNCHANGED vars}.  The \,\verb|\/|\, button next
to $Next$ tells us that formula $Next$ is also a disjunction on which
we can do a case split.  Click on it to produce:
\begin{display}
\pict(251,198){.75}{euclid6.png}
\end{display}
%\pictdim{.75}{euclid6.png}
Look at the definition of $Next$ in the algorithm's translation to see
that $Next$ is indeed the disjunction of the formulas in the two
subcases.  

Make sure that the \textsf{Use SUFFICES} and \textsf{Use CASE} options
are unchecked and click on the \,{\sf P}\, button, which generates
the proof.  The generated steps $<<2>>1$--$<<2>>3$ have no proofs, since
there was originally no proof for step $<<1>>2$.  We're going to need to
expand the definition of $Inv$ and $TypeOK$ in all those proofs.  So we
should have saved ourselves some typing by having a
\begin{display}
\textsc{by def} $Inv$, $TypeOK$ 
\end{display} 
proof on step $<<1>>1$.  Undo the decomposition, add that proof
to $<<1>>1$ and redo the decomposition.  Read the proof this produced
and make sure you understand why $<<1>>2$ follows from $<<2>>1$--$<<2>>3$
and the definition of $Next$, as the proof of the \textsc{qed} step
asserts.  Run the prover on this \textsc{qed} step to see that the
backend provers can verify this simple deduction.

Before we prove this, let's use the \textsf{Decompose Proof} command
to learn a little more about writing proofs.  Undo the decomposition
   and rerun the decomposition, except this time checking the \textsf{Use
SUFFICES} options.  This has changed the resulting proof by adding
the first step%
     \ctindex{1}{suffices@\icmd{textsc}{suffices}}{suffices}%
\begin{display}
\begin{notla}
<2> SUFFICES ASSUME Inv,
                    [Next]_vars
             PROVE  Inv'
  OBVIOUS
\end{notla}
\begin{tlatex}
\@x{\@pfstepnum{2}{}\  {\SUFFICES} {\ASSUME} Inv ,\,}%
\@x{\@s{98.64} [ Next ]_{ vars}}%
\@x{\@s{60.39} {\PROVE}\@s{4.1} Inv \.{'}}%
\@x{\@s{8.2} {\OBVIOUS}}%
\end{tlatex}
\end{display}
{\sloppy
and removing the assumptions $Inv$ and $ [Next]_{vars}$ from steps
$<<2>>1$--$<<2>>3$.  The \textsc{suffices} step asserts that to
prove the goal of $<<1>>2$, which is 
  \tlabox{Inv /\ [Next]_{vars} => Inv'},
it suffices to assume $Inv$ and $ [Next]_{vars}$ and prove $Inv'$.  It
also allows the use of the assumptions $Inv$ and $ [Next]_{vars}$ in
the remaining steps of this proof (the proof of $<<1>>2$).\par}

Note that the \textsc{suffices} step has a level number, but not a name.
Facts asserted by an unnamed step are used by the backend provers
without having to be mentioned in a \textsc{by}~step.  Run the prover
on step $<<2>>1$.  The proof fails because the definition of 
$Lbl\_1$ isn't being used, but examine the obligation.  Note that the
assumption following the declaration of $pc$---an assumption that is
ended by a comma---is $Inv$ with its definition and the definition of
$TypeOK$ expanded.  The next assumption is $[Next]_{vars}$.
Now give the \textsc{suffices} step a name---say $<<2>>x$.
  \marginpar{\popref{step-names}{About step names.}}%
Run the prover once again on step $<<2>>1$.  You'll see that the
two assumptions no longer appear in the obligation.  You can
add those assumptions to the obligation by adding $<<2>>x$ to
the \textsc{by} proof.

Unnecessary assumptions make it harder for the backend provers to
prove an obligation, so unnamed steps should be used with care.  An
unnecessary assumption like $Inv$ has little effect if the definition
of $Inv$ is not expanded.  Similarly, the assumption $[Next]_{vars}$
is pretty harmless if the definitions of $Next$ and $vars$ are not
expanded.  So not numbering this \textsc{suffices} step is not a
problem.

Now undo the decomposition of the proof of $<<1>>2$ and redo it, this
time with both the \textsf{Use SUFFICES} and \textsf{Use CASE} options
checked.  Steps $<<2>>1$--$<<2>>3$ have been changed from
\textsc{assume}/\textsc{prove} steps to \textsc{case} steps.  For
example, 
\begin{display}
\begin{notla}
<2>1. ASSUME Lbl_1
      PROVE  Inv'
\end{notla}
\begin{tlatex}
\@x{\@pfstepnum{2}{1.}\  {\ASSUME} Lbl\_1}%
\@x{\@s{23.88} {\PROVE}\@s{4.1} Inv \.{'}}%
\end{tlatex}
\end{display}
has been changed to
\begin{display}
\begin{notla}
<2>1.  CASE Lbl_1
\end{notla}
\begin{tlatex}
\@x{\@pfstepnum{2}{1.}\@s{4.1} {\CASE} Lbl\_1}%
\end{tlatex}
\end{display}
In general, the step $\CASE A$ is equivalent to the step
 \tlabox{\ASSUME A \ \PROVE G}, 
where $G$ is the current goal.  In this case, the current goal for
steps $<<2>>1$--$<<2>>3$ is set to $Inv'$ by the \textsc{suffices}
step.  

Undo the proof decomposition and run the \textsf{Decompose Proof}
command again, now with \textsf{Use CASE} checked and \textsf{Use
SUFFICES} unchecked.  This time, the proof has no \textsc{case} steps.
The \textsc{assume}/\textsc{prove} steps all have two assumptions, so
they can't be turned into \textsc{case} steps.  I think the version of
the proof with \textsc{suffices} and \textsc{case} is the nicest one.

% However, currently the \textsf{Decompose Proof} command will not
% decompose a \textsc{case} step or a \textsc{suffices} step.  For a
% complicated proof in which we might want to decompose the steps
% further, you might not want to use these two options.

Using whichever decomposition you prefer, let's now check the proof.
We obviously need to add $Lbl\_1$ to the \textsc{def} clause in the
proof of $<<2>>1$ and $vars$ to the \textsc{def} clauses in the proofs
of $<<2>>2$ and $<<2>>3$.  Do that and run the prover on the proof.
You can either run the \textsf{Prove Step or Module} command on step
$<<1>>2$ (by putting the cursor in that step) or on the entire theorem
(by putting the cursor on the statement of the theorem)
or on the entire module (by putting the cursor outside any theorem or
proof).  Because of its use of fingerprints, TLAPS doesn't spend any
significant amount of time rechecking proofs it has already checked.
(It does take time to compute the obligations, so you don't want to
run TLAPS unnecessarily on more than a few dozen steps.)  It will
not check any steps that have no proof, coloring those steps yellow.

TLAPS proves everything except step $<<2>>1$, which is the only
non-trivial one.  Examining the obligation, we see two problems.
First, we've forgotten our assumption $MNPosInt$ telling us that $M$
and $N$ are positive integers.  Second, the obligation can't be proved
without knowing something about $GCD$.  We can try adding $MNPosInt$
to the \textsc{by} clause and expanding the definition of $GCD$, which
requires also expanding the definitions of $SetMax$, $DivisorsOf$, and
$Divides$.  Do that and try the proof again.  It fails.  The
obligation is complicated enough that we'd have a hard time trying to
understand it.  However, the backend provers can handle obligations
that large---especially the SMT solvers.  The SMT solvers are
especially good at dealing with arithmetic formulas that involves
inequalities, addition, and subtraction.  However, this obligation
also involves multiplication.  (The other backends are pretty bad at
any kind of arithmetic.)  Furthermore, the obligation contains a
number of \textsc{choose} expressions and existential quantifiers,
both of which can trip up the provers.

We could further decompose the proof of $<<2>>1$ by trying to prove
each of the conjuncts of $Inv'$ separately.  We would find that the
conjunct whose proof fails is \tlabox{(GCD(x, y) = GCD(M, N))'}, which
is equivalent to \tlabox{(GCD(x', y') = GCD(M, N))}.  We have reached
the point where we can't get any further by blindly decomposing the
proof.  We have to think about why $GCD(x',y')$ equals $GCD(M,N)$.
Going back to the informal proof, we remember that correctness of
Euclid's algorithm rests on properties $GCD1$---$GCD3$, which are
defined in the module $GCD$.  In fact, the invariance of $Inv$ is
proved using $GCD2$ and $GCD3$.  The proof of $<<2>>1$ succeeds
if we add $GCD2$ and $GCD3$ to the \textsc{by} proof.  In fact,
there's no need to decompose step $<<1>>2$.
TLAPS verifies it with the proof
 \[ \BY MNPosInt, \ GCD2, \ GCD3 \DEF Inv, \ TypeOK, \ Next, \ Lbl\_1, \ vars\]
You should be able to prove step $<<1>>3$, using 
$GCD1$.

\subsection{Proving Properties of the GCD}

Checking the correctness of algorithms and systems is hard enough.
You should not have to check the correctness of mathematical theorems.
That should be the job of mathematicians.  So, normally you would not
bother to get TLAPS to check the correctness of theorems
$GCD1$--$GCD3$ of the $GCD$ module.  If a mathematical result is not
very simple, then you should be able to find it in a math book.  You
just have to translate the result from the informal math used by
mathematicians to a \tlaplus\ theorem.  To avoid errors in formalizing
the result, you should check the theorem with TLC.

However, our topic here is how to write proofs, and the proofs of
$GCD1$--$GCD3$ are instructive.  So, let's write them.  Open module
$GCD$ in the Toolbox.  You don't need to make it a separate
specification; you can open it within the $Euclid$ spec with the
\textsf{Open Module} command in the Toolbox's \textsf{File} menu.
(Another way to go to the $GCD$ module is to
apply  the 
  \ctindex{1}{Goto Declaration (Toolbox command)@\icmd{textsf}{Goto Declaration} (Toolbox command)}{goto-declaration}
\textsf{Goto Declaration (F3)} command to the identifier
$GCD$, which will go to its definition within that module.
%
The
\textsf{Return from Goto Declaration (F4)}%
    \ctindex{2}{Return from Goto Declaration (Toolbox command)@\icmd{textsf}{Return from Goto Declaration} (Toolbox command)}{return-from-goto-declaration}
command returns to the original cursor position.)

Let's start trying to prove each of the theorems $GCD1$--$GCD3$ with a
\textsc{by} proof.  We expand the definition of $GCD$ down to its
\tlaplus\ primitive operators, which is done by the following proof:
\begin{display}
\textsc{by def} $GCD$, $SetMax$, $DivisorsOf$, $Divides$
\end{display}
Add that proof after each of the three theorems and run the prover on
all of them (by executing the \textsf{Prove Step or Module} command
with the cursor outside any theorem or proof).  This succeeds in
proving only $GCD2$.  In fact, $GCD2$ follows from the top-level
definition of $GCD$ and commutativity of set union; there is no need
to expand the definitions of $SetMax$ or $DivisorsOf$.  Remove the
unnecessary definition expansions from the proof of $GCD2$ and check
that TLAPS still proves it.

\medskip 

Let's prove $GCD1$ next.  To decompose the proof, let's use the
\textsf{Decompose Proof} command.  It raises this
window
\begin{display}
\pict(247,89){.75}{euclid7.png}
\end{display}
%\pictdim{.75}{euclid7.png}
which gives us only one option: clicking on the \,\verb|\A|\, button.
This changes the dialog to
\begin{display}
\pict(248,102){.75}{euclid8.png}
\end{display}
%\pictdim{.75}{euclid8.png}
%
which tells us that we can prove the goal by assuming that $m$ is
an element of \tlabox{Nat :\: \{0\}} and proving \tlabox{GCD(m,m)=m}.
Check the \textsf{Use SUFFICES} option and click on the \textsf{P}
button to produce this proof.
\begin{display}
\begin{notla}
<1> SUFFICES ASSUME NEW m \in Nat \ {0}
             PROVE  GCD(m, m) = m
  OBVIOUS
<1> QED
    BY DEF GCD, SetMax, DivisorsOf, Divides
\end{notla}
\begin{tlatex}
 \@x{\@pfstepnum{1}{}\  {\SUFFICES} {\ASSUME} {\NEW} m \.{\in} Nat
 \.{\,\backslash\,} \{ 0 \}}%
\@x{\@s{60.39} {\PROVE}\@s{4.1} GCD ( m ,\, m ) \.{=} m}%
\@x{\@s{8.2} {\OBVIOUS}}%
\@x{\@pfstepnum{1}{}\  {\QED}}%
\@x{\@s{16.11} {\BY} {\DEF} GCD ,\, SetMax ,\, DivisorsOf ,\, Divides}%
\end{tlatex}
\end{display}
Run the \textsf{Prove Step or Module} command on the theorem.  It
proves the \textsc{suffices} step, but not the \textsc{qed} step.
Proving the \textsc{suffices} step means proving that assuming its
assumptions and proving its \textsc{prove} formula proves the current
goal, which is the statement of the theorem.  This is trivial because
the \textsc{assume}/\textsc{prove} is equivalent to the theorem.
Naturally, transforming the theorem's goal to something obviously
equivalent isn't going to enable the backend provers to prove it, so
the proof of the \textsc{qed} step fails.  What the \textsc{suffices}
step accomplishes is to remove the quantifier, allowing us to prove 
  $GCD(m, n) = m$
for a particular choice of $m$.

The \textsf{Decompose Proof} command can take us no further; we have
to start thinking.  To show that $m$ is the GCD of $m$ and $m$, we
have to show that (1)~$m$ divides $m$ and (2)~$m$ is the largest
number that divides $m$.  This suggests that we add the following two
steps after the \textsc{suffices} step
\begin{display}
\begin{notla}
<1>1. Divides(m, m)
<1>2. \A i \in Nat : Divides(i, m) => (i =< m)
\end{notla}
\begin{tlatex}
\@x{\@pfstepnum{1}{1.}\  Divides ( m ,\, m )\vs{.2}}%
 \@x{\@pfstepnum{1}{2.}\  \A\, i \.{\in} Nat \.{:} Divides ( i ,\, m )
 \.{\implies} ( i \.{\leq} m )}%
\end{tlatex}
\end{display}
Before trying to prove them, it's best to check that they are all we
need.  So, add the facts 
  \marginpar{\popref{why-number-steps}{Why number these steps?}}
$<<1>>1$ and $<<1>>2$ to the \textsc{by}
clause of the \textsc{qed} step's proof and see if TLAPS will prove
that step.

TLAPS does prove the \textsc{qed} step.  If it hadn't, we would have
had to do some more thinking to see why not.  Our thinking about 
the proof should have led us to realize that proving the theorem from
$<<1>>1$ and $<<1>>2$ doesn't require expanding the definition of
$Divides$.  Expanding the definition made the formula that had to be
proved more complicated than necessary.  An unnecessary definition
expansion can make the difference between a proof succeeding and
failing.  So, remove $Divides$ from the \textsc{def} clause and check
that TLAPS still proves the \textsc{qed} step.

Now we have to prove steps $<<1>>1$ and $<<1>>2$.  We have no useful
facts at our disposal; all we can do is tell the TLAPS to expand the
definition of $Divides$.  So, add the proof
 \[\BY \DEF Divides \]
to both of them and try proving them both by telling TLAPS to prove
the entire theorem.  TLAPS proves $<<1>>1$ and $<<1>>2$, thus proving
the theorem.

\medskip

Finally, we must prove theorem $GCD3$.  Run the \textsf{Decompose Proof}
command, clicking on the \,\verb|\A|\, button, then the \,\verb|=>| button,
and then the \,\textsf{P}\, button to produce this proof.
\begin{display}
\begin{notla}
<1> SUFFICES ASSUME NEW m \in Nat \ {0}, NEW n \in Nat \ {0},
                    n > m
             PROVE  GCD(m, n) = GCD(m, n-m)
  OBVIOUS
<1> QED
    BY DEF GCD , DivisorsOf, SetMax, Divides
\end{notla}
\begin{tlatex}
 \@x{\@pfstepnum{1}{}\  {\SUFFICES} {\ASSUME} {\NEW} m \.{\in} Nat
 \.{\,\backslash\,} \{ 0 \} ,\, {\NEW} n \.{\in} Nat \.{\,\backslash\,} \{ 0
 \} ,\,}%
\@x{\@s{98.64} n \.{>} m}%
 \@x{\@s{60.39} {\PROVE}\@s{4.1} GCD ( m ,\, n ) \.{=} GCD ( m ,\, n \.{-} m
 )}%
\@x{\@s{8.2} {\OBVIOUS}}%
\@x{\@pfstepnum{1}{}\  {\QED}}%
\@x{\@s{16.11} {\BY} {\DEF} GCD ,\, DivisorsOf ,\, SetMax ,\, Divides}%
\end{tlatex}
\end{display}
We should have TLAPS check the proof of the \textsc{suffices} step to
make sure that the \textsf{Decompose Proof} command didn't make an
error.  There's no point wasting time having TLAPS check the
\textsc{qed} step because the decomposition so far has been logically
trivial.  We have to do some thinking to decompose the proof into
simpler steps.

That result of that thinking appears in \popref{gcd-proof-1}{our
informal proof of $GCD3$}.  The first step of that proof states that,
to prove the goal
  $GCD(m, n) = GCD(m, n - m)$
it suffices to prove
\begin{display}
\begin{notla}
\A i \in Int : Divides(i, m) /\ Divides (i, n) 
                 <=>  Divides(i, m) /\ Divides(i, n-m)
\end{notla}
\begin{tlatex}
 \@x{ \A\, i \.{\in} Int \.{:} Divides ( i ,\, m ) \.{\land} Divides ( i ,\, n
 )}%
 \@x{\@s{54.92} \.{\equiv}\@s{5.37} Divides ( i ,\, m ) \.{\land} Divides ( i
 ,\, n \.{-} m )}%
\end{tlatex}
\end{display}
%
Let's see if TLAPS believes that is enough to prove our goal.  Insert
that as an unnumbered step (labeled by $<<1>>$) before the
\textsc{qed} step and now run the TLAPS on the \textsc{qed} step.
It succeeds.  (In fact, it's not necessary to expand the definition
of $Divides$.)  

All we have left to do is to prove the newly added step.  You will find
that TLAPS successfully checks the proof 
 \[\BY \!\!\DEF Divides\]
Run the \textsf{Prove Step or Module} command on the entire module to
verify that we have indeed proved all three theorems.

%try
\newpage 
\section{The Proof Language}

Now that you've seen the basics of using TLAPS, it's time to examine
proofs and the proof language more closely.  Before we do that,
let's examine the theorems that we are trying to prove.

\subsection{What a Theorem Asserts}

A \emph{theorem} consists of one of the equivalent keywords
\begin{display}
\textsc{theorem}\s{2} \textsc{lemma}\s{2} \textsc{corollary}\s{2}
\textsc{proposition}%
 \ctindex{1}{theorem@\icmd{textsc}{theorem}}{theorem}%
 \ctindex{1}{lemma@\icmd{textsc}{lemma}}{lemma}%
 \ctindex{1}{corollary@\icmd{textsc}{corollary}}{corollary}%
 \ctindex{1}{proposition@\icmd{textsc}{proposition}}{proposition}%
\end{display}
followed optionally by an identifier and the symbol $\deq$\,,
followed
by an assertion.  A theorem of the form
 \[ \THEOREM id == A
 \]
defines $id$ to equal assertion $A$.  Naming theorems (and
assumptions) is a good idea, because it makes them easier to use in a
proof.

An 
  \tindex{1}{assertion}%
\emph{assertion} is either a formula or an 
  \ctindex{2}{assume/prove}{assume-prove}%
assume/prove.  An assume/prove has the form
 \[ \begin{noj2}
    \ASSUME & A_{1}\,, \ldots, A_{n} \\
    \PROVE & B
    \end{noj2}
 \]
where $B$ is a formula and each of the assumptions $A_{i}$ is either a
formula or a declaration.  A declaration is something like 
 \ctindex{1}{new@\icmd{textsc}{new}}{new}%
$\NEW v$ 
 \marginpar{\tlaplus\ allows other kinds of
      declarations, but you'll never write them.} 
or $\NEW v \in S$ where $v$ is a variable and $S$ is an expression.
The keyword \textsc{new} can be replaced by \textsc{constant} or
\textsc{new constant}.  The declaration $\NEW v \in S$ is almost equivalent
to the declaration
  \tlabox{\NEW v} followed by the
assumption $v \in S$.  The only difference is that when it appears
in $\NEW v \in S$, the formula $v \in S$ is called a
 \ctindex{1}{domain assumption}{domain-assumption}% 
 \ctindex{1}{assumption, domain}{assumption-domain}% 
\emph{domain assumption}.  However, the formula $v \in S$ is not
a domain assumption when it appears by itself in a separate
\textsc{assume} clause.  The significance of domain assumptions
is explained below.  It's usually best to write $\NEW v \in S$
instead of splitting the assumption in two.

% To simplify the exposition, we assume that a declaration $\NEW v \in
% S$ is replaced by its declaration $\NEW v$ and its domain assumption
% $v \in S$.  However, we distinguish the domain assumptions obtained
% from such declarations from all other assumptions because they are
% handled a bit differently.  In other words, even though $\NEW v \in S$
% has the same meaning as \tlabox{\NEW v}, $v \in S$, the two are
% handled differently by the prover.

To simplify the exposition, we consider an assertion that is a
formula $F$ to be an assume/prove with no assumptions, as if it were
written ${\red\ASSUME \PROVE} F$.  (\tlaplus\ does not allow you to write
such an assume/prove, so I color it red.)

An assertion asserts (the truth of) a formula.  
The assertion 
 \[  \ASSUME \PROVE \ F
 \]
asserts the formula $F$.  The assertion
 \[ \begin{noj2}
    \ASSUME & \NEW x \in S, \ P(x) \\
    \PROVE  & Q \/ R(x)
    \end{noj2}
 \]
asserts the formula \,\tlabox{\A\, x \in S : P(x) => (Q \/ R(x))}\,.
The assertion
 \[ \begin{noj2}
    \ASSUME & \NEW P(\_), \ \NEW x,\ \NEW y, \, x=y \\
    \PROVE  & P(x) = P(y)
    \end{noj2}
 \]
asserts the formula
  \[ {\red\A\, P(\_) :}\, \A\, x : \A\, y : (x=y) => (P(x)=P(y))
 \]
This isn't a legal \tlaplus\ formula, since \tlaplus\ doesn't allow
quantifying over an operator that takes an argument.  I will
use such formulas for the purpose of explaining proofs, coloring the illegal
parts red.




\subsection{The Hierarchical Structure 
of a Proof}
    \ctindex{2}{proof!structured}{proofs-structured}%
\vspace{-2\baselineskip}

\subsubsection{Writing 
Structured Proofs}
A theorem may have a proof.  A proof consists of the optional keyword
\textsc{proof} followed by either:
\begin{itemize}
\item A 
     \ctindex{1}{proof!non-leaf}{proof-nonleaf}%
     \ctindex{1}{non-leaf proof}{nonleaf-proof}%
\emph{non-leaf proof} that is a sequence of steps, ending with a
\textsc{qed} step, each of which may (but need not) have a proof.

\item A 
       \ctindex{1}{proof!nonleaf}{proof-leaf}%
      \ctindex{1}{leaf proof}{leaf-proof}%
\emph{leaf proof}, which is either the keyword \textsc{obvious}, the
keyword 
 \ctindex{1}{omitted@\icmd{textsc}{omitted}}{omitted}%
\textsc{omitted}, or a 
 \ctindex{1}{by proof@\icmd{textsc}{by} proof}{by-proof}%
 \ctindex{1}{proof!by@\icmd{textsc}{by}}{proof-by}%
\textsc{by} proof.
\end{itemize}
The leaf proof \textsc{omitted} is equivalent to having no proof; use
it to indicate that you are deliberately assuming something and have
not simply forgotten to prove it.

A step (of a non-leaf proof) begins with a preface token consisting 
of the following three parts, with no spaces between them:
\begin{itemize}
\item A 
  \ctindex{1}{level specifier}{level-specifier}%
\emph{level specifier} of the form $<<i>>$, where $i$ is a
non-negative integer called the \emph{level number}.  (It is typed
\verb|<|\,$i$\,\verb|>|.)  All steps in a single non-leaf proof must
have the same level number.  If the step has a non-leaf proof, 
the steps of that proof must have a level number greater than $i$.


\item A string of digits, letters, and/or \verb|_| characters 
that may be empty.  If it is non-empty, the step is said to be
\emph{named}, and its \emph{name} is the level specifier followed
by this string.

\item An optional period (\verb|.|).
\end{itemize}
%
For example, \,\,\tlabox{<<3>>2a.\ 1+1=3}\,\, is a named level-3 step
with name $<<3>>2a$ and assertion 
  ${\red\ASSUME \PROVE} 1+1=3$.  If, like me, you prefer to
name most of the steps in a non-leaf proof $<<i>>1$, $<<i>>2$, \ldots\,, see
the help page of the Toolbox's 
  \ctindex{2}{Renumber Proof (Toolbox command)@\icmd{textsf}{Renumber Proof} (Toolbox command)}{renumber-proof}%
\textsf{Renumber Proof} command.

We describe the hierarchical structure of a proof in the usual way as
an upside-down tree (with the root on top), where steps at a lower
level (deeper) in the proof structure (the tree) have larger level
numbers.

\subsubsection{Reading 
Structured Proofs}

  \ctindex{1}{proof!reading}{proof-reading}%
You may have noticed a little $\ominus$ next to theorems and
proof steps.  (For brevity, I will write \emph{step} to mean
either a proof step or the statement of a theorem.)  Clicking on the
$\ominus$ hides the step's proof, replacing the $\ominus$ with
$\oplus$.  Clicking on the $\oplus$ undoes the effect of clicking on
the $\ominus$.

There are a number of commands for viewing the proof as hypertext that
provide finer control of what is shown than you can get by just
clicking on $\ominus$ and $\oplus$.  The following commands
are executed on a step by putting the cursor on the step
and either right-clicking and selecting the command or typing
the indicated keystrokes.
\begin{description}

\item[\sf Show Current Subtree]   
  \marginpar{Clicking on $\ominus$ or $\oplus$ sometimes doesn't work 
      properly.  These other commands should always do what they're
      supposed to.}%
    \ctindex{1}{Show Current Subtree(Toolbox command)@\icmd{textsf}{Show Current Subtree} (Toolbox command)}{show-current-subtree}% 
(\textsf{control+g control+s}) \\
 Reveals the complete proof of the step.
 

\item[\sf Hide Current Subtree]  
    \ctindex{1}{Hide Current Subtree(Toolbox command)@\icmd{textsf}{Hide Current Subtree} (Toolbox command)}{hide-current-subtree}% 
 (\textsf{control+g control+h}) \\
 Hides the proof of the step.
 

\item[\sf Show Children Only]  
    \ctindex{1}{Show Children Only(Toolbox command)@\icmd{textsf}{Show Children Only} (Toolbox command)}{show-children-only}% 
 (\textsf{control+g control+c}) \\
 Reveals the top level of the step's proof.
 

\item[\sf Focus on Step]   
    \ctindex{1}{Focus On Step(Toolbox command)@\icmd{textsf}{Focus On Step} (Toolbox command)}{focus-on-step}% 
(\textsf{control+g control+f}) \\
Hides everything except the top level of the step's proof and the
siblings of (steps at the same level as) the step and of every
ancestor of that step in the proof.  

This is useful when writing the
proof because, after executing this command on a step, the steps
before it that are shown are precisely the ones that can be referred
to in the proof of the step.

\end{description}
The following two commands are performed with the cursor anywhere
in the module.
\begin{description}
\item[\sf Show All Proofs]   
    \ctindex{1}{Show All Proofs(Toolbox command)@\icmd{textsf}{Show All Proofs} (Toolbox command)}{show-all-proofs}% 
(\textsf{control+g control+a}) \\
 Reveals the complete proof of every theorem in the module.
 

\item[\sf Hide All Proofs]   
    \ctindex{1}{Hide All Proofs(Toolbox command)@\icmd{textsf}{Hide All Proofs} (Toolbox command)}{hide-all-proofs}% 
(\textsf{control+g control+n}) \\
 Hides the proof of every theorem in the module.
%
(The \textsf{Focus on Step} command performed outside a proof
is equivalent to \textsf{Hide All Proofs}.)
\end{description}
%
As you have undoubtedly noticed, the Toolbox editor commands having to
do with proofs are executed from the keyboard by typing \textsf{control+g}
plus another control character.  If you just type \textsf{control+g} and
wait a second, you will see a list of all the commands you can execute
with an additional keystroke.

\subsection{The State of a Proof}
  \ctindex{1}{state!of a proof}{state-of-proof}%
  \ctindex{1}{proof!state}{proof-state}%
\vspace{-\baselineskip}


\noindent At each step in a proof, and at each leaf proof, there is a
state that consists of the following components:

\begin{describe}{\FK}
\item[\G] A formula that is the 
  \ctindex{1}{goal, of proof}{goal-proof}%
  \ctindex{1}{proof!goal}{proof-goal}%
current goal of the proof.


\item[\Sym] The sequence of current symbol declarations.  Here are
  examples of symbol declarations:
  \begin{display}
  $ \begin{noj}
    \VARIABLE x \s{2} \CONSTANT C \s{2} \CONSTANT Op(\_,\,\_,\, \_) \V{.4}
         \CONSTANT \_\oplus\_ 
    \end{noj}$ 
  \end{display}
 There is one additional kind of symbol declaration:
  $\CONSTANT id \in S$, where $id$ is an identifier and $S$ is an 
  arbitrary expression.  We call 
  the formula $id \in S$ the declaration's \emph{domain assumption}.

\item[\FK] The known 
  \ctindex{1}{facts!known}{facts-known}%
facts, which is the set of formulas currently
asserted by the user to be true, and which can be used to prove new
facts.

\item[\FU] The 
  \ctindex{1}{facts!usable}{facts-usable}%
usable facts, which is the subset of \FK\
consisting of those facts that are used by default in proofs.

\item[\DK] The set of all user-defined symbols.

\item[\DU] The subset of \DK\ containing all user-defined symbols
whose definitions are by default expanded in proofs.

\item[\BB] A sequence of \popref{backends}{backend provers}.
\end{describe}
The proof state determines the proof obligations that are sent to the
backend provers, and what backend provers they are sent to, as
described below in
\sref{proof}{\xlink{pf-obligations}}{Section~\xref{pf-obligations}}.


The proof-state components other than \G\ are also defined at all
high-level statements in a module.  They are all empty at the
beginning of the module, except that \BB\ equals the default sequence
$<<SMT,\,Zenon,\,Isa>>$ of backend provers.
These components are changed by ordinary module
statements in the following ways:
\begin{description}
\item 
     $\CONSTANTS C,\ F(\_)$ \mbox{}\\
Appends the sequence $<<\CONSTANT C,\ \CONSTANT F(\_)>>$ of declarations 
to \Sym.  In other words, if
\Sym\ equals $<<\VARIABLE x,\ \CONSTANT AB>>$ before the declaration, then it 
equals
 \[ <<\VARIABLE x,\ \CONSTANT AB,\ \CONSTANT C,\ \CONSTANT F(\_)>>\]
immediately after the declaration.  The
\textsc{constants} statement leaves the other components of the proof state
unchanged.  A \textsc{variables} statement has a similar effect.

\item[$ f(a) \ \deq  \ \{x, a\}$] \mbox{}\\
adds the symbol $f$ to \DK, but leaves the other components, including
\DU, unchanged.  Thus, a module's definitions are not usable by
default.  A function definition has essentially the same effect.


\item [\rm $\textsc{theorem}\ thm \,\deq\, \begin{noj2}
                               \textsc{assume} & \NEW i \in S, \ P(i)\\
                               \textsc{prove}  & Q(i)
                               \end{noj2}$] \mbox{}\\
Adds the formula \tlabox{\A\, i \in S : P(i)=> Q(i)} to \FK, 
adds $thm$ to both \DK\ and \DU, and leaves \FU, \Sym, and \BB\ unchanged.
Thus, the definition of the theorem name, but not the formula
asserted by the theorem, is usable by default.  
The statement has the same effect without the
``\,$thm$~$\deq$\,'' except that \DK\ and \DU\ are left
unchanged.  An \textsc{assume} statement (which can also include a
definition) has the same effect.

\item[\rm \textsc{extends} M1, M2] \mbox{}\\
Has the same effect as if the statements of modules $M1$ and $M2$ were
inserted at the beginning of the current module.  

\item[\rm An \textsc{instance} statement] \mbox{}\\ 
%
The definitions imported by the statement are added to \DK, and the
instantiated theorems from the module are added to \FK\ as described
\sref{proof}{instantiated-theorems}{below}.  The components \Sym, \DU,
\FU, and \BB\ are unchanged.

\item[\rm \textsc{recursive} $op(\,\_\,)$] \mbox{}\\
%
Has no effect on the proof state. 
\end{description}
%
To specify the proof state at each step and each leaf proof of a
theorem's proof, we do two things:
\begin{itemize}
\item For a proof step $\Sigma$ that is not a \textsc{qed} step, we specify
the proof state at the next proof step at the same level as $\Sigma$.
(A \textsc{qed} step ends its level of the proof.)

\item For a proof step or theorem that has a proof, we specify
the proof state at the beginning of its proof---which is either
its leaf proof or the first step of its non-leaf proof .
\end{itemize}
How every different kind of proof step affects the proof state is
described below.  Some of the descriptions are given for particular examples
of the steps; the generalizations to arbitrary instances of the steps
should be obvious.

Remember that a step that has a preface token like $<<3>>14.$ is said
to be named, and $<<3>>14$ is its name.  A step with a prefix token
like $<<3>>$ is unnamed.  Similarly, a theorem that begins
\tlabox{\THEOREM thm\,\deq\,} is said to be named and have the name
$thm$.  Other theorems are said to be unnamed.  In the following lists
of proof-step statements, the preface tokens are omitted.

\subsubsection{Steps That Can Have a Proof}
\begin{description}
\item[\rm $\begin{noj2} 
              \textsc{assume} & \NEW i \in S, \ j \in T\\
              \textsc{prove}  & Q(i)
           \end{noj2}$] \mbox{}\\
This can be either a step or the statement of a \textsc{theorem}.  If it
is a step, then the proof context of the next step at the same
level is obtained from the context at the step as follows:
  \begin{itemize}
    \item The formula  \,\tlabox{\A\, i \in S: (j \in T) => Q(i)}\, asserted by
          the assume/prove is added to \FK.  If the step is unnamed, 
          this formula is also added to \FU; otherwise \FU\ is unchanged.
    \item If the step is named, then its name is added to \DK\ and
          \DU; otherwise \DK\ and \DU\ are unchanged.
    \item \G, \Sym, and \BB\ are left unchanged.       
  \end{itemize}
If the step or theorem has a proof, the proof context at the beginning
of the proof is obtained from the context at the step or theorem as
follows:
  \begin{itemize}
    \item The current goal \G\ becomes $Q(i)$.

    \item The declaration $\CONSTANT i \in S$ is appended to \Sym.

    \item The formula $j\in T$ is added to \FK.
          If this is a theorem or an unnamed step,
          then this formula is 
          also added to \FU; otherwise \FU\ is unchanged.  
    \item If the step or theorem is named, then its name is added to
          \DK\ and \DU; otherwise \DK\ and \DU\ are unchanged.
    \item \BB\ is unchanged.
  \end{itemize}
If the step is named $<<3>>14$, then:
  \begin{itemize}
  \item Within the step's proof, $<<3>>14$ names the formula
        $j\in T$.  The name can be used only as
        a fact---for example, in a \textsc{by} proof.  
        A formula that contains the name, such as $(j>i)=> <<3>>14$, 
        is illegal.
        
  \item Starting from the next step at the current level until the
        end of the current-level proof, $<<3>>14$ names the
        formula \tlabox{\A\, i \in S: (j \in T) => Q(i)}\, asserted by
        the assume/prove.  
  \end{itemize}
Observe that facts that would be added to the set \FU\ of usable facts
if the step were unnamed are not added if those facts can be named.
The philosophy behind this is that the user should state explicitly
(usually by name) what facts are needed to prove each step.  This
makes the proof easier for humans to understand and for backend
provers to check.

Unlike unnamed steps, unnamed theorems are not usable by default.
This means that adding a new theorem, whether named or not, will not
affect the proofs of later theorems.  (Adding a new usable fact can't
invalidate a proof, but it can make it harder for a prover to check
it, causing TLAPS to fail to check the proof.)  Since unnamed theorems
can't be referred to by name, using them in a proof is inconvenient.

\item[\rm\ \QED] \mbox{}\\ 
%
The context at the start of a \textsc{qed} step's proof is the same as
for the step that simply asserted \G, the step's current goal.  No
step follows a \textsc{qed} step at the same level.

\item[\rm \textsc{suffices} $\begin{noj2} 
              \textsc{assume} & \NEW i \in S, \ j \in T\\
              \textsc{prove}  & Q(i)
           \end{noj2}$] \mbox{}\\
The proof context of the next step at the same
level is obtained from the context at the step as follows:
  \begin{itemize}
    \item The current goal \G\ becomes $Q(i)$.

    \item The declaration $\CONSTANT i\in S$ is appended to \Sym.

    \item The formula $j\in T$ is added to \FK.
          If this is an unnamed step,
          then the formula is
          also added to \FU; otherwise \FU\ is unchanged.
    \item If the step is named, then its name is added to
          \DK\ and \DU; otherwise \DK\ and \DU\ are unchanged.
    \item \BB\ is unchanged.
  \end{itemize}
If the step has a proof, the proof context at the beginning
of the proof is obtained from the context at the step as
follows:
  \begin{itemize}
    \item The current goal \G\ is unchanged.

    \item The formula \,\tlabox{\A\, i \in S: (j \in T) => Q(i)}\, asserted by
          the assume/prove is added to \FK\ and \FU.
    \item If the step is named, then its name is added to \DK\ and
          \DU; otherwise \DK\ and \DU\ are unchanged.
    \item \G, \Sym, and \BB\ are left unchanged.       
  \end{itemize}
If the step is named $<<3>>14$, then:
  \begin{itemize}
  \item Starting from the next step at the same level until the
        end of the current-level proof, $<<3>>14$ names the formula
        $j\in T$.  
%        The name can be used only as
%        a fact---for example, in a \textsc{by} proof---not in
%        a formula such as $(j>i)=> <<3>>14$.
        
  \item Within the step's proof, $<<3>>14$ names 
      \tlabox{\A\, i \in S: (j \in T) => Q(i)}\,, the formula asserted 
       by the assume/prove.
  \end{itemize}
Observe that there is a duality between an assume/prove step and a
\textsc{suffices} assume/prove step.  The proof state at the beginning
of the proof of an assume/prove step is the state after the
\textsc{suffices} assume/prove step and its proof, and vice-versa.
This reflects the fact that by renumbering steps, we can convert a proof 
\begin{display}
$<<3>>14.\ \begin{noj2} 
              \textsc{assume} & \NEW i \in S, \ j \in T\\
              \textsc{prove}  & Q(i)
           \end{noj2}$ \\
\s{2}\fbox{Level-4 proof of $Q(i)$ using assumptions $i\in S$
       and $j \in T$.}\V{.5}
\fbox{Rest of level-3 proof that proves \G\ using 
        \tlabox{\A\, i \in S: (j \in T) \Rightarrow Q(i).}}\s{-5}
\end{display}
to the equivalent proof
\begin{display}
$<<3>>14.\ \SUFFICES \begin{noj2} 
              \textsc{assume} & \NEW i \in S, \ j \in T\\
              \textsc{prove}  & Q(i)
           \end{noj2}$ \\
\s{2}\fbox{Level-4 proof of \G, using 
           \tlabox{\A\, i \in S: (j \in T) \Rightarrow Q(i)}.}\V{.5}
\fbox{Rest of level-3 proof that proves $Q(i)$ using assumptions 
             $i\in S$ and $j \in T$.}\s{-5}
\end{display}
The proof that \tlabox{\A\, i \in S: (j \in T) \Rightarrow Q(i)}
implies \G\ is usually a simple leaf proof; the proof of $Q(i)$ is often
complicated, requiring multiple levels.  Therefore, the
\textsc{suffices} proof usually has one fewer level.  The main
function of the \textsc{suffices} construct is to reduce the depth of
proofs.


\item[\rm\textsc{case} $F$] 
 \ctindex{1}{case (proof step)@\icmd{textsc}{case} (proof step)}{case-pf-step}%
\mbox{}\\
This step is equivalent to the step 
 \[ \ASSUME F \ \PROVE \G  \]
where \G\ is the current goal at the step.

\item[\rm$\textsc{pick} \ i \in S, \ j \in T : P(i, j)$] 
 \ctindex{1}{pick (proof step)@\icmd{textsc}{pick} proof step}{pick-pf-step}%
\mbox{}\\
%
This step produces the same proof state at the next statement at the
same level as the step
 \[ \SUFFICES \begin{noj2} 
              \textsc{assume} & \NEW i \in S, \ \NEW j \in T\\
                              & P(i, j) \\ 
              \textsc{prove}  & \G
           \end{noj2}
 \]
having the same prefix token as the \textsc{pick}
step, and where \G\ is the current goal at the step.  It produces the same
state at the beginning of the step's proof as the step
 \[ \E\, i \in S, \, j \in T : P(i, j)
 \]
Thus, to prove the step, you have to prove the existence of $i\in S$
and $j\in T$ satisfying $P(i,j)$.  After the step, $i$ and $j$ are
declared to be constants, with domain assumptions $i\in S$ and $j\in
T$, formula $P(i,j)$ is added to the known facts, and the current goal
remains the same.  (As with the corresponding assume/prove step, whether
$P(i,j)$ is usable depends on whether the step is named.)

In general, a \textsc{pick} statement can be anything that is a
legal expression if the \textsc{pick} is replaced by \,$\E$\,---for example:
 \[ \begin{noj} 
    \PICK i,\,j \in S,\; k \in T : P(i,j,k) \V{.3}
    \PICK i, \ j, \ k : (i \notin j) /\ Q(j,k)
    \end{noj}
 \]
The meaning of these statements and their effect on the proof state
should be clear.

\end{description}

\subsubsection{Steps That Cannot Have a Proof}

For a step that cannot have a proof, we need describe only how
it changes the proof state at the step to obtain the proof state
at the following step (which must be at the same level).

\paragraph{The {\rm\textsc{use}} and {\rm\textsc{hide}} Steps}
  \ctindex{1}{use@\icmd{textsc}{use}}{use}%
  \ctindex{1}{hide@\icmd{textsc}{hide}}{hide}%
\mbox{}\V{.2}
%
These two steps modify the sets \FU\ and \DU\ of usable facts and
definitions; a \textsc{use} step can also modify \BB. They can appear
either as a proof step with a preface token, or as a top-level module
statement with no preface token.  When they appear as a step, it makes
no difference whether they are named or not.  A \textsc{use} or
\textsc{hide} step can have a name, but that name can't be used
anywhere.
\begin{description}
\item[\rm \textsc{use} $\langle 2\rangle 2$, $Isa$, $i>1$, $thm$,
                       $SMT$
                          \  \textsc{def} $F$, $\oplus$] 
\marginpar[-1]{The keywords {\rm\textsc{def}}
 and \textsc{defs} are equivalent.} 
      \ctindex{2}{DEF@\icmd{textsc}{def}}{def}%
\mbox{}\\
where $thm$ is a theorem name.  The step adds to \FU\ the formula
$i>1$ and the facts named by the step name $<<2>>2$ and the theorem
name $thm$. It adds to \DU\ the symbols $F$ and $\oplus$, which must
be in $\DK$.  It leaves \G, \Sym, \FK, and \DK\ unchanged, and it
makes \BB\ 
  \marginpar{Remember that the ``facts'' $Isa$ and $SMT$ 
             specify backend provers.}
equal $<<Isa,\,SMT>>$.  (A \textsc{use} step that specifies no backend
provers leaves \BB\ unchanged.)  This step produces the same proof
obligations as the step
 \[ \begin{noj2}
    <<3>>14. & \TRUE \\
             & \s{-2}\BY \langle 2\rangle 2,\ Isa,\ i>1,\  thm,\ SMT
                          \  \DEF F,\ \oplus
    \end{noj2}
 \]


\item[\rm \textsc{hide} $\langle 2\rangle 2$, $thm$
                          \  \textsc{def} $F$, $\oplus$] \mbox{}\\
Removes from \FU\ the facts named by the step name $<<2>>2$ and the theorem
name $thm$.  It removes from $\DU$ the symbols $F$ and $\oplus$.
It leaves \G, \Sym, \FK, and \DK\ unchanged.  
\end{description}
Observe that while a \textsc{use} step can add arbitrary formulas to
\FU, a \textsc{hide} step can remove only named facts from \FU.

\paragraph{The {\rm\textsc{define}} Step} 
      \ctindex{1}{DEFINE@\icmd{textsc}{define}}{define}%
\mbox{}\V{.2}
%
This step makes definitions that are local to the current level of the
proof and its subproofs.
\begin{description}
\item[\rm \textsc{define} $\begin{noj}
                           f(a) \;\deq\; a + 1 \\
                           g \;\deq\; f(42) * b\vs{.4}
                           \end{noj}$] \mbox{}\\
%
adds to \DK\ and \DU\ the symbols $f$ and $g$, which have the
specified definitions everywhere within the scope of the
\textsc{define}---which is the rest of the current proof (and its
subproofs).  The other proof-state components are unchanged.  The
 \marginpar{We may change TLAPS to make the step name refer to
  both definitions when used in 
  the \textsc{def} clause of a \textsc{use} or \textsc{hide} step.}
step may be
named, but its name should not be used.
\end{description}
Observe that, unlike ordinary definitions in the module, definitions
made in a \textsc{define} step are usable by default.  They can
be hidden (removed from \DU) with a \textsc{hide} step.

\paragraph{Other Steps That Cannot Have a Proof} 

\begin{description}
\item[\rm\textsc{witness} $n-2\in Nat$,\, $2*m \in 1\dd n$] 
 \ctindex{1}{witness (proof step)@\icmd{textsc}{witness} proof step}{witness-pf-step}%
\mbox{} \\
For this step to be legal, the current goal \G\ must be 
obviously equivalent to 
\begin{display}
G0. \ $\E\, a \in S, \; b\in T : P(a, b)$
\end{display}
for some identifiers $a$ and $b$, expressions $S$ and $T$, and
operator $P$.  For example, \G\ might be the formula
\begin{display}
G1. \ $\E\, i, j \in Int : i + j \leq 3*(n+1)$
\end{display}
To prove G0, it suffices to prove $P(v, w)$ for particular values $v$
in $S$ and $w$ in $T$.  In our example, $v$ is $n-2$, $w$ is $2*m$,
and $S$ and $T$ both equal $Int$.  The \textsc{witness} step we
would generally use to prove G1 is
  \[ \textsc{witness}\  n-2\in Int,\  2*m \in Int
  \]
I have chosen a different, rather silly example to explain how a 
\textsc{witness} step works in general.  Our example \textsc{witness}
step is equivalent to the step
 \[ \SUFFICES \begin{noj2}
             \ASSUME & n-2\in Nat, \ 2*m \in 1\dd n \\
              \PROVE & (n-2) + (2*m) \leq 3*(n+1)
             \end{noj2}
 \]
with the proof 
 \[ \BY 1\dd n \subseteq Int, \ 2*m \in 1\dd n, \ 
        Nat \subseteq Int, \ n-2 \in Nat 
 \]
Thus, the step changes \G\ to \tlabox{(n-2) + (2*m) \leq 3*(n+1)}.
If the step is named, it adds formulas $n-2\in Nat$ and $2*m \in 1\dd n$
to \FK; if it is unnamed, it adds these formulas to \FK\ and \FU.

There is also an unbounded form (without the $\in$) of the
\textsc{witness} statement:
\begin{display}
\textsc{witness} $n-2$,\, $2*m$
\end{display}
that can be used if the goal is of the form $\E\, a, b : P(a, b)$.
It changes \G\ the same way as the corresponding bounded
\textsc{witness}, but leaves the other state components unchanged.  It
generates no proof obligations.

A \textsc{witness} statement helps the backend provers by explicitly
telling them how to prove an existentially quantified formula.  They
seldom need this help.  The provers will usually deduce G0 by
themselves from the facts $v \in S$, $w\in T$, and $P(v,w)$.  

% With goal G1, the example \textsc{witness} statement can usually be
% replaced by the step


\end{description}
%
The \textsc{have} and \textsc{take} steps that are described next were
added to the language to save some typing.  I never use them,
preferring the equivalent \textsc{suffices} steps.  Readers
encountering \tlaplus\ proofs for the first time can find them
forbidding.  For such readers, it's a good idea to use as few
different kinds of steps as you can.  If you use \textsc{have} and
\textsc{take} steps, it's best to do so only in the lowest-levels
of the proof.
\begin{description}
\item[\rm\textsc{have} $F$] 
 \ctindex{1}{have@\icmd{textsc}{have}}{have}%
\mbox{} \\
where $F$ is an arbitrary formula.  The current goal \G\ must be
of the form $P => Q$, in which case the step is equivalent to
 \[ \SUFFICES \begin{noj2}
              \ASSUME & F \\ \PROVE & Q
              \end{noj2}
 \]
with a leaf proof \textsc{obvious}.  To check this leaf proof, TLAPS
has to prove $P => F$.  This statement is most often used with
$F$ equal to $P$.


\item[\rm\textsc{take} $i, j \in U$, $k\in V$] 
 \ctindex{1}{take@\icmd{textsc}{take}}{take}%
\mbox{} \\
For this statement to be legal, the current goal \G\ must be equivalent
to
 \[ \A\, a, b \in S, \ c \in T : P(a, b, c) \]
In this case, the step is equivalent to the step
 \[ \SUFFICES \begin{noj2} \ASSUME & \NEW i \in U, \ \NEW j \in U, \ 
                                   \NEW k \in V \\
                           \PROVE & P(i, j, k)
              \end{noj2}
 \]
with the leaf proof 
 \[ \BY T \subseteq V,\ S \subseteq U
 \] 
The \textsc{take} step is almost always used with $U=S$ and $V=T$.
In this case, the \textsc{suffices} step can be generated with the
Toolbox's \textsf{Decompose proof} command by selecting the
\textsf{Use suffices} option.  

There is also an analogous unbounded version:
\begin{display}
\textsc{take} $i$, $j$, $k$
\end{display}
It is equivalent to 
 \[ \SUFFICES \begin{noj2} \ASSUME & \NEW i, \ \NEW j, \ 
                                   \NEW k \\
                           \PROVE & P(i, j, k)
              \end{noj2}
 \]
with proof \textsc{obvious}.




\item[An {\rm\textsc{instance}} Statement] 
 \ctindex{1}{instance (proof step)@\icmd{textsc}{instance} (proof step)}{instance-pfstep}%
\mbox{} \\ The
\textsc{instance} step has the same syntax as a module level
\textsc{instance} step (not preceded by $\,\deq\,$).  It leaves the
current goal \G\ unchanged and changes the other components of the
proof state the same way that an ordinary \textsc{instance} statement
in the module does.  TLAPS does not (yet) handle \textsc{instance}
steps.

\end{description}


\subsection{Proof 
Obligations} \xlabel{pf-obligations}

   \tindex{2}{proof obligation}%
  \ctindex{2}{obligation!proof}{obligation-proof}%
  \vspace{-\baselineskip}

\noindent
 Proof obligations are generated by leaf proofs and by the following
kinds of steps that do not take proofs: \textsc{use},
\textsc{witness}, \textsc{take}, and \textsc{have}.  These four kinds
of steps are explained above in terms of equivalent steps with
\textsc{obvious} or \textsc{by} leaf proofs.  The proof obligations
generated by the steps are the ones generated by those leaf proofs.
We therefore need to consider only the proof obligations generated by
an \textsc{obvious} or \textsc{by} leaf proof.  

An \textsc{obvious} proof generates a single proof obligation.  
Suppose the proof context at the proof has these components:
\begin{display}
\begin{describe}{\FU:}
\item[\Sym:] $<<\VARIABLE x, \ \CONSTANT i \in S>>$

\item[\FU:] $\{ v < 0,\ 2*y = 14   \}$

\item[\DU:] $\{ S % = Nat
               , \ w % = i + 2
               , \ y
\}$

\item[\G:] $i + 3 > v + w$
\item[\BB:] $<<Zenon,\ SMT>>$
\end{describe}
\end{display}
where $S$, $y$, and $w$ are defined by:
 \[ \begin{noj3}
    S & \deq & Nat \\
    y & \deq & i-1 \\
    w & \deq & y + 2
    \end{noj3}
 \]
The proof \textsc{obvious} then generates this proof obligation:
 \[ \begin{noj2}
    \ASSUME & \VARIABLE x, \\ & \CONSTANT i \in Nat \\
            & v < 0,\\ & 2*(i-1) = 14  \\
    \PROVE  & i + 3 > v + ((i-1)+2)
    \end{noj2}
\] Note how all occurrences of $S$, $y$, and $w$ have been replaced by
their definitions.  The obligation is sent to $Zenon$ and, if $Zenon$
fails to prove it, it is sent to $SMT$.

To describe the obligations generated by a \textsc{by} proof, we
consider this proof step:
\begin{display}
\begin{tabbing}
$<<3>>8$. \  $\begin{noj2}
              \ASSUME \NEW i \in Nat, \ P(i) \\
              \PROVE  Q(i)
              \end{noj2}$ \\
\s{1}$\BY <<2>>5,\  F > 1,\  Isa,\  2 \oplus 3 = 5,\  <<3>>8,\  SMT,\  G(42),\  Zenon
    \ \DEF F,\  \oplus,\  Q $ 
\end{tabbing}
\end{display}
where step $<<2>>5$ is
\begin{display}
$<<2>>5$. \  $Step2\_5$ 
\end{display} 
for some formula $Step2\_5$.  Step $<<3>>8$ and its proof are then
equivalent to the following steps.  Note how each \textsc{by} fact
that isn't a name of a previously asserted fact must be proved using
the preceding \textsc{by} facts; and the definitions of symbols in the
\textsc{def} clause are expanded in all these proofs.
\begin{display}
\begin{tabbing}
$<<2>>5$. \  \= $"<3>7"$ \kill 
\pushtabs
$<<3>>8$. \  \> $\begin{noj2}
              \ASSUME & \NEW i \in Nat, \ P(i) \\
              \PROVE &  Q(i) \\
              \end{noj2}$ \\
\s{1}\= $<<4>>1.$ \ \= \kill
\> $<<4>>$ \> \textsc{use def} $F,\  \oplus,\  Q$\V{.4}
\> $<<4>>1.$  \> $\begin{noj2}
\ASSUME & \NEW \CONSTANT i \in Nat,\ 
       Step2\_5 \\
\PROVE &  F  >  1 
                       \end{noj2}$ \\
\>\s{1}\textsc{obvious} \V{.4}
\>$<<4>>$ \> $\USE Isa$\V{.4}
\> $<<4>>2.$ \ \> $\begin{noj2}
\ASSUME & \NEW \CONSTANT i \in Nat,\  
       Step2\_5 ,\ 
       F  >  1 \\
\PROVE &  2\oplus 3  =  5                      
                       \end{noj2}$ \\
\>\s{1}\textsc{obvious} \V{.4}
%
\>$<<4>>$ \> $\USE Isa,\ SMT$\V{.4}
%
\> $<<4>>3.$ \ \> $\begin{noj2}
\ASSUME & \NEW \CONSTANT i \in Nat,\ 
       Step2\_5 ,\ 
       F  >  1 ,\ 
       2\oplus 3  =  5 ,\ 
       P(i) \\
\PROVE &  G(42)                       
                       \end{noj2}$ \\
\>\s{1}\textsc{obvious} \V{.4}
%
\>$<<4>>$ \> $\USE Isa,\ SMT,\ Zenon$\V{.4}
%
\> $<<4>>4.$ \ \> $\begin{noj2}
\ASSUME & \NEW \CONSTANT i \in Nat,\ 
       Step2\_5 ,\ 
       F  >  1 ,\ 
       2\oplus 3  =  5 ,\ 
       P(i), \ 
       G(42) \\
\PROVE &  Q(i)                       
                       \end{noj2}$ \\
\>\s{1}\textsc{obvious} \V{.4}
\> $<<4>>4.$ \ \> \QED \V{.4}
\poptabs
$<<3>>$ \> $\USE Isa,\ SMT,\  Zenon$
\end{tabbing}
\end{display}
The \textsc{use def} step is omitted if the \textsc{by} proof has no
\textsc{def} clause.  The other \textsc{use} steps are omitted if
the \textsc{by} facts do not specify any backend prover.  In that
case, the value of \BB\ at step $<<3>>8$ determines the backend
provers used to check the proofs.


\subsection{Further Details}

Here are some miscellaneous facts about proofs and the proof language.

\subsubsection{Additional Language Features}

\paragraph{@ Expressions} 
 \ctindex{1}{+6ea@\mmath{"@} (in proof step)}{+6ea}% "
\mbox{}\\*[.2em]
%
Suppose you want to prove an equality $a > d$ by proving $a\geq b$, $b= c$,
and $c>d$, where $a$, $b$, $c$, and $d$ may be large expressions.
To save some typing, you can write:
 \[ \begin{noj2}
    <<3>>6. \ & a \geq b \\
    <<3>>7. & @ = c \\
    <<3>>8. & @  > d
    \end{noj2}
 \]
In this case, the $@$ in step $<<3>>7$ is an abbreviation for $b$, and
the $@$ in step $<<3>>8$ is an abbreviation for $c$.  The symbol $@$
does not mean $b$ in a proof of $<<3>>7$, nor does it mean $c$ in
a proof of $<<3>>8$.  The symbol $@$ can be used in the same way in
subproofs of those steps' proofs. 

\paragraph{Using Unnamed Facts} \mbox{}\V{.2}
%
The description of TLAPS given thus far makes it seem impossible to
use an unnamed fact such as
 \[ \ASSUME N + 1 > M \]
in a proof.  There is one additional feature of TLAPS that makes it
possible to use such a fact: TLAPS will accept as proved a formula $F$
appearing in a \textsc{by} clause or \textsc{use} statement if $F$ is
identical to a fact in \FK. This allows the use of unnamed facts in a
proof.  For example, the statement adds to \FK\ the fact $N+1>M$.
Thus the assumption above can be used in a proof as follows: \[
\begin{noj}
   <<3>>7a. \ \  2*N + 2 > 2*M \\ 
      \s{1}\BY N + 1 > M
   \end{noj}
 \] 
The proof $\BY 1 + N > M$ will fail, with TLAPS complaining that it
can't prove $1 + N > M$ (assuming that this formula is not implied by
facts in \FU).  To use a formula $F$ this way, $F$ must have the same
parse tree as a formula in \FK.


\paragraph{Implicit Level Specifiers} 
  \ctindex{1}{+4os@\mmath{\icmd{langle}*\icmd{rangle}} (in proofs)}{+4os}%
  \ctindex{1}{+4op@\mmath{\icmd{langle}\!+\!\icmd{rangle}} (in proofs)}{+4op}%
\mbox{}\V{.2}
%
%%%
% The following rewritten 19 Feb 2014 because we decided to
% remove named <*> steps, like <*>42.
% 
% You can avoid writing explicit level numbers in a level
% specifier by using the level specifiers $<<*>>$ and $<<+>>$, whose
% meanings are defined as follows.  
% 
% In a proof with level number $i$, the level specifier $<<*>>$ is
% euivalent to $<<i>>$.  A proof has level number~$i$ if either (i)~it
% begins with a step having the level specifier $<<i>>$; or (ii)~it
% begins with \textsc{proof}~$<<*>>$ and is the proof of a level $i-1$
% step; or (iii) $i=0$, it is the top-level proof of a theorem, and it
% begins with either $<<*>>$ or \textsc{proof}~$<<*>>$.  A proof
% beginning with either $<<+>>$ or \textsc{proof}~$<<+>>$ has the same
% level as if it began with \textsc{proof}~$<<*>>$.
% 
% As an example, here are two equivalent ways to number proof
% steps.
% \begin{display}
% \begin{minipage}[t]{.4\textwidth}
% \begin{tlatex}
% \@x{ {\THEOREM} \.{\dots}}%
% \@x{\@s{12.29} \@pfstepnum{*}{1} .\@s{4.1} \.{\dots}}%
% \@x{\@s{20.07}\@pfstepnum{3}{2} . \@s{4.1} \.{\dots}}%
% \@x{\@s{20.07} \@pfstepnum{*}{3} . \@s{4.1}{\QED}}%
% \@x{\@s{12.29} \@pfstepnum{*}{4} . \@s{4.1}\.{\dots}}%
% \@x{\@s{20.07} \@pfstepnum{+}{5} . \@s{4.1}\.{\dots}}%
% \@x{\@s{20.07} \@pfstepnum{*}{\,6} . \,\@s{4.1}{\QED}}%
% \@x{\@s{28.63} {\PROOF}}%
% \@x{\@s{32.63} \@pfstepnum{*}{7} . \@s{4.1}\.{\dots}}%
% \@x{\@s{32.63} \@pfstepnum{*}{8} . \@s{4.1}{\QED}}%
% \@x{\@s{20.07} \@pfstepnum{*}{9} . \@s{4.1}{\QED}}%
% \@x{\@s{12.29} \@pfstepnum{*}{10} . \@s{2.1}{\QED}}%
% \end{tlatex}
% \end{minipage}
% %
% \begin{minipage}[t]{.3\textwidth}
% \begin{tlatex}
% \@x{ {\THEOREM} \.{\dots}}%
% \@x{\@s{12.29} \@pfstepnum{0}{1} .\@s{4.1} \.{\dots}}%
% \@x{\@s{20.07}\@pfstepnum{3}{2} . \@s{4.1} \.{\dots}}%
% \@x{\@s{20.07} \@pfstepnum{3}{3} . \@s{4.1}{\QED}}%
% \@x{\@s{12.29} \@pfstepnum{0}{4} . \@s{4.1}\.{\dots}}%
% \@x{\@s{20.07} \@pfstepnum{1}{5} . \@s{4.1}\.{\dots}}%
% \@x{\@s{20.07} \@pfstepnum{1}{6} . \,\@s{4.1}{\QED}}%
% \@x{\@s{28.63} {\PROOF}}%
% \@x{\@s{32.63} \@pfstepnum{2}{7} . \@s{4.1}\.{\dots}}%
% \@x{\@s{32.63} \@pfstepnum{2}{8} . \@s{4.1}{\QED}}%
% \@x{\@s{20.07} \@pfstepnum{1}{9} . \@s{4.1}{\QED}}%
% \@x{\@s{12.29} \@pfstepnum{0}{10} . \@s{2.1}{\QED}}%
% \end{tlatex}
% \end{minipage}
% \end{display}
% You can use $<<*>>$ and $<<+>>$ in exactly the same way in preface
% tokens of unnamed proof steps.  You can also use a step name like
% $<<*>>7$ as a fact in a \textsc{by} proof; it 
%   \marginpar{TLAPS currently does not accept step names like
%    $\langle*\rangle7$ in a \textsc{by} proof.}
% is then equivalent to
% $<<i>>7$ where $i$ is the level number of the step being proved by the
% \textsc{by} proof.
% 
% The only reason to use $<<*>>$ and $<<*>>$ is because it makes it
% easier to reorganize a proof, since you can move a step with preface
% token $<<*>>3$ from one proof to another one without having to rename
% it because the proofs are at a different level.  You can
%
You can write a proof consisting entirely of unnamed steps without
writing explicit level numbers.  Just write $<<\!+\!>>$ (typed
\verb|<+>|) as the level specifier of the first step and $<<*>>$
(typed \verb|<*>|) as the level specifiers of subsequent steps.  For
example, here is a possible subproof structure.
\begin{display}
\begin{tlatex}
\@x{\@pfstepnum{4}{2.}\@s{4.1} \.{\dots}}%
\@x{\@s{8.2} {\langle} \.{\!+\!} {\rangle}\@s{4.1} \.{\dots}}%
\@x{\@s{8.2} {\langle} \.{*} {\rangle}\@s{4.1} \.{\dots}}%
\@x{\@s{8.2} {\langle} \.{*} {\rangle}\@s{4.1} \.{\dots}}%
\@x{\@s{21.53} {\langle} \.{\!+\!} {\rangle}\@s{4.1} \.{\dots}}%
\@x{\@s{21.53} {\langle} \.{*} {\rangle}\@s{4.1} \.{\dots}}%
\@x{\@s{21.53} {\langle} \.{*} {\rangle}\@s{4.1} {\QED}}%
\@x{\@s{8.2} {\langle} \.{*} {\rangle}\@s{4.1} \.{\dots}}%
\@x{\@s{8.2} {\langle} \.{*} {\rangle}\@s{4.1} {\QED}}%
\end{tlatex}
\end{display}
You can write \,\textsc{proof}~$<<*>>$\, or \,\textsc{proof}~$<<\!+\!>>$\,
instead of $<<\!+\!>>$.

Each $<<\!+\!>>$ or $<<*>>$ is equivalent to $<<i>>$ for a suitable
level number~$i$.  You can even mix steps labeled $<<*>>$ and steps
labeled $<<i>>$ or $<<i>>j$ in the same proof.  (A little
experimentation will reveal what the appropriate level $i$ is.)
However, this is a bad idea.  You should use $<<\!+\!>>$ and $<<*>>$
only in short, low-level subproofs that need no step names.

\paragraph{Subexpression Names} 
  \ctindex{1}{subexpression names}{subexpression-names}%
\mbox{}\V{.2}
%
When writing proofs, it is often necessary to refer to subexpressions
of a formula.  In theory, one could use definitions to name all these
subexpressions.  For example, if
 \[ Foo(y) == (x + y) + z
 \]
and we need to mention the subexpression $(x+13)$ of $Foo(13)$, we could
write
 \begin{display}
  $Newname(y) == (x + y)$\\
  $Foo(y) == NewName(y) + z$
 \end{display}
This doesn't work in practice because it results in a mass of
non-locally defined names, and because we may not know when we define
the formula which subformulas will need to be mentioned later.

\tlaplus\ provides a method of naming subexpressions of a definition.
If $F$ is defined by $\,F(a, b) \,\deq\, \ldots\,$, then any
subexpression of the formula obtained by substituting expressions $A$
for $a$ and $B$ for $b$ in the right-hand side of this definition has
a name beginning ``$F(A, B)\,!$''.  (Although this is a new use of the
symbol ``!'', it is a natural extension of its use with module
instantiation.)  \popref{subexpressions}{Here is a complete
explanation of subexpression names.}

You can use subexpression names in any expression.  When writing a
specification, you can define operators in terms of subexpressions of
the definitions of other operators.  Don't! Subexpression names should
be used only in proofs.  In a specification, you should use
definitions to give names to the subexpressions that you want to
re-use in this way.


\subsubsection{Importing}

\paragraph{Instantiated Theorems} \target{instantiated-theorems} 
 \ctindex{1}{instance@\icmd{textsc}{instance}!theorems imported by}{instance-thms}%
\mbox{}\V{.2}
%
The statement 
 \[ I == \INSTANCE M \ \WITH \ldots \]
imports definitions and theorems into the current module.  If module
$M$ defines
 \[ D == \psi \]
for some expression $\psi$, then the \textsc{instance} statement
defines $I!D$ in the current module to equal $\overline{\psi}$, which is
the formula obtained from $\psi$ by performing the substitutions for
the \textsc{constant} and \textsc{variable} parameters of $M$
specified by the \textsc{with} clause.  Suppose module $M$ contains
the theorem
  \[ \THEOREM \ Thm == \begin{noj2}
              \ASSUME & A \\
             \PROVE  & \Gamma
              \end{noj2}\]
and that this theorem is preceded in module $M$ by the two assumptions
  \[ \ASSUME B \s{2}\mbox{and} \s{2} \ASSUME C
 \]
The \textsc{instance} statement then imports the theorem $I!Thm$, which
asserts
 \[  \begin{noj2}
              \ASSUME & \overline{B}, \ \overline{C}, \ \overline{A} \\
             \PROVE  & \overline{\Gamma}
              \end{noj2}
  \]
This is the case even if there are additional assumptions following
theorem $Thm$ in module $M$.

Everything works the same if the ``$\,I\deq\,$'' is removed from the
\textsc{instance} statement, except that definitions and theorem names
are imported from $M$ without renaming.

\paragraph{Special Modules} 
  \ctindex{1}{module, special}{module-special}%
\mbox{}\V{.2}
%
There are certain special modules whose defined operators are
treated as if they were built-in operators.  That is,
knowledge about the meanings of those operators are built into the
backend provers.  Putting those operators in the \textsc{def} clause
of a \textsc{use} or \textsc{hide} statement has no effect.
Those special modules are
 \[ Naturals\s{2} Integers\s{2} Sequences\s{2} TLAPS\s{2} TLC \]
Although all the operators defined in the TLC module are treated
by TLAPS like built-in operators, the backend provers have useful
knowledge only about $:>$ and $@@$.


\paragraph{Local Definitions} 
\ctindex{1}{definition!local}{def-local}%
\mbox{}\V{.2}
%
You probably did not realize that \tlaplus\ has \textsc{local}
definitions, and it's unlikely that you will ever have any
reason to use them.  But if you do, here's what you need to know
if your proofs use facts or definitions imported from modules
containing local definitions.  Suppose module $M$ contains
 \[  \begin{noj}
     \LOCAL\ L == 22*i \\
     G == L + 14
     \end{noj}
 \]
In a module that imports $M$, the definition of $G$ can be expanded
in a proof by 
 \[ \USE \DEF G
 \]
(If the module is instantiated with renaming, $G$ is replaced with
something like $I!G$.)  However, the definition of $L$ can't be
expanded because $L$ cannot be referenced in the importing module.
Currently, the definition of $L$ is automatically expanded if module
$M$ is imported with the \textsc{extends} statement.  It is left
unexpanded if $M$ is imported with instantiation.  This may change.

\subsubsection{Recursively Defined Functions and Operators}

A recursive 
  \ctindex{1}{function!recursive definition of!use in proof}{fcn-recursive-def-of-pf}%
  \ctindex{1}{recursive!function definition!use in proof}{recursive-fcn-def-pf}%
function definition is treated as if it were the
equivalent non-recursive definition in terms of \textsc{choose}.  For
example
 \[ fact[n \in Nat] == \IF{n=0} \THEN 1 \LSE n*fact[n-1] 
 \]
is treated as if it were
 \[ fact == \CHOOSE f : 
              f = [n \in Nat |-> \IF{n=0} \THEN 1 \LSE n*f[n-1]]\s{-5}
\]
The library modules $NaturalsInduction$ and $WellFoundedInduction$
provide useful theorems for reasoning about recursively defined
functions.

Recursive operator 
    \ctindex{1}{operator!recursive definition of!use in proof}{op-recursive-def-of-pf}%
    \ctindex{1}{recursive!operator definition!in proof}{recursive-op-def-pf}%
definitions are more problematic.  The statements
 \[ \begin{noj}
    \RECURSIVE Fact(\_) \\
    Fact(n) == \IF{n=0} \THEN 1 \LSE n*Fact(n-1)
    \end{noj}
 \]
are equivalent to a definition
  \[ Fact(n) == \ldots
 \]
whose right-hand side is very complicated and approximately
incomprehensible.  We hope eventually to provide library modules that
make it possible to prove things about recursively defined operators.
For now, operators that are declared in a \textsc{recursive} statement
are treated by TLAPS like declared operators rather than defined
operators.  Their definitions cannot be expanded, and there is no way
to prove anything about them from their definitions.  If you must
use a recursively defined operator like $Fact$ now, you should assume
without proof a theorem like:
 \[ \LEMMA FactDef == \A\, n \in Nat :  
             Fact(n) = (\IF{n=0} \THEN 1 \LSE n*Fact(n-1))\s{-5}
 \]
You should use TLC to check the theorem.  When TLAPS handles
recursively defined operators, you should be able to prove it.


\subsubsection{The Fine Print}

Some of the explanations of proofs and the proof language given above
were not completely accurate.  Here is what was omitted.
\begin{itemize}
\item \FK\ and \FU\ are not really sets of formulas; they are actually
sets of formulas and names of steps and theorems.  For example,
the sequence of steps:
  \[ \begin{noj2}
     <<2>> & x = 2 \\
     <<2>>1.\s{-.3}& x = 2 \\
     <<2>> & \USE <<2>>1
     \end{noj2}
 \] 
adds the formula $x=2$ and the name $<<2>>1$ to \FU.  The step
  \[  <<2>> \HIDE <<2>>1
  \]
removes the name $<<2>>1$ from \FU, but not the formula $x=2$.

The step 
 $\HIDE <<2>>1$ removes that step name from \FU.  It does not remove
the name of a step that names the same formula as $<<2>>1$, nor 
the fact named by $<<2>>1$ if that formula is also in \FU.

\item Examining TLAPS's console output reveals that, in addition to
the proof obligations described above, there are some trivial
obligations that TLAPS proves easily.

\end{itemize}

%try

% The following stuff removed on 19 Feb 2014
% \section{\puce Some Topics about Proofs}
% 
%   \tindex{2}{hypertext proofs}%
%   \vspace{-\baselineskip}%
% \subsection{Reading the Proof as Hypertext}
% 
% 
% 
% 
% 
% \subsection{Writing the Proof with Fewer  Keystrokes}
% 
% Examining \popref{euclid-proof-5}{our proof}, we see that every
% \textsc{by} proof in the proof of $<<1>>2$ uses the fact $<<2>>1$, and
% they all use most of the same definitions.  It is a nuisance
% to keep typing the same facts and \textsc{def} terms.
% 
% The easiest way to avoid having to type the step name $<<2>>1$ is not
% to give that step a name.  If we write only the level number $<<2>>$
% instead of the name $<<2>>1$, then the facts represented by that name
% (formulas $Inv$ and $Next$) are automatically used wherever those
% facts are taken to be true---that is, for the rest of the proof of
% $<<1>>2$.  Try it.  Change $<<2>>1$ to $<<2>>$ (in both cases, the
% following period (``.'') is optional) and remove all uses of $<<2>>1$
% from the \textsc{by} proofs.  The proof will still work.  Try doing
% the same thing with step $<<3>>1$ in the proof of $<<2>>2$, labeling
% the step only with the level number and removing its explicit use in
% the proof of $<<3>>2$.
% 
% 
% Another way to accomplish the same thing is with a \textsc{use}
% statement.  Put back the step name $<<2>>1$, but not any of its uses.
% Immediately after step $<<2>>1$, add the statement%
%   \ctindex{1}{use@\icmd{textsc}{use}}{use}%
%  \[<<2>> \ \USE <<2>>1\]
% This has the effect of automatically using step $<<2>>1$ in the rest
% of the proof of $<<1>>2$.  We could give the \textsc{use} step a name
% instead of just a level.  However, that would change nothing and the
% step name could not be used, so there is no reason to give a
% \textsc{use} step a name.
% 
% A \textsc{use} step can also make definitions usable (that is,
% expanding them) in all the proof obligations generated by the rest of
% the proof of $<<1>>2$.  Change the \textsc{use} statement to
%  \[ <<2>> \  \USE <<2>>1 \ \DEF Inv,\; Next,\; TypeOK,\; GCDInv
%  \]
% delete all the \textsc{def} clauses from the proof of $<<1>>2$
% and check that proof.  It still works.
% 
% Notice that the \textsc{use} step is now colored green, indicating
% that it was proved.  Like a \textsc{by} fact, a \textsc{use} step's
% fact can be any expression, and TLAPS must check that the fact follows
% from the facts that are considered at that point to be true.  However,
% a \textsc{use} step takes no proof, so any fact it asserts must be
% either trivially true (for example, the name of a fact like $<<2>>1$
% or $GCD3$ that is considered true) or else must follow from currently
% usable facts.  The facts in a \textsc{use} statement are usually the
% names of steps or theorems.
% 
% There is also a 
%     \ctindex{1}{hide@\icmd{textsc}{hide}}{hide}%
% \textsc{hide} statement that makes facts and
% definitions unusable.  Its facts can only be the names of steps or
% theorems.  Try proving things after adding and removing step names,
% \textsc{use} statements, and \textsc{hide} statements to make sure you
% understand how these statements work.
% 
% Remember that there are benefits to explicitly stating what facts and
% definitions a proof uses.  It tells a human reader exactly what the
% proof needs, making the proof easier to understand.  It also makes
% mechanical checking of the proof easier, because the prover can search
% for the proof using only what it needs, without getting confused by
% irrelevant facts and formulas made unnecessarily complicated by
% expanding definitions.  For example, TLAPS easily checks the proof of
% the \textsc{qed} step in the proof of $<<2>>2$.  Now add the
% unnecessary facts $GCD1$ and $GDC2$ to the \textsc{by} proof and run
% \textsf{Prove}.  With the computer on which I am writing this, Zenon
% fails and I don't know whether or not Isabelle would eventually
% succeed.
% 
% Expanding definitions has the added cost of making it harder for you
% to read an obligation if its proof fails.  Indeed, if a proof fails,
% it is sometimes helpful to rerun the proof with some definitions
% hidden (by adding a \textsc{hide} statement) just so you can more
% easily see what is being proved.
% 
% You should avoid excessive reliance on implicitly used facts and
% definition.  It is best to restrict them to the lowest-level proofs,
% where the same facts or definitions are likely to be used in most or
% all of the steps.  For example, it would be reasonable to eliminate
% the step names in the two level-3 proofs in the Euclid module.  It
% might also be reasonable to make certain facts, such as a
% type-correctness invariant (and its definition), implicitly usable for
% larger portions of a proof.  This is best done with a \textsc{use}
% statement, so the fact can be hidden if necessary.
% 
% There is one case in which using a fact or definition makes no
% difference to the provers and little difference to a human reader.
% This is the case when a fact is useless without a corresponding
% definition.  For example, in the $Euclid$ proof, using $<<2>>1$ in a
% \textsc{by} or \textsc{use} makes the facts $Inv$ and $Next$ usable.
% However, those facts are useless unless the definitions of $Inv$ and
% $Next$ are usable.  Writing both $<<2>>1$ and ``\textsc{def} $Inv$,
% $Next$'' is essentially redundant.  Expanding the definitions of
% $Inv$ and $Next$ makes no difference if neither $Inv$ nor $Next$
% occurs in the obligation.  Conversely, making $<<2>>1$ implicitly
% usable has only a negligible effect if the definitions of $Inv$ and
% $Next$ are not usable.  (It adds two hypotheses that the provers will
% essentially ignore.)  Therefore, removing the step number $<<2>>1$
% makes it unnecessary to refer to that step explicitly, but does not
% otherwise affect the proof.  Alternatively, you could leave that step
% numbered but add the step
%   \[<<2>>\ \USE \DEF Inv, TypeOK, GCDInv, Next\]
% and eliminate all the \textsc{def} clauses in the proof of $<<1>>2$.
% A human reader will understand what you're doing---especially if you
% add an explanatory comment.
% 
% \medskip
% 
% \tlaplus\ also provides some abbreviations for common idioms that can
% save typing.  
% % 
% % First, we could have written step $<<2>>1$ with just a
% % single assumption as follows, using the same proof as before:
% %  \[ \begin{noj}
% %     <<2>>1.\ \SUFFICES \begin{noj2}
% %             \ASSUME & Inv /\ Next \\
% %             \PROVE  & Inv' 
% %             \end{noj2} \\
% %         \s{1}\OBVIOUS
% %     \end{noj}
% %  \]
% % 
% Proving an implication by a \textsc{suffices}
% assume/prove like step $<<2>>1$
% is very common.  \tlaplus\
% allows us to replace that step and its \textsc{obvious} proof with%
%       \ctindex{1}{have@\icmd{textsc}{have}}{have}%
%  \[ <<2>>1.\ \HAVE Inv /\ Next
%  \]
% In general, if a level-$i$ goal has the form $F => G$, then the
% step 
%  \[ <<i>>j.\ \HAVE F\]
% changes the 
% \marginpar{\popref{proof -note-1}{The fine print.}}
% level-$i$ proof's goal to $G$ and makes $<<i>>j$ refer to
% the fact $F$.  If the step has just a level number, but no name, then
% the fact $F$ is usable by default for the rest of the level-$i$ proof.
% 
% The \textsc{have} statement changes the current goal without
% explicitly stating what the new goal is.  This makes the proof a
% little harder to read.  If the \textsc{have} step starts the proof,
% then the goal is generally easy to see, so this is not a problem.
% However, \textsc{have} is an additional construct that a human reader
% must understand.  You should consider this before using a
% \textsc{have} instead of a \textsc{suffices}
% \textsc{assume}/\textsc{prove} when you are writing a proof for other
% people to read.
% 
% \medskip
% 
% I have already mentioned that the period following a step name (or
% level number) is optional.  You can save a few more keystrokes by not
% using step names for \textsc{qed} steps.  There is no reason to name a
% \textsc{qed} step because that name can never be used.  The use of
% periods after names and of names for \textsc{qed} steps is a matter of
% personal taste.  (Remember that, even if a step is not named, it still
% needs a level number.)
% 
% 
% \subsection*{\puce Using an SMT Solver Back End} \xlabel{euclid-smt}



\newpage
\definecolor{rose}{rgb}{1,.95,.95}
\pagecolor{rose}


\section{\puce The Bounded Buffer Proof}

In 
  \rref{main}{\xlink{main:bbuf-impl-bchan}}{Section~\xref{main:bbuf-impl-bchan}},
we showed informally that the bounded buffer algorithm of module
$PCalBoundedBuffer$ implements the bounded channel specification of
module $PCalBoundedChannel$ under a refinement mapping.  We now examine
the \tlaplus\ proof of that result.  

First, download%
 \marginpar{\popref{downloading-bbuf-spec}{Click here if you have trouble downloading these files.}}
\hyperref{http://research.microsoft.com/en-us/um/people/lamport/tla/hyper-tla/PCalBoundedBuffer.tla}{}{}{the file \texttt{PCalBoundedBuffer.tla}} 
that contains the specification's root module with the proof,
as well as 
  \hyperref{http://research.microsoft.com/en-us/um/people/lamport/tla/hyper-tla/PCalBoundedChannel.tla}{}{}{the file \texttt{PCalBoundedChannel.tla}}
%
that contains an instantiated module.  I suggest that you put these
files in a different folder (directory) than the files by that name
that you have already created, but you can overwrite the existing
files if you wish.  (If you want to overwrite the existing files, make
sure that the Toolbox does not have a specification open that contains
either of those files.)  Open the specification in the Toolbox.

\subsection*{Definitions and Assumptions}
This version of module $PCalBoundedBuffer$ is the same, up through the
statement of the algorithm, as the one you saw in
  \rref{main}{\xlink{main:bbuf-alg}}{Section~\xref{main:bbuf-alg}}.
However, I have given the assumption on line 15 the name $NAssump$ so
it can be used in the proof.

The definitions of $PCInv$, $TypeOK$, and $Inv$ (lines 79--85) 
 \marginpar{\popref{goto-line}{How to go to line~79.}}%
are the same as before, except that I defined $TypeOK$ in terms of the
new state function $BufCtr$ that is defined on line~77.  When defining
formulas to be used in a TLAPS proof, we use auxiliary definitions to
break them into smaller pieces than we would for use in a
specification or a hand proof.  We do this are two reasons:
\begin{itemize}
\item TLAPS proofs are generally pretty long, and using auxiliary
definitions to abbreviate subexpressions can save a fair amount of
typing.

\item Replacing the subexpression $0\dd(2*N-1)$ by the name $BufCtr$
helps the theorem provers, since they then don't have to consider facts
about multiplication, subtraction, 0, 1, 2, or the operator ``$\dd$''
when searching for a proof.
\end{itemize}
The algorithm uses the modulus operator \tlabox{\%}.  To prove
properties of the algorithm, we need to reason about this operator.
The backend provers do not yet know anything about \tlabox{\%}, so we
need to assert some facts about it.  The most elegant way to do this
would be to assert the defining properties of \tlabox{\%} and
\tlabox{\div}, explained in
 \rref{math}{\xlink{math:arithmetic}}{Section~\xref{math:arithmetic}}.
This would lead us to the following axiom:
 \[ \A n \in Int, d \in Nat :\: \{0\} :
     \begin{conj}
       n \,\%\, d \in 0\dd (d-1) \V{.2}
       n \div d \in Int \V{.2}
       n = d * (n \div d) + (n \,\%\,d)
     \end{conj}
 \]
However, I decided that it would be easier to assume Lemma $ModDef$
(line 93).  This lemma should be strong enough to prove the axiom
above from properties of integers if \tlabox{n\div d} were defined to
equal
 \[ \CHOOSE q \in Int : n - (n \,\%\,d) = d * q \]
so it should be the only assumption about \tlabox{\%} that we need.
However, I found that I also needed Lemma $ModMod$ (line 100).
Although $ModMod$ can be proved from $ModDef$ using only the
properties of integers, the proof would be rather tiresome.
Therefore, I decided to assume $ModMod$ as well.  

As I explained \lref{proof:gcd1-3}{when we introduced $GCD1$--$GCD3$}
in the proof of Euclid's algorithm, we should use TLC to check
anything that we assume without proof.  We can check Lemmas $ModDef$
and $ModMod$ the same way we checked $GCD1$--$GCD3$: by substituting 
finite sets of numbers for \tlabox{Int} and \tlabox{Nat :\: \{\}}.

Lemmas $ModDef$ and $ModMod$ should be the only facts that we need to
assume to prove that $Inv$ is an invariant of the algorithm.  However,
a bug in an earlier backend prover that was used for reasoning about
arithmetic prevented it from proving that $2*N$ is an integer.  I got
around that bug by assuming that $2*N$ equals $N+N$.  Because that
prover had trouble handling $2*N$, it seemed best to hide $2*N$ from
it by defining $K$ to equal $2*N$ and to use the definition of $K$
only to prove some simple properties of it.  The expression $2*N$
therefore appears in almost no proof obligations.  The definition of
$K$ is on line 113, and the assertion that it equals $N+N$ appears as
the lemma $KDef$ on line 114, which the $SMT$ prover checks easily.

\subsection*{\puce Navigating the Module}

Before getting into the actual proofs, let's examine the commands
provided by the Toolbox for navigating through specifications and
proofs.  None of the following navigation commands work if the Toolbox
reports a parsing error in the specification.

\subsubsection*{Finding a Definition or Declaration}

The \textsf{Goto Declaration}%
    \ctindex{3}{Goto Declaration (Toolbox command)@\icmd{textsf}{Goto Declaration} (Toolbox command)}{goto-declaration}
command jumps to the definition or
declaration of a symbol.  You select the symbol by putting the
\popref{cursor}{cursor} anywhere in or just to the left of an
occurrence of it or by selecting any portion of the occurrence.  

Select the symbol $Msg$ on \popref{goto-line}{line 81} and execute the
\textsf{Goto Declaration} command---either by right-clicking anywhere
and choosing it from the menu or by hitting the \textsf{F3} key.  This
will jump to and highlight the symbol $Msg$ in the \textsc{constant}
declaration.  Jump back to the $Msg$ on line 81 by executing the
\textsf{Return from Goto Declaration}%
    \ctindex{1}{Return from Goto Declaration (Toolbox command)@\icmd{textsf}{Return from Goto Declaration} (Toolbox command)}{return-from-goto-declaration}
 command---either from the menu
raised by right-clicking or by hitting the \textsf{F4} key.

Another way to jump to a symbol's definition or declaration is to hold
down the \textsf{control} key and move the mouse pointer over the
symbol.  The symbol should become marked as a link---perhaps by
changing color and being underlined.  Clicking on the symbol then
jumps to its declaration or definition.

The \textsf{Show Declarations}%
    \ctindex{1}{Show Declarations (Toolbox command)@\icmd{textsf}{Show Declarations} (Toolbox command)}{show-declarations} 
command produces a pop-up menu with a
list of all symbols that have a meaning in the module, including ones
imported from other modules.  However, the list does not include any
symbols defined in a standard module or any bound identifiers.  The
command can be executed either from the menu raised by right-clicking
or by hitting the \textsf{F5} key.  You can jump to a symbol's
definition or declaration by clicking on its name in the list.  You
can also select the symbol by moving up and down the list with the
arrow keys; hitting the \textsf{Enter} key jumps to the definition or
declaration of the highlighted symbol.  You can also pare the list of
symbols displayed by typing into the menu.  Only symbols that begin
with the letters you have typed (ignoring case) will be shown.  

The \textsf{Return from Goto Declaration} command returns the cursor
to where it was when you last executed a \textsf{Goto Declaration}
or \textsf{Show Declarations} command.


The \textsf{Show Uses}%
    \ctindex{1}{Show Uses (Toolbox command)@\icmd{textsf}{Show Uses} (Toolbox command)}{show-uses}  
command highlights all uses of a symbol and
jumps to the first use.  To execute it, select an occurrence of the
symbol and execute the command either by right-clicking and selecting
it from the menu or by pressing the \textsf{F6} key.  The 
\textsf{Goto Next Use}%
    \ctindex{1}{Goto Next Use (Toolbox command)@\icmd{textsf}{Goto Next Use} (Toolbox command)}{goto-next-use} 
(\textsf{F8}) and \textsf{Goto Prev Use}%
    \ctindex{1}{Goto Prev Use (Toolbox command)@\icmd{textsf}{Goto Prev Use} (Toolbox command)}{goto-prev-use}  
(\textsf{F7})
commands let you cycle through all the uses of the symbol.  Remember
that you can return to the symbol's definition or declaration with the
\textsf{Goto Declaration} command.


Experiment with these commands.  Execute the \textsf{Show
Declarations} command and choose $C!Init$.  This will take you to the
definition of $Init$ in the instantiated module $PCalBoundedChannel$.
See what happens when you execute \textsf{Show Uses} for $Init$.

Return to module $PCalBoundedBuffer$.  Observe that the \textsf{Show
Uses} command also works for bound identifiers such as the symbols $i$
and $j$ in Lemma $ModDef$.  (If you've forgotten where that lemma is,
use the \textsf{Show Declarations} command to find it.)  

Find the uses of Lemma $ModDef$.  Cycle up (\textsf{F7}) to the
next to last use, which is on line~1003.

Observe that a step name such as $<<2>>1$
is considered to be a (non-global) defined symbol, so the \textsf{Goto
Declaration} and \textsf{Show Uses} commands work for it.  

Go to the comment that begins on line~468 and observe that you can
jump to the definition of a symbol even from an occurrence inside a
comment.

Check that you can't go to the definition of any symbol defined in 
a standard module.  However, you can show the uses of such a symbol.
You can show neither definitions nor uses of built-in \tlaplus\
operators like~\tlabox{=>}.

\bigskip \noindent These navigation commands may do strange things if
the module has been modified since it was last saved and parsed.  They
usually act as if a symbol's occurrences are at the location in the
file where they were when the module was last parsed.  Most of the
commands do nothing if the Toolbox reports a parsing error in the
specification.

\subsubsection*{The Proofs}

\bigskip

\bf
The rest of the description of the bounded buffer proof has not
yet been written.  However, comments in the \tlaplus\ files explain
the proof.  You should try reading those files.




\end{document}

%% Stuff from old Euclid proof


We begin our study of \tlaplus\ proofs by proving that the formula
  \[ Safe == (x=y) => (x = GCD(M,N)) \]
is an invariant of
\rref{main}{main:euclid}{Euclid's algorithm}.
%
Recall that Euclid's algorithm is specified in the
  \popref{module-euclid1}{$Euclid$ module}, 
which you should review now.  Here is
  \popref{module-euclid1-ascii}{the \textsc{ascii} version
           of the module}.
Open the specification in the Toolbox.

In discussing 
  \rref{main}{\xlink{main-why-euclid-works}}{why Euclid's algorithm works}, 
we explained that Euclid's algorithm works because it
preserves the invariance of the formula $Inv$.  Since $Inv$ implies
$Safe$, the invariance of $Inv$ implies that $Safe$ is an invariant.

As we observed in our \rref{main}{main:rigorous-inv-proof}{rigorous
invariance proof}, we prove the invariance of $Inv$ by proving
these two formulas:
\begin{enumerate}
\item $Init => Inv$\target{proof:1-3}
\item $Inv /\ Next => Inv'$
\end{enumerate}
To prove the invariance of $Safe$, we then have to prove
\begin{enumerate}
\item[3.] $Inv => Safe$.
\end{enumerate}

\vspace{-1.2\baselineskip}%

\pause\noindent
%
We saw in our informal proofs that, to prove 1--3, we need 
the simple facts \rref{main}{main:gcd1}{GCD1--3} about the gcd.
Here's how we assert them in \tlaplus\ as theorems:%
 \target{proof:gcd1-3}
\begin{display}
 $\THEOREM GCD1 == \A m \in Nat :\: \{0\} : GCD(m, m) = m$\ascii{gcd1-3}\V{.5}
%
 $\THEOREM  GCD2 == \A m, n \in Nat :\: \{0\} : GCD(m, n) = GCD(n, m)$\V{.5}
%
 $\THEOREM  
    GCD3 == \begin{noj}
            \A m, n \in Nat :\: \{0\} : \V{.2} \s{1}
                     (n > m) => (GCD(m, n) = GCD(m, n-m))
            \end{noj}$
\end{display}
Add them to module $Euclid$.

We could prove these facts from simpler facts about numbers.  However,
we are interested in verifying Euclid's algorithm, not facts about
gcd.  We will therefore omit their proofs.  To indicate that we
have not forgotten to write the proofs but are deliberately omitting
them, I like to add the following ``proof'' after each of
these theorems:%
  \ctindex{1}{OMITTED@\icmd{textsc}{omitted}}{omitted}%
  \ctindex{1}{PROOF@\icmd{textsc}{proof}}{proof}%
 \[\PROOF \OMITTED \]
The keyword \textsc{proof} is always optional; wherever it is
permitted, it can be omitted.

It's dangerous to assume facts like $GCD1$--$GCD3$.  Even if the
facts are obvious, it's easy to make a mistake in writing the precise
formulas that express them.  If we make a false assumption, it is in
principle possible to prove anything---including $\FALSE$.
Fortunately, we can let TLC check these formulas as follows.

In the Toolbox, \popref{create-new-model}{create} or
\popref{clone-model}{clone} a new model.  You will have to assign
values to the constants $M$ and $N$ even though we don't care about
them.  Have TLC evaluate the (constant) expression 
  \marginpar{\popref{evaluate-constant-expression}{How to evaluate a constant
     expression with TLC.}}
$GCD1$.  This produces the error%
  \ctindex{1}{is not enumerable TLC error@\icmd{texttt}{is not enumerable} TLC error}{non-enumerable-error}%
\begin{display}
\begin{verbatim}
The `Evaluate Constant Expression� section�s evaluation failed.
Attempted to enumerate S \ T when S:
Nat
is not enumerable.
\end{verbatim}
\end{display}
To evaluate $GCD1$, TLC must evaluate $GCD(m, m)=m$ for all $m$ in the
set $Nat :\: \{0\}$, something it obviously can't do because the set
$Nat$ of natural numbers is infinite.  To check the formula, you have
to tell TLC to override the definition of $Nat$ 
   \marginpar{\popref{definition-override}{How to override a definition
   in TLC.}}
with some finite set of numbers in place of $Nat$.  Have TLC define
$Nat$ to equal $0\dd 50$, and run TLC again.  This time, it will report
that the value of $GCD1$ is $\TRUE$.

You can check $GCD2$ and $GCD3$ in the same way.  However, it's easier
to check all three at once by having TLC check the value of
the following triple:%
 \marginpar{\rref{math}{tuples}{The meaning of a tuple.}}
\begin{twocols}
$<<GCD1, GCD2, GCD3>>$ 
 \midcol 
\verb|<<GCD1, GCD2, GCD3>>|
\end{twocols}
TLC reports the value of this triple to be
 \[<<\TRUE, \TRUE, \TRUE>>\]
so all three formulas are true when $Nat$ is replaced by $0\dd50$.
Thus, TLC has verified that these facts about gcd are true for all
numbers from 0 through 50.  This shows that it's highly unlikely that
we made an error in writing the theorems $GCD1$--3.


\subsection*{The Proof}

This example is very simple--- so simple that it becomes trivial if we
use the full power of the SMT solver back end prover.  We will
therefore write the proof using $SMT$ only when necessary.
  \lref{\xlink{euclid-smt}}{Section~\xref{euclid-smt}}
shows a much simpler proof using an SMT solver back end.

\subsubsection*{The First Try}

Let us now prove \lref{proof:1-3}{formulas 1--3}.  We first assert
their truth by adding the following theorem to the $Euclid$ module.
\begin{twocols}
$\THEOREM \begin{conj}
          Init => Inv \\
          Inv /\ Next => Inv' \\
          Inv => Safe
          \end{conj}$
\midcol
\begin{verbatim*}
THEOREM /\ Init => Inv
        /\ Inv /\ Next => Inv'
        /\ Inv => Safe
\end{verbatim*}
\end{twocols}
The proof (which is not required) comes immediately after the
statement of the theorem.  We know that a proof must use
$GCD1$--$GCD3$.  Let's be optimistic and hope that TLAPS will be able
to prove the theorem by itself from those facts, without having to
decompose the problem.  So, let's try this leaf proof:%
\marginpar[1]{We can write \textsc{proof by}
instead of \textsc{by}.}%
  \ctindex{1}{BY@\icmd{textsc}{by}}{by}%
%  \ctindex{2}{PROOF@\icmd{textsc}{proof}}{proof}%
\begin{twocols}
 $\BY GCD1,\; GCD2,\; GCD3$
 \midcol
\verb|BY GCD1, GCD2, GCD3|
\end{twocols}
Save the module and execute the 
\textsf{Prove} command 
  \marginpar{\popref{prove-command}{How to execute the Prove command.}}
to have TLAPS check this proof.
This raises a window labeled \emph{Interesting Obligations for Euclid.tla}.
It contains a single obligation, with a label that first says
\begin{display} \sf
Obligation 1 - status: zenon : failed
\end{display}
then changes to 
\begin{display} \sf
Obligation 1 - status: isabelle : being proved, zenon : failed
\end{display}
and finally to
\begin{display} \sf
Obligation 1 - status: isabelle : failed, zenon : failed
\end{display}
The Toolbox also colors the theorem approximately pink to indicate
that its proof failed.

Before reading the obligation, click on it.  (Click on the part of the
window containing the formulas.)  Notice that this highlights the word
\textsc{by} in the proof and jumps to it.  If TLAPS were checking a
proof containing more than one leaf proof, this would show you which
one generated the obligation on which TLAPS failed.  Now, we are
checking just a single leaf proof, and there is just a single proof
obligation for TLAPS to check.

TLAPS first sent this obligation to the \emph{Zenon} prover, which
quickly decided it couldn't prove the obligation and failed.  It then
sent the obligation to \emph{Isabelle}, which tried for a while to
prove it and finally decided that it couldn't and failed.  The window
shows the exact obligation that \emph{Zenon} and \emph{Isabelle} were
trying to prove, which is:
\begin{display}
\begin{verbatim}
ASSUME NEW CONSTANT M,
       NEW CONSTANT N,
       NEW VARIABLE x,
       NEW VARIABLE y,
       \A m \in Nat \ {0} : GCD(m, m) = m,
       \A m, n \in Nat \ {0} : GCD(m, n) = GCD(n, m),
       \A m, n \in Nat \ {0} : n > m => GCD(m, n) = GCD(m, n - m)
PROVE  /\ Init => Inv
       /\ Inv /\ Next => Inv'
       /\ Inv => Safe
\end{verbatim}
\end{display}
The provers do not see the specification; this obligation is all
that they see.  In particular, they do not see the definitions
of $Init$, $Inv$, $Next$, and $Safe$.  Those four identifiers
are just meaningless symbols to them.  Thus, there's no way they
could possibly check this obligation.

Definitions are not automatically expanded in obligations.  If you
want the prover to use a definition, you have to explicitly direct it
to do so.  This applies only to ordinary definitions, not named facts.
As you can see, TLAPS automatically expanded the definitions of
$GCD1$--$GCD3$, which are the last three items in the \texttt{ASSUME}.

We tell TLAPS to use the definitions of $Init$, \ldots, $Safe$
by changing the leaf proof to%
%    \ctindex{1}{DEF@\icmd{textsc}{def}}{def}%
 \begin{display}
 $\BY GCD1,\; GCD2,\; GCD3 \ \ \DEF Init,\; Inv,\; Next,\; Safe$
 \end{display}
(Remember that \textsc{by} and \textsc{def} are typed in uppercase
letters.)  Make the change and run the \textsf{Prove} command again.

Expanding those definitions makes the \texttt{PROVE} part of the
obligation 10 lines long, but all this accomplishes is to make the
provers take longer to fail.  You can tell a prover to stop trying to
prove an obligation by clicking on the \textsf{Stop Proving} bar on
the window displaying the obligation.  You can stop an entire proof by
clicking on the \textsf{Cancel} button of the \textsf{Prover Launch}
popup window.  (Sometimes, you will also have to stop one or more
proofs that are in progress with their \textsf{Stop Proving} bars.)

If you look closely at the \texttt{PROVE} part of the obligation, you
will see that it contains the identifiers $TypeOK$ and $GCDInv$.  You
can add $TypeOK$ and $GCDInv$ to the \textsc{def} clause of the leaf
proof and try again.  However, \emph{Zenon} will still fail quickly and
\emph{Isabelle} will spend a very long time trying to prove it.  If a prover
spends more than 30 seconds trying to prove something, giving it more
time usually does not help.

When a prover fails to prove an obligation in a reasonable amount of
time, it may mean that the proof is too hard and you need to break it
into steps.  This is the case now.  
% 
% You might think that it's too hard because the formula to be proved is
% too long.  However, expanding the definition of $TypeOK$ makes the
% \textsc{prove} formula 12 lines long, which is not very long for the
% provers.  We will see exactly why TLAPS needs the proof broken into
% steps.
% 
As you use TLAPS and learn by experience what the provers can and
can't do, you will get pretty good at knowing when you need to break a
proof into steps.  You will then find that when a prover can't find a
proof quickly, it is usually because the obligation cannot be proved.
The most common reason is that you have forgotten to tell TLAPS to use
some fact or definition that it needs.  However, it can be
surprisingly difficult to see what additional fact or definition is
needed.  The quickest way to discover the problem is often to break
the proof into simpler steps.

\subsubsection*{The Level-1 Proof}

We now start decomposing the proof.  The obvious way to do that is to
prove each of the three conjuncts as separate 
%  \ctindex{1}{proof step@proof step\icmd{target}{pfsteptarget}}{proof-step}%
  \ctindex{1}{step!proof|see{proof step}}{step-proof}%
steps.  The structure of
a formal \tlaplus\ proof is the one introduced in the
\popref{euclid-inv-1}{structured invariance proof of formula~2} that
was developed in
   \rref{main}{\xlink{proving-invariance}}{Section~\xref{proving-invariance}}.
Delete the
\textsc{by} proof that failed and replace it with this proof:
\begin{twocols}
\begin{notla}
<1>1. Init => Inv
<1>2. Inv /\ Next => Inv'
<1>3. Inv => Safe  
<1>4. QED
\end{notla}
\begin{tlatex}
\@x{\@pfstepnum{1}{1.}\  Init \.{\implies} Inv}%
\@x{\@pfstepnum{1}{2.}\  Inv \.{\land} Next \.{\implies} Inv \.{'}}%
\@x{\@pfstepnum{1}{3.}\  Inv \.{\implies} Safe}%
\@x{\@pfstepnum{1}{4.}\  {\QED}}%
\end{tlatex}
\midcol
\begin{verbatim}
<1>1. Init => Inv
<1>2. Inv /\ Next => Inv'
<1>3. Inv => Safe  
<1>4. QED
\end{verbatim}
\end{twocols}
Note that, even though they are printed with angle brackets like the
ones used for tuples, step
  \ctindex{1}{proof step!name}{proof-step-name}%
names like $<<1>>1$ are typed with \verb|<| and \verb|>| rather than
\verb|<<| and \verb|>>|.

When you decompose a proof into a sequence of steps, the first step
you should prove is the \textsc{qed} step.  This is almost always the
easiest step, and there is no point proving the other steps until
you're sure that they're the ones you need.  The proof of this
\textsc{qed} step is quite simple, since the theorem follows immediately
from the first three steps.
\begin{twocols}
\begin{notla}
BY <1>1, <1>2, <1>3
\end{notla}
\begin{tlatex}
 \@x{\@s{12.29} {\BY}\@pfstepnum{1}{1} ,\,\@pfstepnum{1}{2}
 ,\,\@pfstepnum{1}{3}\ }%
\end{tlatex}
 \midcol
\verb|BY <1>1, <1>2, <1>3|
\end{twocols}
Add this proof right after step $<<1>>4$ and run the \textsf{Prove}
command just on it (with the cursor in the step or its proof).  The
proof will finish quickly and the Toolbox will color
  \ctindex{1}{proof step!green}{pf-step-green}%
step $<<1>>4$ green, indicating that its proof has been
successfully checked.%
\marginpar{\popref{green-meaning}{What does green really mean?}}

Note that the theorem itself is still approximately pink, indicating a
failed proof, even though the proof that failed has been deleted.  The
color 
  \ctindex{1}{proof step!when color changed}{pf-step-color-changed}%
of a step indicates its last proof status reported by TLAPS\@.
If you edit a step or its proof, the color of the step will not
change.  You will have to rerun TLAPS on that step to determine its
current status.

Run the \textsf{Prove} command 
  \popref{prove-command}{on the entire theorem}.  
The theorem and the other three steps will be colored 
  \ctindex{1}{proof step!yellow}{pf-step-yellow}%
yellow,
indicating that there are proofs missing for those steps and for
substeps of the theorem.  Now execute \textsf{Prove} on the entire
module.  This causes theorems $GCD1$--$GCD3$ to be shown in a
different color that indicates that their proofs have been
intentionally omitted.  You can find out what the colors mean, and you
can 
    \ctindex{1}{proof step!colors}{pf-step-change-colors}%
change those colors and their meanings, with the TLAPS
preferences page.  This page is reached by
\textsf{File/Preferences/TLA+ Preferences/TLAPS}\@.  See the Toolbox's
TLAPS help pages for instructions.

TLAPS remembers what obligations it has proved. It never reproves an
obligation it has already proved, unless you tell it to by using the
special 
%    \ctindex{1}{Launch Prover (Toolbox command)@\icmd{textsf}{Launch Prover} (Toolbox command)}{launch-prover}%
\textsf{Launch Prover} command.  You should feel free to run
\textsf{Prove} on a step or an entire theorem even though some of its
substeps have already been proved.  It takes TLAPS very little time to
discover that a step has already been proved.

The 
    \ctindex{1}{Check Status (Toolbox command)@\icmd{textsf}{Check Status} (Toolbox command)}{check-status}%
\textsf{Check Status} command (\textsf{control+g}~\textsf{control+t}) is
like the \textsf{Prove} command, except TLAPS does no proving.  If an
obligation has not changed since TLAPS last tried to prove it, then
TLAPS reports its proof status.  With the default coloring, if a 
  \ctindex{1}{proof step!uncolored}{pf-step-uncolored}%
step
is uncolored after its proof status has been checked, then some leaf
proof in that step's proof has not been checked.

\subsubsection*{The Easy Level-1 Steps}

We now must prove steps $<<1>>1$--$<<1>>3$.  Step $<<1>>1$ is the
easiest, so let's try it first.  It looks like it should follow
immediately from the definitions of $Init$, $Inv$, $TypeOK$, and
$GCDInv$.  So, put this proof immediately after the step:
\begin{display}
\verb|BY DEF Init, Inv, TypeOK, GCDInv|
\end{display}
and run the \textsf{Prove} command on that step.  The proof fails
on the following obligation.
\begin{widedisplay}
\begin{verbatim}
ASSUME NEW CONSTANT M,
       NEW CONSTANT N,
       NEW VARIABLE x,
       NEW VARIABLE y
PROVE  x = M /\ y = N
       => x \in Nat \ {0} /\ y \in Nat \ {0} /\ GCD(x, y) = GCD(M, N)
\end{verbatim}
\end{widedisplay}
Can you see why it failed?  TLAPS has been told to prove that $x=M$
and $y=N$ imply $x$ and $y$ are positive integers.  But it can't
because it knows nothing about $M$ and $N$.  The proof requires our
assumption $MNPosInt$, which asserts that $M$ and $N$ are positive
integers.  
Change the proof of $<<1>>1$ to
\begin{display}
\verb|BY MNPosInt DEF Init, Inv, TypeOK, GCDInv|
\end{display}
TLAPS now proves the step.

Step $<<1>>3$ is the simplest of the remaining two steps, so let's
prove it next.  TLAPS will obviously need to expand the definitions of
$Inv$ and $Safe$.  To fully expand the definition of $Inv$, it needs
also to expand the definition of $TypeOK$ and $GCDInv$.  Try having
TLAPS prove it just by expanding those definitions.  It will fail.
\begin{aproblem}{problem-euclid-step1-1-proof}
Prove step $<<1>>3$ by examining the obligation whose proof failed and
figuring out what else TLAPS needs to use.
\end{aproblem}

\subsubsection*{The Proof of \protect\ensuremath{\langle1\rangle2}}

We now prove $<<1>>2$.  So far, we have been able to get TLAPS to
prove steps by simply telling it what facts and definitions it needs
to use.  That won't work for $<<1>>2$.  When we don't know what
else to do, we decompose the proof into simpler steps.  Step $<<1>>2$
asserts
 $ Inv /\ Next => Inv' $.
The standard way to prove this formula is to assume that $Inv$
and $Next$ are true and to use those assumptions to prove $Inv'$.
Such a proof can be written in \tlaplus\ as:
%   \ctindex{1}{assume (proof construct)@\icmd{textsc}{assume} (proof construct)}{assume}%
%   \ctindex{1}{prove@\icmd{textsc}{prove}}{prove}%
\begin{twocols}
$\begin{noj}
 \begin{noj3}
 <<2>>1. & \ASSUME\!\!& Inv, \; Next \\
          & \PROVE  & Inv' \\
  \mbox{}\\
 <<2>>2. & \QED 
 \end{noj3} \\
 \s{1}\BY <<2>>1
 \end{noj}$
 \midcol \fixverbatim
\begin{verbatim*}
<2>1. ASSUME Inv, Next
      PROVE  Inv'

<2>2. QED
  BY <2>1
\end{verbatim*}
\end{twocols}
Step $<<2>>1$ asserts that, by assuming $Inv$ and $Next$, we can
prove step $<<1>>2$.  This is obvious, so TLAPS easily checks the
\textsc{qed} step's proof.  Inside the proof of $<<2>>1$, we are allowed
to use the assumptions $Inv$ and $Next$.  To use them, we must name them.
Within that proof, $<<2>>1$ names both of these assumptions.

This decomposition is quite simple, but it adds a level to the proof
while accomplishing very little.  The provers know very well that they
can prove
  $ Inv /\ Next => Inv' $
by assuming $Inv$ and $Next$ and proving $Inv'$.  Proving $<<2>>1$
needs yet another level of decomposition.  Complicated proofs
can require quite a few levels of proof; we want to avoid
decompositions that don't do much.

The provers treat the assume/prove of $<<2>>1$
pretty much the same way they treat the implication of $<<1>>2$.  We
could simply rewrite $<<1>>2$ as this assume/prove
and eliminate one level of proof.  However, this would make the
level-1 proof a tiny bit harder to understand, since $<<1>>2$ would no
longer exactly match the second conjunct of the theorem.  

We can avoid the extra level of proof without changing step $<<1>>2$
by changing $<<2>>1$ to%
%    \ctindex{2}{suffices@\icmd{textsc}{suffices}}{suffices}%
\begin{display}
$\begin{noj3}
 <<2>>1.\ \SUFFICES & \ASSUME\!\! & Inv, \; Next \\
                    & \PROVE  & Inv'
 \end{noj3} $
\end{display}
This statement asserts that, to prove the current goal, it suffices to
assume $Inv$ and $Next$ and to prove $Inv'$.  It changes the goal of
this level-2 proof to $Inv'$.  The rest of the level-2 proof can use the
assumptions $Inv$ and $Next$, which together are named $<<2>>1$, just
as if they were inside the proof of the original
assume/prove step.

Let's start writing this level-2 proof.  The \textsc{suffices} step
needs a proof.  It follows immediately from the level-1 goal (the
statement of $<<1>>2$) that assuming $Inv$ and $Next$ and proving
$Inv'$ proves that goal; TLAPS needs no further facts.  The leaf
proof for step $<<2>>1$ is simply 
  \ctindex{1}{obvious@\icmd{textsc}{obvious}}{obvious}%
\textsc{obvious}.
 \marginpar{We can write\\ \s{1}\textsc{proof obvious} \\ instead of
  \textsc{obvious}.}
Every non-leaf proof and subproof needs a \textsc{qed} step.  So, we
start by writing this as the proof of step~$<<1>>2$:
\begin{twocols}
\begin{notla}
<2>1. SUFFICES ASSUME Inv, Next
               PROVE  Inv'
  OBVIOUS

<2>x. QED
\end{notla}
\begin{tlatex}
\@x{\@pfstepnum{2}{1.}\  {\SUFFICES} {\ASSUME} Inv ,\, Next}%
\@x{\@s{68} {\PROVE}\@s{4.1} Inv \.{'}}%
\@x{\@s{8.2} {\OBVIOUS}}%
\par\vspace{8.0pt}%
\@x{\@pfstepnum{2}{x.}\  {\QED}}%
\end{tlatex}
\midcol
\fixverbatim
\begin{verbatim*}
  <2>1. SUFFICES ASSUME Inv, Next
                 PROVE  Inv'
    OBVIOUS

  <2>x. QED
\end{verbatim*}
\end{twocols}
Note that naming proof 
  \ctindex{2}{proof step!name}{pf-step-name}%
steps with consecutive numbers is just a
convention.  A level-2 step name can consist of $<<2>>$ followed by
any nonempty string of numbers and letters.  In fact, levels don't
have to be numbered consecutively; the proof of a level $n$ step can
have any level number greater than $n$.  We'll give the \textsc{qed}
step a nicer name later.  Meanwhile, save the module and run the
\textsf{Prove} command on step $<<2>>1$ (or on the entire module if
you wish) to check that you didn't make a mistake in copying the text.

Replacing the implication by the assume/prove
makes it more convenient to write the proof.  Steps that would
otherwise have the form $Inv /\ Next => F$ become $F$, since
$Inv$ and $Next$ are being assumed.  However, it doesn't help
TLAPS\@. Since it can't prove $<<1>>2$ in a single step, assuming
$Inv$ and $Next$ doesn't enable it to prove $Inv'$ in one step.
The level-2 steps that follow $<<2>>1$ must decompose the proof
of $Inv'$.

Since $Inv$ is a conjunction of two formulas, we could decompose its
proof into two steps, each proving one conjunct.  However, the formula
is so simple that this shouldn't be necessary.  In fact, decomposing
the proof of a conjunction in this way is about the first thing the
provers try.  So we should try something else.  

Examining the assumptions $Inv$ and $Next$, we see that $Next$ is a
disjunction.  A basic method of proving that a disjunction implies a
goal is to prove the goal separately for the cases when each single
disjunct holds.  For the assumption $Next$, this suggests splitting
the proof into two cases: $y > x$ and $x > y$.  This is a good idea
because \emph{Zenon} and \emph{Isabelle} often fail to find proofs that require
a case split.  We express this case split by these two steps, which
should immediately follow step $<<2>>1$.%
  \ctindex{2}{case@\icmd{textsc}{case}}{case}%
  \marginpar{\popref{case-vs-case}{{\rm \textsc{case}} proof steps versus {\rm \textsc{case}}
            expressions.}}%
  \target{case-stmt}
\begin{twocols}
  $<<2>>2.\ \CASE y > x$\\
  $<<2>>3.\ \CASE x > y$
 \midcol
\begin{verbatim*}
  <2>2. CASE y > x
  <2>3. CASE x > y
\end{verbatim*}
\end{twocols}
For any formula $F$, the statement $\CASE F$ is equivalent
to $\ASSUME F\ \PROVE G$, where $G$ is the current goal.  Thus,
step $<<2>>2$ is equivalent to 
 \[\ASSUME y > x\ \PROVE Inv'\]
By its definition, $Next$ implies that one of these two cases holds.
So, we expect the proof of the \textsc{qed} step to be
\begin{twocols}
$\BY <<2>>2,\; <<2>>3 \ \DEF Next$
 \midcol
\begin{verbatim*}
    BY <2>2, <2>3 DEF Next
\end{verbatim*}
\end{twocols}
Add steps $<<2>>2$ and $<<2>>3$ and the proof of the \textsc{qed}
step (which we named $<<2>>x$), and run \textsf{Prove} to check
this proof.  TLAPS fails, reporting the obligation:
\begin{display}
\begin{verbatim}
ASSUME NEW CONSTANT M,
       NEW CONSTANT N,
       NEW VARIABLE x,
       NEW VARIABLE y,
       ASSUME y > x
       PROVE  Inv',
       ASSUME x > y
       PROVE  Inv'
PROVE  Inv'
\end{verbatim}
\end{display}
The assumption $Next$ is not being used because we didn't
say that it should be.  Adding $<<2>>1$ to the facts in the \textsc{qed}
step's \textsc{by} proof fixes that problem.

\subsubsection*{Proving \protect\ensuremath{\langle2\rangle2}}

Let's now prove $<<2>>2$.  Perhaps we've decomposed the proof enough
so we can prove $Inv'$ directly.  We need to expand all the
definitions, and we need all the assumptions we have at this point in
the proof---namely, $Inv$ and $Next$ (which are named by $<<2>>1$)
and $y > x$ (named by $<<2>>2$).  A little thought reveals that to
deduce that $GCD(x', y') = GCD(M, N)$, we will need to use $GCD3$.
So, let's try this proof:
 \begin{display}
   $\BY <<2>>1,\; <<2>>2,\; GCD3 \ \DEF Inv,\; TypeOK,\; GCDInv,\;
     Next$\ascii{euclid-proof-1}
\end{display}
\emph{Zenon} fails quickly, and \emph{Isabelle} spends a long time trying to prove
this obligation:
\begin{display}
\begin{verbatim}
ASSUME NEW CONSTANT M,
       NEW CONSTANT N,
       NEW VARIABLE x,
       NEW VARIABLE y,
       /\ x \in Nat \ {0} /\ y \in Nat \ {0}
       /\ \/ /\ x > y
             /\ x' = x - y
             /\ y' = y
          \/ /\ y > x
             /\ y' = y - x
             /\ x' = x,
       GCD(x, y) = GCD(M, N),
       y > x,
       \A m, n \in Nat \ {0} : n > m => GCD(m, n) = GCD(m, n - m)
PROVE  x' \in Nat \ {0} /\ y' \in Nat \ {0} /\ GCD(x', y') = GCD(M, N)
\end{verbatim}
\end{display}
It won't succeed, so stop the proof.  All the necessary facts are
there, so the obligation should be provable.  The reason \emph{Zenon} and
\emph{Isabelle} don't prove it is that they are very bad at arithmetic.  In
fact, they can make only the most trivial deductions about numbers.

Look at the three conjuncts of the goal.  The first, 
  $x' \in Nat :\: \{0\}$, 
follows immediately from the assumptions, which imply 
  $x \in Nat :\: \{0\}$ and $x'=x$.  
No knowledge of arithmetic is needed to prove that conjunct.  However,
to prove the second conjunct, 
  $y' \in Nat :\: \{0\}$,
we need to deduce 
  $y-x \in Nat :\: \{0\}$
from
  $x \in Nat :\: \{0\}$, $y \in Nat :\: \{0\}$, and $y > x$
We also need 
  $y-x \in Nat :\: \{0\}$
to deduce the third conjunct from $GCD3$.
%
What is less obvious is that we need arithmetic to deduce
 $y'=y-x$ and $x'=x$ as well.
To deduce these formulas, we must know that the second disjunct of
$Next$ has to be true because the assumption $y>x$ implies that $x>y$
is false.

Proving  $<<2>>2$ therefore requires deducing 
  $y - x \in Nat :\: \{0\}$ and  $~(x > y)$.
Let's check that these facts are all that we lack.  Replace the \textsc{by}
proof of $<<2>>2$ with the following, where the proof of 
$<<3>>2$ is the \textsc{by} proof we tried before with the added
fact $<<3>>1$.
\begin{display}
$\begin{noj}
     <<3>>1.\ (y - x \in Nat :\: \{0\}) \; /\ \; ~(x > y) \\
\mbox{}\\
    <<3>>2.\ \QED \\
      \s{1}\BY <<2>>1,\; <<2>>2,\; GCD3,\;  <<3>>1\ \DEF Inv,\; TypeOK,\; 
                GCDInv,\; Next
 \end{noj}$\ascii{euclid-proof-2}
\end{display}
Run TLAPS and check that it proves this \textsc{qed} step.

\begin{question}
Check that TLAPS proves the \textsc{qed} step without using the
\textsc{case} assumption
$<<2>>2$.  Why doesn't it need this assumption?
\end{question}
%
We now have to prove $<<3>>1$.  The default provers, \emph{Zenon} and
\emph{Isabelle}, won't do it.  Instead, we have to tell TLAPS to use a
special prover for arithmetic.  To do this, we first import a
  \ctindex{1}{TLAPS module@\mmath{tlaps} module}{tlaps-module}%
standard module
named $TLAPS$, which we do by changing the $Euclid$ module's \textsc{extends}
statement to
\begin{twocols}
$\EXTENDS Integers,\; TLAPS$
 \midcol
\verb|EXTENDS Integers, TLAPS|
\end{twocols}
The TLAPS module defines 
    \ctindex{2}{SMT@\mmath{SMT}}{SMT}%
$SMT$ to be a ``fact'' that instructs TLAPS to use a backend prover
that is good for reasoning about arithmetic.  The facts that are
needed to prove $<<3>>1$ are (i)~$x$ and $y$ are positive integers and
(ii)~$y>x$.  The first comes from the $TypeOK$ conjunct of $Inv$,
which is one of the assumptions $<<2>>1$; the second is the $<<2>>2$
\textsc{case} assumption.  The proof of $<<3>>1$ is therefore
\begin{display}
$      \BY <<2>>1,\; <<2>>2,\; SMT \ \ \DEF Inv,\; TypeOK$
\ascii{euclid-proof-3}
\end{display}
The $SMT$ prover easily verifies this step.  

\subsubsection*{Proving \protect\ensuremath{\langle2\rangle3}}

Having proved step $<<2>>2$, we just have to prove $<<2>>3$ to 
complete our proof of the theorem.  We would expect this to be
essentially the same as the proof of $<<2>>2$, so let's just copy
that proof and make the obvious changes, to get the following.
(Note that we must replace the use of the \textsc{case} assumption
$<<2>>2$ with $<<2>>3$.)
\begin{display}
$\begin{noj}
     <<3>>1.\ (x - y \in Nat \ \{0\})  \, /\  \, ~(y > x) \\
      \s{1}\BY <<2>>1,\; <<2>>3,\; SMT \ \ \DEF Inv,\; TypeOK\\
    <<3>>2.\ \QED\\
      \s{1}\BY <<2>>1,\; <<3>>1,\; GCD3 \ \ \DEF Inv,\; TypeOK,\; 
          GCDInv,\; Next
 \end{noj}$
\ascii{euclid-proof-4}%%% BAD ASCII
\end{display}
TLAPS proves $<<3>>1$, but \emph{Zenon} fails on $<<3>>2$ and
\emph{Isabelle} spends a long time trying to prove it.  Examine the
obligation and see if you can figure out what the problem is.

In this case, we are trying to use $GCD3$ to prove 
 $ GCD(x - y, y) = GCD(x, y) $.
However, from $GCD3$, we can deduce only
  $ GCD(y, x - y) = GCD(y, x) $.
To deduce 
  $ GCD(x - y, y) = GCD(x, y) $, 
we also need $GCD2$.  So, let's add $GCD2$ to the facts in the 
proof of $<<3>>2$ and try it again.

Once again, \emph{Zenon} fails and \emph{Isabelle} keeps thinking about it.  Stop
the proof; TLAPS needs some help.  Let's tell it to try proving
  $GCD(x - y, y) = GCD(x, y)$
first.  Instead of adding this as another step, we can add it to the
facts in the proof of $<<3>>2$.  A fact in a \textsc{by} proof can be
any formula, not just a named fact.  TLAPS checks all the facts in the
\textsc{by} statement from left to right, using the facts it has
already checked when checking the next one.  (It expands all the
definitions in the \textsc{def} clause first.)  So, put
    $GCD(y, x - y) = GCD(y, x)$
as the last fact.  With this hint, TLAPS proves $<<3>>2$.

Just to see what happens, change the $x-y$ in this formula to $x+y$,
save the module, and try proving the step again.  As expected, this
can't be proved; \emph{Isabelle} will try for a long time to prove it.  Click
on the obligation.  This highlights the formula and jumps to it,
showing you exactly where this obligation came from.

We could change the $+$ back to a $-$ and keep this as the proof.
However, I find the proof easier to read if we make the formula a
separate step.  Put the step right after $<<3>>1$ and give it the name
$<<3>>1a$.  In the proof of $<<3>>2$, replace the formula with
$<<3>>1a$.  (The proof of $<<3>>1a$ requires $GCD3$ but not $GCD2$;
the proof of the \textsc{qed} step requires $GCD2$ but not $GCD3$.)
Run the \textsf{Prove} command on the entire theorem and, if you
haven't made an editing error, TLAPS will prove it.

\medskip

Let's make our proof look a little nicer by removing those inelegant
step names $<<2>>x$ and $<<3>>1a$.  The 
    \ctindex{1}{Renumber Proof (Toolbox command)@\icmd{textsf}{Renumber Proof} (Toolbox command)}{renumber-proof}%
\textsf{Renumber Proof} command renumbers the top-level proof of the
selected step.  (This command can be executed only if the module has
not been modified since it was last successfully parsed.)  Put the
  \popref{cursor}{cursor} in step $<<2>>3$ and execute this command---either by
right-clicking and selecting it or by typing
\textsf{control+g~control+r}.  This renames the level-3 steps with
consecutive numbers and changes references to those names
appropriately.

Since the \textsc{qed} step $<<2>>x$ is not used anywhere, its name is
mentioned only on the statement of the step.  You could therefore
simply replace $<<2>>x$ with $<<2>>4$.  But it's good to get into the
habit of using the \textsf{Renumber Proof} command, so execute it for
step $<<1>>2$ (after saving and reparsing the module) to renumber step
$<<2>>x$.
%
Check that your proof now looks like \popref{euclid-proof-5}{this}.
%\popref{euclid-proof-5-ascii}{ascii here}.
