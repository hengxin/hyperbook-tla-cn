\documentclass[fleqn,leqno]{article}
\usepackage{hypertlabook}
\pdftitle{Is this really a formula?}
\begin{popup}
Actually, $1 + 1/2 + 1/4 + 1/8 + \cdots \;=\; 2$ isn't a real formula
because ``$\cdots$'' isn't a real operation.  This ``formula'' is an
abbreviation for the formula mathematicians usually write:
 \[ \NOTLA \sum_{0}^{\infty} 2^{-i} = 2
   \]
This in turn is an example of informal mathematical notation that can be
challenging to make formal.  A good exercise for mathematically
sophisticated readers is to define an operator $InfiniteSum$ so that
if $F$ is defined by
 \[ F(i) == 2^{i}
\]
then the formula above can be written
  \[ InfiniteSum(F,0) = 2 \]
However, if you're reading this section, you're probably
not ready for such an exercise.
\end{popup}

\makepopup
