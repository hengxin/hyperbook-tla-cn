\documentclass[fleqn,leqno]{article}
\usepackage{hypertlabook}
\pdftitle{Where EXTENDS Goes}
\makeindex
\file{extends-location}
\setpopup{10.3}
\begin{document}

  \ctindex{1}{extends@\icmd{textsc}{extends}!where it goes}{extends}%
  \vspace{-2.2\baselineskip}%
\subsection*{Where an {\rm\textsc{extends}} Statement Goes}

An \textsc{extends} statement must be the first statement in the
module; nothing can come between it and the module opening except for
comments, which are described \tref{main}{main:comments}{later}.  This
means that a module can have at most one \textsc{extends} statement.
That statement can be used to import multiple modules, whose names are
separated by commas.


\end{document}