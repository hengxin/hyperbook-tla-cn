\documentclass[fleqn,leqno]{article}
\usepackage{hypertlabook}
\pdftitle{Proof}
\newcommand{\cset}{\ensuremath{\mathcal{CW}}}
\pflongnumbers
\pflongindent
\beforePfSpace{5pt, 2pt, 2pt}
\afterPfSpace{10pt, 5pt, 2pt}
\interStepSpace{10pt, 5pt, 2pt}

\begin{popup}
\textbf{C1.} If $D$ and $E$ are dominating sets, then $D\subseteq E$
or $E \subseteq D$.
\begin{proof}
%%\setlength{\stepsep}{.35em}
\step{1}{It suffices to assume $D\not\subseteq E$ and $E\not\subseteq D$
         and obtain a contradiction.}
\begin{proof}
\pf\ Obvious.
\end{proof}

\step{2}{Pick $d\in D$ and $e\in E$ such that $d\notin E$ and $e\notin D$.}
\begin{proof}
\pf\ The step 1 assumption implies the existence of $d$ and $e$.
\end{proof}

\step{3}{$d \succ e$  and $e \succ d$}
\begin{proof}
\pf\ Step 2 asserts $d\in D$ and $e\notin D$, which imply $d\succ e$
because $D$ is a dominating set.  Similarly, step~2 and $E$ a
dominating set imply $e\succ d$.
\end{proof}      

\qedstep
\begin{proof}
\pf\ Step 3 and the definition of $\succ$ (which implies
that $d\succ e$ and $e\succ d$ cannot both be true) yield the required contradiction.
\end{proof}
\end{proof}
\end{popup}
\makepopup