\documentclass[fleqn,leqno]{article}
\usepackage{hypertlabook}

\pdftitle{A Better Proof of GCD3}
\begin{popup}
\subsection*{A Better Proof of GCD3}

\begin{proof}
\pflongnumbers
\pflongindent
\beforePfSpace{0pt}
\afterPfSpace{0pt}
\interStepSpace{0pt}
\step{1}{
         \sassume{$m$, $n$, and $d$ are integers}
         \sprove{$d$ divides both $m$ and $n$ iff $d$ divides both 
                $m$ and $n-m$}}
% It suffices to assume that $m$, $n$, and $d$ are integers
% and prove that $d$ divides both $m$ and $n$ iff % \popref{iff}{iff}
% $d$ divides both $m$ and $n-m$.}
\vspace{.201em}
\begin{proof}
%\pf\ By definition of the gcd.
\pf\ Since the gcd of two numbers is the largest integer that
divides both of them, it suffices to show that $m$ and $n$
have the same common divisors as $m$ and $n-m$.
\end{proof}

\vspace{.6em}

\step{2}{\assume{$d$ divides both $m$ and $n$} 
         \prove{$d$ divides both $m$ and $n-m$}} \vspace{.201em}

\begin{proof}
\pf\ That $d$ divides $m$ follows by the assumptions; that it divides
$n-m$ follows from the assumptions and Lemma~Div.
\end{proof}

\vspace{.6em}

\step{3}{\assume{$d$ divides both $m$ and $n-m$}
         \prove{$d$ divides both $m$ and $n$}} \vspace{.201em}
\begin{proof}
\pf\ That $d$ divides $m$ follows from the assumptions; that it divides
$n$ follows from the assumptions, Lemma~Div, and the simple algebraic
relation:
 $n = m + (n-m)$.
\end{proof}

\vspace{.6em}

\qedstep
\vspace{.201em}
\begin{proof}
\pf\ By \stepref{1}, \stepref{2}, and \stepref{3}.
\end{proof}
\end{proof}

\end{popup}
\makepopup