\documentclass[fleqn,leqno]{article}
\usepackage{hypertlabook}
\pdftitle{Cursor versus Mouse Pointer}
\file{cursor}
\makeindex

\begin{popup}

\subsection*{The Cursor versus the Mouse Pointer}

Do not
   \tindex{1}{cursor}%
   \tindex{1}{mouse pointer}%
confuse the cursor with the mouse pointer.  The cursor marks the
position where text that you type is entered into the module.  You can
move it by left-clicking on the mouse or with certain keyboard
commands---for example, by hitting an arrow key.  The mouse pointer
is what moves when you move the mouse around.

\medskip
\noindent
%
What the Toolbox's prover commands prove and where the Toolbox's
comment-making commands make the comments depends on the cursor, not
the mouse pointer.  This can be confusing if you issue those commands
from the menu raised by right-clicking the mouse instead of from the
keyboard.


\end{popup}
\makepopup