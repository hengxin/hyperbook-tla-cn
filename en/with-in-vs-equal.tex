\documentclass[fleqn,leqno]{article}
\usepackage{hypertlabook}
\begin{popup}
\pdftitle{The with Statement: Equals Versus In }
\subsection*{The with Statement: Equals Versus In}

Recall that in the $DieHard$ algorithm, we used
the following \textbf{with} statement that has ``=''
instead of ``$\!\in\!$'':
\begin{display}
\begin{nopcal}
with (poured = Min (big + small, 5) - big) { ... }
\end{nopcal}
\begin{tlatex}
 \@x{ {\p@with} {\p@lparen} poured \.{=} Min ( big \.{+} small ,\, 5 ) \.{-}
 big {\p@rparen} {\p@lbrace} \.{\dots} {\p@rbrace}}%
\end{tlatex}
\end{display}
This statement is equivalent to:
\begin{display}
\begin{nopcal}
with (poured \in {Min (big + small, 5) - big}) { ... }
\end{nopcal}
\begin{tlatex}
 \@x{ {\p@with} {\p@lparen} poured \.{\in} \{ Min ( big \.{+} small ,\, 5 )
 \.{-} big \} {\p@rparen} {\p@lbrace} \.{\dots} {\p@rbrace}}%
\end{tlatex}
\end{display}
Letting $poured$ equal an arbitrary element of a set containing just
one element is equivalent of letting it equal that element.

\end{popup}
\makepopup

