\documentclass[fleqn,leqno]{article}
\usepackage{hypertlabook}
\pdftitle{Proof}
\newcommand{\cset}{\ensuremath{\mathcal{CW}}}
\pflongnumbers
\pflongindent
\beforePfSpace{10pt, 5pt, 2pt}
\afterPfSpace{10pt, 5pt, 2pt}
\interStepSpace{10pt, 5pt, 2pt}
\begin{popup}
\textbf{C2.} $c\succeq^{+}d$ holds,
for any elements $c$ and $d$ of $\cset$.
\begin{proof}
%%\setlength{\stepsep}{.35em}
\step{1}{Let $c$ be an arbitrary element of $\cset$ and define
\[D == \{d \in \cset :~(c\succeq^{+}d)\}\]
It suffices to assume $D$ is nonempty and obtain a contradiction.}
\begin{proof}
\pf\ Obvious.
\end{proof}

\step{2}{$d\succ c$ holds, for all $d \in D$.}
\begin{proof}
\pf\ $~(c\succeq^{+}d)$ implies $~(c\succeq d)$ (because $c\succeq d$
implies $c\succeq^{+}d$ by definition of the transitive closure), and
$~(c\succeq d)$ equals $d \succ c$ by definition of $\succeq$.
\end{proof}

\step{3}{For all $d \in D$ and all $e \in \cset :\: D : d\succ e$.}

\begin{proof}
\step{3.1}{It suffices to assume $d\in D$, $e\in \cset :\: D$, and
$~(d\succ e)$ and obtain a contradiction}
\begin{proof}
\pf\ Obvious.
\end{proof}

\step{3.2}{$c\succeq^{+}e$}
\begin{proof}
\pf\ By the assumption of 3.1, which implies $e\notin D$, and the
definition of $D$.
\end{proof}

\step{3.3}{$e\succeq d$}
\begin{proof}
\pf\ By the assumption of 3.1, which asserts $~(d\succ e)$, and the
definition of $succeq$.
\end{proof}

\qedstep
\begin{proof}
Steps 3.2 and 3.3 and the definition of the transitive closure
imply $c\succeq^{+}d$, which by the definition of $D$ contradicts
the step 3.1 assumption $d\in D$.
\end{proof}

\end{proof}

\step{4}{$D$ is a dominating set.}
\begin{proof}
\pf\ We must prove that if $d\in D$, then $d\succ e$ for any candidate
$e$ not in $D$.  If $e$ is not in $\cset$, this follows because $\cset$ is
a dominating set.  If $e\in \cset$, then $e\in \cset:\: D$ and
this follows from step~3.
\end{proof}

\qedstep
\begin{proof}
\pf\ $D$ is a subset of $\cset$ by definition.  It is a proper subset
of $\cset$ because $c\in \cset$ by the step~1 assumption and step~2
implies $c\notin D$.  Therefore, step~4 implies that $\cset$ is not
the smallest dominating set, which is a contradiction.
\end{proof}
\end{proof}
\end{popup}
\makepopup