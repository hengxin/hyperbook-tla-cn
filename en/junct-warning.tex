\documentclass[fleqn,leqno]{article}
\usepackage{hypertlabook}
\pdftitle{Warning}
\begin{popup}
\subsection*{Warning}

When you get accustomed to writing conjunctions and disjunctions as 
lists like this, it's easy to forget that the last formula in the list
``captures'' tokens in the expression to 
the right of its $/\ $ or $\/ $--even if they occur on a separate line.
For example,
 \[ \begin{noj2}
    /\ & A \\
    /\ & B \\ 
       & \s{1} => C
    \end{noj2}
\]
means $A /\ (B =>C)$, not $(A /\ B) => C$.  You can write the latter expression
in any of the following ways:
 \[  
    \begin{noj2}
    & \begin{noj2}
    /\ & A \\
    /\ & B \\ 
    \end{noj2} \\
    => & C
    \end{noj2}
\s{3}
    \begin{noj}
    \begin{noj2}
    /\ & A \\
    /\ & B \\ 
    \end{noj2} \\
        => C
    \end{noj}
\s{3}
   \begin{noj3}
     (\!\! & /\ & A \\
       & /\ & B\,\,) \\ 
        && \s{1.5} => C
     \end{noj3}
\s{2.5}
\begin{noj3}
     (\!\!& /\ & A \\
       & /\ & B\,\,) \, => \, C
     \end{noj3}
 \]
\vspace*{0pt}
\end{popup}
\makepopup